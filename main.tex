\documentclass{article}
\usepackage[T1]{fontenc}
\usepackage[utf8]{inputenc}
\usepackage[swedish]{babel}
\usepackage[a4paper,margin=2cm,headheight=12pt,includehead,includefoot]{geometry}
\usepackage{fancyhdr}
\usepackage{graphicx}
\usepackage{lastpage}
\usepackage{enumitem}
\usepackage{paracol}
\usepackage{parskip}
\usepackage[hidelinks]{hyperref} %Note: Finns en länk som går till ftek.se Reglemente §12.3.1
\usepackage{xfrac}

\newcommand{\motionsnamn}{Totalreform av styrdokument}
\newcommand{\motionar}{Medlemmar insamlade som ämnar revidera (MISÄR)}

\newlist{beslut}{itemize}{1}
\setlist[beslut]{label=\textbf{att}}
\newlist{dels}{itemize}{1}
\setlist[dels]{label=\emph{dels},noitemsep}
\newlist{aligganden}{itemize}{1}
\setlist[aligganden]{label=\textbf{att}, labelsep=1.7ex, itemsep=-0.5ex, topsep=0.5ex}

\newenvironment{lydelse}
    {\begin{paracol}{2}%
        \emph{Nuvarande lydelse}%
        \switchcolumn%
        \emph{Föreslagen lydelse}%
    \end{paracol}%
    \begin{enumerate}[label=\thesubsection.\arabic*]%
    \begin{paracol}{2}%
    }{\end{paracol}\end{enumerate}}
\newcommand{\itemb}{\item[\textbullet]}

% Unfucka hyperref
\renewcommand{\theHsection}{\Roman{part}.\arabic{section}}

\pagestyle{fancy}
\fancyhead[L]{Motion om \motionsnamn}
\fancyhead[R]{\emph{\motionar}}
\cfoot{\thepage\ (\pageref*{LastPage})}

\begin{document}
\begin{center}
%\textsc{\huge\itshape Betänkande}\\[0.5cm]
\textsc{\Huge Motion om}\\[0.5cm]
\textsc{\huge \motionsnamn}\\[0.5cm]
\textsc{\large \motionar\\}
Ruben Seyer (föred.),
%Felix Augustsson,
Erik Broback,
Alexandru Golic,\\
Linnea Hallin,
Joseph Löfving,
Albert Vesterlund,
Tobias Wallström,
David Winroth
\end{center}

\section*{Bakgrund}
Sektionens styrdokument är i grunden en sammansättning av varierande förlagor från kåren och andra sektioner som genom åren lappats och lagats med diverse mindre ändringar, tills de i nuläget innehåller flertalet redundanser, otydligheter och föråldrade bestämmelser.

Vårt mål är att genom en total genomgång av gällande dokument ge styrdokumenten enhetligt språk, reda ut oklarheter, kodifiera praxis och effektivisera sektionens arbete.

Syftet är inte att i grunden bygga om sektionens funktion, utan att uppdateringen ska leda till att dokumenten bättre motsvarar den verklighet som i dag råder på sektionen.

\section*{Sammanfattning}
En sammanfattning av de väsentliga ändringarna jämfört med tidigare dokument är
\begin{itemize}
    \item Före detta särskild ledamot erhåller rösträtt (och ska likställas med ordinarie medlem).
    \item Inspektor omfattas nu av offentlighetsprincipen.
    \item Sektionsordförande och SAMO är ej längre separata organ.
    \item Praxis kodifieras angående frågor om mandat, bl.a. valbarhetshinder, avsägelser och entlediganden. Regelverket i dessa frågor har väsentligen tillämpats sedan 20/21.
    \item Bestämmelser om misstroendeförklaring förtydligas, men ska i allt materiellt tolkas lika.
    \item Avsnittet om sektionsmöten städas upp och regler om personval flyttas till reglemente.
    \item En långtgående elimination av redundans i definitionen av organen,
      särskilt mellan stadga och reglemente. En konsekvens är att
      sektionsföreningar också ska representeras på sektionsmöte.
    \item En konsolidation av bestämmelser om ekonomi för sektionen och dess organ, och tydliggörande av frågan verksamhetsår/räkenskapsår.
    \item Förtydligande görs kring stadgeändringar och reglementesändringar för att lösa frågeställningar kring ändringsyrkanden. Dessutom utökat skydd för vissa beslut i kallelse, t.ex. fonduttag.
    \item Antalet verksamhetsplaner och -berättelser som presenteras för sektionsmötet minskas kraftigt eftersom detta inte fungerat på den nivå som tänktes under många år; det är bättre att vi kontrollerar det som det finns intresse av att kontrollera. I stället fastställs alla utom de som rör styrelsen och SNF av styrelsen. De blir fortfarande tillgängliga för alla medlemmar genom styrelseprotokoll.
    \item Möjligheten till reducerat antal poster i organ tas bort. Poster som ej tillsätts måste vakantsättas. 
    \item Extremt många redundanta åligganden i reglemente stryks.
    \item Styrelsens åläggs anslå väsentliga ändringar i policydokument eftersom de blir gällande för alla medlemmar.
    \item Sektionens regelverk kring sekretess konsolideras. Styrelsens och sektionsaktiv befattning med sekretessbelagd uppgift förydligas.
    \item Reglementeskapitlet om kommittéoverall stryks och blir policy.
\end{itemize}    

\section*{Yrkanden}
\begin{beslut}
	\item sektionsmötet antar förslag till reviderad stadga.
	\item sektionsmötet antar förslag till reviderat reglemente, som träder i kraft samtidigt som reviderad stadga.
	\item talmanspresidiet åläggs beakta revisionen i förslaget till ny sammanträdesordning i läsperiod 1.
	\item sektionsstyrelsen åläggs uppdatera arbetsbeskrivningar tills det att reviderad stadga träder i kraft.
	\item sektionsstyrelsen åläggs ta fram policy om kommittéoveraller.
	%\item sektionsstyrelsen åläggs göra övriga ändringar i styrdokument tills det att reviderad stadga träder i kraft.
	\item sektionsmötet beslutar om följande övergångsbestämmelser:
	\begin{itemize}
    \item alla särskilda ledamöter utnämns till särskilda medlemmar då reviderad stadga träder i kraft,
    \item alla sektionsaktiva valda till post de ej längre är valbara till enligt avsnitt \ref{4.x:tillsättande} tillåts kvarstå till mandatperiodens slut.
	\end{itemize}
\end{beslut}

\vspace{10ex}

Det är vår stora ära att härmed få överlämna motionen om totalreform av styrdokument.

Göteborg den \today

% SIGNATURER HÄR ;)


\clearpage
\tableofcontents %Note: Den är fucked pga två dokument (stadga & reglemente) som har samma numrering
\clearpage

%%%%%%%%%%%%%%%%%%%%%%%%%%%%%%%%%%%%%%%%%%

\part{Stadga}
\section{Inledande bestämmelser}
Härigenom föreskrivs i fråga om stadgans kapitel 1 att kapitlets rubrik ska lyda ''Inledande bestämmelser''.


\subsection{Ändamål}
Syftet är att förnya språket och ta bort redundans.

Härigenom föreskrivs i fråga om stadgans avsnitt 1.1 att \S 1.1.1 ska ha följande lydelse.

\begin{lydelse}
    \item Fysikteknologsektionen, benämns nedan som sektionen, vid Chalmers studentkår är en ideell förening bestående av studerande vid utbildningsprogrammen Teknisk fysik eller Teknisk matematik vid Chalmers tekniska högskola och av studenter vid därtill associerade mastersprogram som betalat sektionsavgift till densamma.
  \switchcolumn
    \item Fysikteknologsektionen \emph{vid Chalmers studentkår, nedan benämnt sektionen}, är en ideell förening \emph{för studenter} vid utbildningsprogrammen Teknisk fysik, Teknisk matematik \emph{och därtill associerade masterprogram} vid Chalmers tekniska högskola.
\end{lydelse}

\subsection{Verksamhet}
Syftet är dels att förtydliga hur begreppet verksamhetsår används, dels att tillfoga styrelsens säte vilket är allmän föreningsformalia som saknas.

Härigenom föreskrivs i fråga om stadgans avsnitt 1.2
\begin{dels}
    \item att avsnittets rubrik ska lyda ''Verksamhet'',
    \item att avsnittet ska ha följande lydelse.
\end{dels}

\begin{lydelse}
    \item Sektionens verksamhetsår löper från 1 juli till 30 juni.
    \item[] ---
  \switchcolumn
    \item Sektionens \emph{räkenskapsår} löper från 1 juli till 30 juni. \emph{Ett organ på sektionen får ha annat verksamhetsår enligt reglemente.}
    \label{S:Verksamhetsar}
    \item \emph{Sektionsstyrelsen ska ha sitt säte i Göteborg.}
\end{lydelse}

\subsection{Skyddshelgon}
Syftet är att flytta bestämmelserna om skyddshelgon, som inte kan anses motivera ett eget kapitel.

Härigenom föreskrivs i fråga om stadgans kapitel 17
\begin{dels}
    \item att kapitlet ska omformas och flyttas till avsnitt 1.3,
    \item att det nya avsnittet i sin helhet ska ha följande lydelse.
\end{dels}

\begin{lydelse}
  \setcounter{section}{17}
  \setcounter{subsection}{1}
    \item Sektionens skyddshelgon är Dragos.
    \item Sektionens medlemmar vördar Dragos.
  \setcounter{section}{1}
  \setcounter{subsection}{3}
  \switchcolumn   
  \setcounter{enumi}{0}
    \item Sektionens skyddshelgon är Dragos.
    \item Sektionens medlemmar vördar Dragos.
\end{lydelse}

\section{Medlemskap}
Härigenom föreskrivs i fråga om stadgans kapitel 2 att kapitlets rubrik ska lyda ''Medlemskap''.

\subsection{Allmänt}
Syftet är att konsolidera allmänna bestämmelser.
Efterkommande ändringar i kapitlet påverkar detta avsnitt.

Härigenom föreskrivs i fråga om stadgans avsnitt 2.1 och 2.4
\begin{dels}
    \item att rubriken för avsnitt 2.1 ska lyda ''Allmänt'',
    \item att avsnitt 2.4 ska utgå,
    \item att avsnitt 2.1 i sin helhet ska ha följande lydelse.
\end{dels}

\begin{lydelse}
    \item Medlem i sektionen är kårmedlem som studerar vid utbildningsprogrammen för Teknisk fysik eller Teknisk matematik vid Chalmers tekniska högskola, som betalt sektionsavgift. Dessutom är kårmedlemmar som studerar vid programmen associerade mastersprogram, som betalt sektionsavgift, medlemmar. Därutöver kan sektionen ha hedersmedlemmar, seniormedlemmar, phatri\-arker/\-mathri\-arker, och särskild ledamot.
    
  \switchcolumn
  \setcounter{enumi}{0}
    \item \emph{Sektionen har ordinarie medlemmar}, hedersmedlemmar, seniormedlemmar, \emph{alumner} och särskilda \emph{medlemmar}.
    
  \switchcolumn*
  \switchcolumn
    \item \emph{Medlem som ska erlägga sektionsavgift, eller summa motsvarande denna, gör detta terminsvis.}

  %\switchcolumn*
  %  (\S 2.2.1, \S 2.5.4, \S 2.6.2, \S 2.7.2, \S 2.8.2)
  %\switchcolumn
  %  \item Medlem \emph{har} närvaro-, och yttranderätt på sektionsmöte.
  %    \label{2.1:närvaro} 

  \switchcolumn*
    \item[] (\S 2.2.1, \S 2.2.2, \S 2.5.4, \S 2.6.2, \S 2.7.2, \S 2.8.2)
    \item[] (\S 2.2.4, \S 2.2.5, \S 2.8.2)
  \switchcolumn
    \item \emph{Rättigheter tillkommer enligt detta kapitel, \S\ref{5.x:rösträtt}, \S\ref{5.x:grundrätt}, \S\ref{7.x:SNFrätt}, \S\ref{11.x:intrrätt} och \S\ref{15.x:offentlighet}.} \label{2.x:allarätt} 
    
  \switchcolumn*
    (\S 2.3.1, \S 2.5.5, \S 2.6.3, \S 2.7.3, \S 2.8.3)

  \switchcolumn
    \item Medlem är skyldig att rätta sig efter sektionens stadgar, regle\-mente, beslut samt övriga styrdokument.
      \label{2.1:skyldig}
    
  \switchcolumn*
  \setcounter{subsection}{4}
  \setcounter{enumi}{0}
    \item Sektionens förtroendeposter avser poster listade i reglementet som förtroendeposter.
    
    \item Sektionsaktiv är person vald till post av sektionsmötet eller sektionsstyrelsen.
    
  \switchcolumn
    \emph{(ersätts av \S \ref{4.x:fortroende}, \S \ref{4.x:aktiv})}

  \switchcolumn*
  \setcounter{subsection}{1}
  \setcounter{enumi}{1}
    \item Person sittandes på en post inom sektionen och vars medlemskap avslutas behöver sektionsstyrelsens godkännande för att preliminärt sitta kvar.
    Beslutet fastställs på det kommande sektionsmötet.

  \switchcolumn
    \emph{(ersätts av \S\ref{4.x:kvarstå})} %Person \emph{vald till} post inom sektionen och vars medlemskap avslutas behöver sektionsstyrelsens godkännande för att sitta kvar.
    %Beslutet fastställs på \emph{nästa} sektionsmöte.
\end{lydelse}

\subsection{Ordinarie medlemmar}
Syftet är att ta bort tvetydigheten i begreppet medlem genom att införa begreppet \emph{ordinarie medlem}, så att begrepp medlem kan reserveras för den generella bemärkelsen.
Dessutom görs strukturen enhetlig med övriga typer av medlem.
Konceptet motionsrätt anses redundant och helt täckt av förslagsrätten.

Härigenom föreskrivs i fråga om stadgans avsnitt 2.2 och 2.3
\begin{dels}
    \item att rubriken för avsnitt 2.2 ska lyda ''Ordinarie medlemmar'',
    \item att avsnitt 2.3 ska utgå,
    \item att avsnitt 2.2 i sin helhet ska ha följande lydelse.
\end{dels}

\begin{lydelse}
    (se \S 2.1.1)

  \switchcolumn
    \item \emph{Ordinarie} medlem i sektionen är kårmedlem som studerar vid \emph{sektionens program och har} betalt sektionsavgift.

    \subsubsection*{Rättigheter}
  \switchcolumn*
  \setcounter{subsection}{2}
    \item Varje medlem har närvaro-, yttrande-, förslags- och rösträtt på sektionsmöte.
   
    \item Varje medlem har motionsrätt till sektionsmöte
    
  \switchcolumn
    \item[] \emph{(ersätts av \S\ref{2.x:allarätt})} 
    %\item \emph{Ordinarie} medlem \emph{har dessutom} förslags- och rösträtt på sektionsmöte.

  \switchcolumn*

    \item Medlem är valbar till post inom sektionen.

    \item Medlem har rätt till medlemskap i sektionens intresseföreningar.

    \item Medlem har rätt att ta del av mötesprotokoll och sektionens övriga handlingar, undantaget de dokument som listas som icke offentliga i reglementet.
    
    \item Medlem har rätt att utnyttja av sektionen erbjudna tjänster.
    
  \switchcolumn
    
    \item \emph{Ordinarie} medlem är valbar till post inom sektionen.
    \label{S:OrdinarieMedlem}
    
    \item[] \emph{(ersätts av \S\ref{2.x:allarätt})}
    
    \item \emph{Ordinarie} medlem har rätt att \emph{nyttja} av sektionen erbjudna tjänster.
    
    %\subsubsection*{Skyldigheter}
    
  \switchcolumn*
  \setcounter{subsection}{3}
  \setcounter{enumi}{0}
    \item Medlem är skyldig att rätta sig efter sektionens stadgar, regle\-mente, övriga beslut samt Fysikteknologsektionens övriga styrdokument.
    
  \switchcolumn
    \emph{(ersätts av \S \ref{2.1:skyldig})}
\end{lydelse}

\setcounter{subsection}{2}
\subsection{Hedersmedlemmar}
Syftet är att numrera alla punkter och uppdatera bestämmelser som försummats vid tidigare revisioner.
Dessutom görs bestämmelser om arrangemangsrätt enhetliga.

Härigenom föreskrivs i fråga om stadgans avsnitt 2.5
\begin{dels}
    \item att avsnittet omnumreras till 2.3,
    \item att avsnittet i sin helhet ska ha följande lydelse.
\end{dels}

\begin{lydelse}%
    \subsubsection*{Grundkrav}
    \itemb Till hedersmedlem kan kallas nu levande person som främjat F- eller TM-tekno\-loger, sektionen, ämnesområdet fysik eller matematik eller på annat sätt till\-för\-skansat sig F- eller TM-tekno\-logens vördnad och respekt.
    \item[]
    
\switchcolumn
    \setcounter{subsection}{3}
    \item Sektionens hedersmedlemmar \emph{förtecknas} i reglemente.
    
    \item Till hedersmedlem kan kallas nu levande person som främjat sektionen, \emph{teknologerna vid dess program}, ämnesområdet fysik eller matematik\emph{;} eller på annat sätt till\-för\-skansat sig \emph{sektionsmedlemmarnas} vördnad och respekt.
    
\switchcolumn*
    \subsubsection*{Förslag och kallande}%
    \itemb Förslag till hedersmedlem lämnas skriftligt till sektionsstyrelsen med minst 25 namnunderskrifter från sektionsmedlemmar.

    \itemb Ärendet ska tas upp på nästa sektionsmöte. Beslut om kallande skall bifallas med minst 2/3 majoritet.
    
    \itemb Vid bifall kallas personen till nästa sektionsmöte där val förrättas.

    \itemb Personen skall närvara vid valet, eller ha inkommit med skriftligt bifall.

    \itemb Beslut om inval skall bifallas med minst 2/3 majoritet.
    
\switchcolumn
    \subsubsection*{Förslag och kallande}%
    
    \item Förslag till hedersmedlem lämnas \emph{skriftligen} till sektionsstyrelsen \emph{och talmanspresidiet} med minst 25 namnunderskrifter från medlemmar.

    \item Ärendet ska tas upp på nästa sektionsmöte\emph{, dock tidigast 6 läsdagar efter att förslaget inkommit}.
    Beslut om kallande skall \emph{fattas} med minst \sfrac{2}{3} majoritet. \label{maj:hm1}
    
    \item Vid bifall kallas personen till nästa sektionsmöte\emph{, adjungerad med närvaro- och yttranderätt, där} val förrättas.

    \item Personen ska närvara vid valet, eller ha inkommit med skriftligt bifall.

    \item Beslut om inval ska \emph{fattas} med minst \sfrac{2}{3} majoritet. \label{maj:hm2}
    
\switchcolumn*
    \subsubsection*{Förteckning}%
    \itemb Sektionens hedersmedlemmar är listade i reglementet.
    
\switchcolumn
\switchcolumn*
    \subsubsection*{Hedersmedlemmars rättigheter}%
    \itemb Hedersmedlem har närvaro- och yttranderätt på sektionsmöte.
    
    \itemb Hedersmedlem har närvaro- och intaganderätt på alla sektionens
arrangemang.
    
\switchcolumn
    \subsubsection*{Rättigheter}%
    \emph{(ersätts av \S \ref{2.x:allarätt})}
    
    \item Hedersmedlem har närvaro- och intaganderätt på alla sektionens arrangemang \emph{öppna för samtliga medlemmar}.

%TODO \clearpage
\switchcolumn*
    \subsubsection*{Hedersmedlemmars skyldigheter}%
    \itemb Hedersmedlem är skyldig att rätta sig efter sektionens stadgar, regle\-mente, övriga beslut samt  Fysikteknologsektionens övriga styrdokument.
    
\switchcolumn
    \emph{(ersätts av \S \ref{2.1:skyldig})}
\end{lydelse}

\subsection{Seniormedlemmar}
Syftet är att numrera alla punkter och uppdatera bestämmelser som försummats vid tidigare revisioner.

Härigenom föreskrivs i fråga om stadgans avsnitt 2.6
\begin{dels}
    \item att avsnittet omnumreras till 2.4,
    \item att avsnittet i sin helhet ska ha följande lydelse.
\end{dels}

\begin{lydelse}%
    \subsubsection*{Definition}
    \itemb F- eller TM-teknolog har rätt att efter avslutade eller definitivt avbrutna studier skriftligen ansöka om seniormedlemskap av sektionsstyrelsen som därefter fastställer seniormedlemskap. Seniormedlem kvarstår som senior\-med\-lem så länge en summa motsvarande sektionsavgiften skänks till sektionen.

\switchcolumn
  \setcounter{enumi}{0}
    \item \emph{Seniormedlem är person utnämnd efter ansökan som har skänkt en summa motsvarande sektionsavgiften till sektionen.}
    
    \subsubsection*{Ansökan}
  
    \item \emph{Teknolog vid sektionens program} har rätt att efter avslutade eller definitivt avbrutna studier ansöka om seniormedlemskap.
    \emph{Ansökan lämnas skriftligen till sektionsstyrelsen.}
  
    \item \emph{Sektionsstyrelsen utnämner seniormedlemmar efter ansökan.}
    
\switchcolumn*
    \subsubsection*{Seniormedlemmars rättigheter}%
    \itemb Seniormedlem har närvaro- och yttranderätt på sektionsmöte.

    \itemb Seniormedlem har närvaro- och intaganderätt på alla sektionens arrangemang som är öppna för samtliga sektionens medlemmar.

    \itemb Seniormedlem har rätt att utnyttja av sektionen erbjudna tjänster.
   
    \itemb Seniormedlem är valbar till post inom sektionen.
    
\switchcolumn
    \subsubsection*{Rättigheter}%
    \emph{(ersätts av \S \ref{2.x:allarätt})}
    
    \item Seniormedlem är valbar till post inom sektionen.
    
    \item Seniormedlem har rätt att utnyttja av sektionen erbjudna tjänster.
   
    \item Seniormedlem har närvaro- och intaganderätt på alla sektionens arrangemang öppna för samtliga medlemmar.

  \switchcolumn*
    \subsubsection*{Seniormedlemmars skyldigheter}%
    \itemb Seniormedlem är skyldig att rätta sig efter sektionens stadgar, regle\-mente, övriga beslut samt  Fysikteknologsektionens övriga styrdokument.
    
  \switchcolumn
    \emph{(ersätts av \S \ref{2.1:skyldig})}
\end{lydelse}

\subsection{Alumner}
Syftet är att numrera alla punkter och uppdatera bestämmelser som försummats vid tidigare revisioner.
Vidare ersätts begreppet phatriark/mathriark av ett nytt könsneutralt begrepp i linje med sektionens policy.
Omdefinitionen av medlemskap innebär automatisk rätt att kvarstå på post.

Härigenom föreskrivs i fråga om stadgans avsnitt 2.7
\begin{dels}
    \item att avsnittets rubrik ska lyda ''Alumner''
    \item att avsnittet omnumreras till 2.5,
    \item att avsnittet i sin helhet ska ha följande lydelse.
\end{dels}

\begin{lydelse}%
  \subsubsection*{Definition}
    \itemb Med Phatriark/Mathriark avses den som som avlagt masters- eller civil\-ingenjörs\-examen som medlem av Fysik\-teknolog\-sektionen vid Ch\-al\-mers tekniska högskola.

  \switchcolumn
  \setcounter{enumi}{0}
    \item \emph{Alumn är} den som avlagt master- eller civilingenjörsexamen som \emph{ordinarie} medlem av sektionen.
    
  \switchcolumn*
    \subsubsection*{Phatriarkers/Mathriarkers rättigheter}%
    \itemb Phatriark/Mathriark har närvaro- och yttranderätt på sektionsmöten.

    \itemb Phatriark/Mathriark må kvarstå på post inom sektionen.
    
  \switchcolumn
    \emph{(ersätts av \S \ref{2.x:allarätt})}

  \switchcolumn*
    \subsubsection*{Phatriarkers/Mathriarkers skyldigheter}%
    \itemb Phatriark/Mathriark är skyldig att rätta sig efter sektionens stadgar, reglemente, övriga beslut samt Fysikteknologsektionens övriga styrdokument.
    
  \switchcolumn
    \emph{(ersätts av \S \ref{2.1:skyldig})}
\end{lydelse}

\subsection{Särskild medlem}
Syftet är att numrera alla punkter och uppdatera bestämmelser som försummats vid tidigare revisioner.
Begreppet ledamot ersätts av medlem, eftersom det redan är i bruk i andra mer vanligt förekommande definitioner.
Enligt den demokratiska principen att den som betalar avgift till föreningen också ska vara fullvärdig medlem tillskrivs nu särskild medlem per automatik alla rättigheter som ordinarie medlem, inklusive rösträtt.
(Det är dock olämpligt att icke-kårmedlemmar har denna rätt, så detsamma gäller ej seniormedlemmar.)
\emph{Tillhörande övergångsbestämmelse finns.}

Härigenom föreskrivs i fråga om stadgans avsnitt 2.8
\begin{dels}
    \item att avsnittets rubrik ska lyda ''Särskild medlem'',
    \item att avsnittet omnumreras till 2.6,
    \item att avsnittet i sin helhet ska ha följande lydelse.
\end{dels}

\begin{lydelse}%
    \subsubsection*{Definition}
    \itemb Särskild ledamot är kårmedlem vid Chalmers tekniska högskola som sektionsmöte med minst 2/3 majoritet beslutar, och som skänkt en summa motsvarande sektionsavgiften till sektionen.
  
    \itemb Kårmedlem vid Chalmers Tekniska Högskola har rätt att söka till särskild ledamot. Kårmedlem som önskar söka särskild ledamot skall meddela detta till sektionsstyrelsen och talmanspresidiet senast 6 läsdagar i förväg.

  \switchcolumn
  \setcounter{enumi}{0}
    \item \emph{Särskild medlem} är kårmedlem vid Chalmers tekniska högskola, \emph{utnämnd av sektionsmötet, som har} skänkt en summa motsvarande sektionsavgiften till sektionen.
    
    \subsubsection*{Ansökan}
    \item Kårmedlem vid Chalmers tekniska högskola har rätt att söka till särskild \emph{medlem}.
    \emph{Ansökan lämnas skriftligen} till sektionsstyrelsen och talmanspresidiet.
    
    \item \emph{Ärendet ska tas upp på nästa sektionsmöte, dock tidigast 6 läsdagar efter att förslaget inkommit.
    Beslut om utnämnande ska fattas med minst \sfrac{2}{3} majoritet.} \label{maj:sm}
    
\switchcolumn*
    \subsubsection*{Särskilda ledamöters rättigheter}%
    \itemb Särskild ledamot har närvaro- och yttranderätt på
   sektionsmöte.
   
   \itemb Särskild ledamot är valbar till post inom sektionen.

   \itemb Särskild ledamot har rätt att ta del av mötesprotokoll och sektionens övriga handlingar, undantaget de dokument som listas som icke offentliga i reglemente.
   
   \itemb Särskild ledamot har rätt att utnyttja av sektionen erbjudna tjänster.
    
\switchcolumn
    \subsubsection*{Rättigheter}%
    %\emph{(ersätts av \S \ref{2.1:närvaro})}

    %\item Särskild \emph{medlem} är valbar till post inom sektionen.
  
    %\emph{(ersätts av \S \ref{2.1:offentlighet})}

    %\item Särskild \emph{medlem har} rätt att utnyttja av sektionen erbjudna tjänster.

    \item \emph{Särskild medlem har samma rättigheter som ordinarie medlem.}

\switchcolumn*
    \subsubsection*{Särskilda ledamöters skyldigheter}%
    \itemb Särskild ledamot är skyldig att rätta sig efter sektionens stadgar, regle\-mente, övriga beslut samt  Fysikteknologsektionens övriga styrdokument.
    
\switchcolumn
    \emph{(ersätts av \S \ref{2.1:skyldig})}
\end{lydelse}

\section{Inspektor}
Syftet är att numrera alla punkter och använda enhetligt språk.
Eftersom alla sammanträden är tillgängliga för inspektor görs även offentlighetsprincipen gällande.

Härigenom föreskrivs i fråga om stadgans kapitel 3 att kapitlet i sin helhet, inklusive avsnitt, ska ha följande lydelse.

\subsection{Definition}
\begin{lydelse}
    \itemb Sektionens inspektor skall vara fysik- eller matematikprofessor vid institutionen för fysik eller institutionen för matematiska vetenskaper, och tillvarata F- och TM-teknologens intressen, samt fungera som en länk mellan teknologer och anställda.

    \itemb Inspektor har närvaro-, yttrande- och förslagsrätt vid sammanträde i sektionens samtliga organ.
    
  \switchcolumn
  \setcounter{enumi}{0}
    
    \item Sektionens inspektor \emph{ska} vara professor vid institutionen för fysik eller institutionen för matematiska vetenskaper och tillvarata \emph{sektionens teknologers} intressen, samt fungera som en länk mellan teknologer och anställda.

    \item Inspektor har närvaro-, yttrande- och förslagsrätt vid sammanträde i sektionens samtliga organ.
    
    \item \emph{Inspektor har rätt att ta del av mötesprotokoll och sektionens övriga handlingar, undantaget de dokument som listas som icke offentliga i reglemente.}
\end{lydelse}

\subsection{Val}
\begin{lydelse}
    \itemb Sektionens inspektor väljs på två på varandra följande sektionsmöten med minst 2/3 majoritet, för mandat på tre år. 
	
	\itemb Förslag till inspektor inlämnas till sektionsstyrelsen med minst 25 namnunderskrifter från sektionsmedlemmar eller lyfts av sektionsstyrelsen med enkel majoritet. Innan val skall sektionsstyrelsen tillfråga den nominerade om denne är att betrakta som valbar.
	
    \itemb Inspektors mandat kan förlängas med tre år åt gången av sektionsmötet, om detta sker med enkel majoritet. Om mandatet ej förlängs skall nyval av inspektor ske på nästkommande sektionsmöte. Inspektors mandat förlängs då tills dess att ny inspektor blivit vald.
    
  \switchcolumn
  \setcounter{enumi}{0}
    \item Sektionens inspektor väljs på två på varandra följande sektionsmöten med minst \sfrac{2}{3} majoritet, för mandat på tre år. \label{maj:in} 
	
	\item \emph{Nominering} till inspektor inlämnas till sektionsstyrelsen med minst 25 namnunderskrifter från medlemmar, eller lyfts av sektionsstyrelsen med enkel majoritet.
    Innan val skall sektionsstyrelsen \emph{inhämta} den nominerades \emph{samtycke till kandidatur}.
	
	\item Inspektors mandat kan förlängas med tre år åt gången av sektionsmötet.
    Om mandatet ej förlängs \emph{ska} nyval av inspektor ske på \emph{nästa} sektionsmöte.
	\emph{Nuvarande inspektor kvarstår interimistiskt} tills dess att ny inspektor blivit vald.
\end{lydelse}




\section{Organisation och ansvar}
\subsection{Verksamhetsutövning}
Syftet är att förnya språket och konsolidera verksamhetsgemensamma bestämmelser.
Sektionsordförande och studerandearbetsmiljöombud är ej längre separata organ inom sektionen, utan omfattas båda av sektionsstyrelsen per reglemente.

Härigenom föreskrivs i fråga om stadgans avsnitt 4.1
\begin{dels}
    \item att rubriken för avsnittet ska lyda ''Verksamhetsutövning'',
    \item att avsnittet i sin helhet ska ha följande lydelse.
\end{dels}

\begin{lydelse}
    \item Sektionens verksamhet ut\-övas på det sätt denna stadga med
   till\-hör\-ande regle\-mente föreskriver genom:
      \begin{enumerate}[label=\arabic*]
       \item Sektionsmötet
       \item Studerandearbetsmiljöombud
       \item Sektionens valberedning
       \item Sektionens revisorer
       \item Talmanspresidiet
       \item Sektionsstyrelsen
	   \item Sektionsordföranden
       \item Studienämnden
       \item Sektionskommitéer
       \item Sektionsföreningar
       \item Sektionsfunktionärer
       \item Intresseföreningar
      \end{enumerate}
    
  \switchcolumn
  \setcounter{enumi}{0}
    
    \item Sektionens verksamhet utövas på det sätt denna stadga med
   tillhörande reglemente föreskriver genom \emph{följande organ}:
      \begin{itemize}
       \item Sektionsmötet
       \item Sektionsstyrelsen
       %\item Sektionsordföranden
       \item Studienämnden
       \item Kommittéer
       \item Sektionsföreningar
       \item Funktionärer
       \item Intresseföreningar
       \item Talmanspresidiet
       \item Valberedningen
       \item Revisorer
      \end{itemize}
    
  \switchcolumn*
  \setcounter{enumi}{1}
    
	  \item Uppgifter och ansvar får delegeras enligt följande hierarki:
		\begin{enumerate}
			\item[-] Sektionsmötet har rätt att delegera både ansvar och uppgifter till valberedning, revisor, talmanspresidie, sektionsstyrelse, nämnder, sektionskommittéer, sektionsföreningar och sektionsfunktionärer.
			\item[-] Sektionsstyrelsen har rätt att delegera uppgifter till nämnder,  sektionskommittéer, sektionsföreningar och sektionsfunktionärer, förutsatt att uppgiften ligger under deras verksamhetsområde.
		\end{enumerate}
	
	\switchcolumn
	  \emph{(ersätts av \S\S \ref{4.1:högststart}--\ref{4.1:högstend}, \S\ref{4.x:delegation})}
  	%\item Uppgifter och ansvar får delegeras enligt följande hierarki:
  	%	\begin{itemize}
  	%		\item Sektionsmötet \emph{har} rätt att delegera både ansvar och uppgifter till \emph{alla organ, undantaget intresseföreningar}.
  	%		\item Sektionsstyrelsen \emph{har} rätt att delegera uppgifter till studienämnden,  kommittéer, sektionsföreningar och funktionärer, förutsatt att uppgiften ligger under deras verksamhetsområde.
  	% \end{itemize}
    
	\switchcolumn*
    \item[] (se \S 8.1.5, \S 9.2.1, \S 10.2.1, \S 11.2.1, \S 12.4)
  
    \item[] (se \S 8.1.6, \S 9.3.1, \S 10.3.1, \S 11.3.1, \S 12.5)

    \item[] (se \S 8.4.6, \S 9.3.2, \S 12.5.2)

  \switchcolumn
    \item \emph{Alla organ} har rätt att i namn och emblem använda sektionens namn och dess symboler.
      \label{4.1:emblem}

    \item \emph{Alla organ} är skyldiga att rätta sig efter sektionens stadgar, reglemente\emph{, beslut} samt övriga styrdokument.
      \label{4.1:rätta}
      
    \item \emph{Alla organ, med undantag för funktionärer, ska ha representation vid sektionsmötet.} \label{4.1:rep}

  \switchcolumn*
      (se \S 2.4.1, \S 2.4.2)
      
  \switchcolumn
    \item Sektionsaktiv är person vald till post av sektionsmötet eller sektionsstyrelsen.
      \label{4.x:aktiv}
      
    \item Förtroendeposter \emph{förtecknas} i \emph{stadga eller} reglemente.
      \emph{Förtroendevald är person vald till förtroendepost.}
      \label{4.x:fortroende}

    \item \emph{Ekonomiskt ansvariga förtecknas i stadga eller reglemente.}

\end{lydelse}

\subsection{Ansvarsförhållanden} \label{4.2:ansvar}
Syftet är att göra frågan om ansvar mindre stel och eliminera redundansen i frågan om delegation.

Härigenom föreskrivs i fråga om stadgans avsnitt 4.2 att avsnittet i sin helhet ska ha följande lydelse.

\begin{lydelse}
	\item Sektionsmötet är sektionens högsta beslutande organ.
    \item Dragos är sektionens högste beskyddare och utövar fanfareriets högsta befäl.
    \item Sektionsmötet har till sitt förfogande valberedning,revisorer, talmanspresidiet, sektionsstyrelsen och sektionsordförande.
    \item Övrig verksamhet lyder under sektionsstyrelsen, enligt stadgans kapitel~7.
    
  \switchcolumn
  \setcounter{enumi}{0}  
    \item Sektionsmötet är sektionens högsta beslutande organ.
      \label{4.1:högststart}
    
    \item[] \emph{(ersätts av reglemente)}

    \item Sektionsmötet har till sitt förfogande \emph{alla sektionens organ, undantaget intresseföreningar}.
    
    \item \emph{Sektionsstyrelsen har till sitt förfogande studienämnden, kommittéer, sektionsföreningar och funktionärer, inom deras respektive verksamhetsområden}.
      \label{4.1:högstend}
	
  \switchcolumn*
    (se \S 4.3.9)
    
  \switchcolumn
   \item \emph{Sektionsstyrelsen, talmanspresidiet, valberedningen, revisorer och förtroendevalda svarar inför sektionsmötet.} \label{4.2:ansvar1}
   \item \emph{Övriga sektionsaktiva svarar inför sektionsstyrelsen.}
      \label{4.2:ansvar2}
    %\item \emph{Hierarkin mellan sektionens organ i fråga om ansvarsförhållanden är som följer:
	%\begin{enumerate}
	%    \item Sektionsmötet är högsta instans.
	%    \item Förtroendevalda lyder direkt under sektionsmötet.
	%    \item Övriga sektionsaktiva lyder under sektionsstyrelsen.
	%   \end{enumerate}}

  \switchcolumn*
    \subsubsection*{}
    \item[] (se \S 4.1.2, \S 8.1.6, \S 9.3.3, \S 10.3.2, \S 11.3.2)

  \switchcolumn
    \subsubsection*{Delegation} 
    \item \emph{Överordnat organ enligt detta avsnitt har rätt att delegera ansvar och uppgifter samt beordra åtgärd.} 
      \label{4.x:delegation}
  
\end{lydelse}

\subsection{Tillsättande} \label{4.x:tillsättande}
Syftet är att införa två avsnitt som konsoliderar frågor om mandat och val, eftersom det ansvaret utövas delvis parallellt av sektionsmöte och styrelse.
I detta avsnitt tydliggörs även valbarhetsregler.
\emph{Tillhörande övergångsbestämmelse finns.}

Härigenom förskrivs i fråga om stadgan
\begin{dels}
    \item att ett nytt avsnitt 4.3 införs,
    \item att rubriken för avsnittet ska lyda ''Tillsättande'',
    \item att avsnittet i sin helhet ska ha följande lydelse.
\end{dels}

\begin{lydelse}
    (se \S 5.12.3, \S 5.13.1, \S 9.1.3)
  \switchcolumn
    \item \emph{Post i sektionens organ tillsätts i vanliga fall genom personval på sektionsmöte.} \label{4.x:tillsätt}

    \item Vid \emph{vakans kan} sektionsstyrelsen preliminärt tillsätta posten.
      \emph{Beslutet fastställs} på nästkommande sektionsmöte. \label{4.x:styretval}

    \item \emph{Kandidat ska närvara vid inval eller lämna skriftligt samtycke till sin kandidatur.} \label{4.x:samtycke}

  \switchcolumn*
    \item[] (se \S 5.9.1, \S 6.2.2, \S\S 13.1.4--5, samt reglemente)
    \item[] (se \S 5.9.3, \S 13.1.8)
    \item[] (se \S 13.1.2, samt reglemente)
    \switchcolumn  
    \subsubsection*{Valbarhet}
    \item \emph{Ingen kan samtidigt inneha fler än en förtroendepost, eller fler än en post i samma organ.}
    \label{S:ValbarhetMaxEnFortroende}

    \item \emph{Ingen kan samtidigt inneha post i fler än en av sektionsstyrelsen, talmanspresidiet, revisorer och valberedning.} \label{4.x:valbar.dubbel}

    \item \emph{Talman, vice talman och ledamot i valberedningen får ej inneha post i studienämnden eller kommitté.} \label{4.x:valbar.ober}

    \item \emph{Revisor får ej inneha ekonomiskt ansvar inom sektionen.}
      
    \item \emph{Talmanspresidiet och revisorer behöver ej vara medlemmar av sektionen.}

    \item \emph{Revisor ska vara myndig.} \label{4.x:valbar.revisormyndig}
      
    \item \emph{Ekonomiskt ansvarig ska vara förtroendevald och myndig.}
    \label{S:ValbarEkonomiskMyndig}
\end{lydelse}

\subsection{Entledigande}
Syftet är stadga generella bestämmelser om entledigande i stället för onödigt detaljerade regler om misstroende.
Dessutom förnyas språket och organisatoriska detaljer som varit utspridda eller redundanta samlas.
Förenklingar genomförs i enlighet med föregående avsnitt.

Härigenom föreskrivs i fråga om stadgans avsnitt 4.3
\begin{dels}
    \item att avsnittets rubrik ska lyda ''Entledigande'',
    \item att avsnittet omnumreras till 4.4,
    \item att avsnittet i sin helhet ska ha följande lydelse.
\end{dels}

\begin{lydelse}
  \item[] (se \S 2.1.2, \S 4.3.6, \S 4.3.8)
  \item[] (se \S 4.3.8)
  
\switchcolumn
  \setcounter{enumi}{0}
  \item \emph{Sektionsaktiv kan avsäga sig sin post och på begäran entledigas av sektionsstyrelsen.
      Beslutet fastställs på nästkommande sektionsmöte.}

  \item \emph{Sektionsaktiv som upphör som medlem ska entledigas, såvida inte sektionsstyrelsen beslutar annorlunda.
      Beslutet fastställs på nästkommande sektionsmöte.}
  \label{4.x:kvarstå}

  \item \emph{Vid entledigande av ekonomiskt ansvarig ska bokslut för perioden fram till fyllnadsvalet upprättas inom fyra veckor från fyllnadsvalet, och revision ska ske.}
    \label{4.2:enteko}

    \switchcolumn*
    \item[] (se \S\S 4.3.6--7, \S 4.3.10)
    \switchcolumn
    \subsubsection*{Sektionsstyrelsen}
    \item \emph{Om ledamot i sektionsstyrelsen entledigas ska extra sektionsmöte utlysas inom 15 läsdagar där fyllnadsval ska äga rum}.

    \item Om sektionsstyrelsen \emph{i sin helhet entledigas ska omedelbart} interimstyrelse och ny valberedning väljas.

    \item Interimstyrelsen övertar styrelsens befogenheter och skyldigheter \emph{tills ny styrelse är vald}, men får endast handha löpande ärenden.

    \item Om sektionsordförande \emph{entledigas} tar vice sektionsordförande över \emph{som tillförordnad sådan} tills ny sektionsordförande är vald.

    \subsubsection*{Talmanspresidiet}
    \item \emph{Om talmanspresidiet i sin helhet entledigas sker utlysning av och kallelse till extra sektionsmöte där fyllnadsval äger rum av sektionsordförande.
    Sektionsmötet väljer ett temporärt presidium för mötet.}

    \item \emph{Om talmannen entledigas tar vice talman över som tillförordnad sådan tills ny talman är vald.}
     

  \switchcolumn*
    \setcounter{subsection}{3}
    \item Misstroendevotum kan kallas av
	  \begin{itemize}
		\item[-] enskild styrelseledamot, 
  		\item[-] 25 medlemmar, eller
  		\item[-] endera av sektionens revisorer
  	  \end{itemize}
	
	\item Misstroendevotum får endast behandlas av instans högre än målet, enligt punkt 4.3.9.

  	\item Misstroendevotum ska tas upp för behandling inom 15 läsdagar.

  	\item Målet för misstroendevotum har rätt att närvarva vid och delta i diskussionen rörande beslutet.

  	\item Misstroendevotum resulterar i avsättande vid 2/3 majoritet i frågan. Omröstning ska ske slutet. 

  	\item Om sektionsstyrelsen avsätts genom misstroendevotum skall interimstyrelse och ny valberedning väljas. Talmanspresidiet utfärdar kallelse till extra sektionsmöte där ny ordinarie styrelse skall väljas. Detta sektionsmöte skall hållas inom 15 läsdagar. Interimstyrelsen övertar ordinarie styrelses befogenheter och skyldigheter tills ny ordinarie styrelse är vald, men får endast handha löpande ärenden. Inom fyra veckor från valet av den nya styrelsen skall ett bokslut för perioden fram till och med datumet för detta val upprättas. Sektionsmötet beslutar i samband med avsättandet om detta bokslut skall upprättas av avgående styrelse eller den nya styrelsen. Den nya styrelsen skall ej hållas ansvarig för brister i denna ekonomiska redovisning.

  	\item Om sektionsordförande avsätts genom misstroendevotum tar vice sektionsordförande över rollen vid avsättande fram tills ny sektionsordförande är vald.

  	\item Om ekonomiskt ansvarig avsätts genom misstroendevotum skall fyllnadsval ske inom 15 läsdagar efter avsättandet. Inom fyra veckor från fyllnadsvalet skall ett bokslut för perioden fram till och med datumet för detta val upprättas. Sektionsmötet beslutar i samband med avsättande av kassören om detta bokslut skall upprättas av den avgående eller den nya kassören. Den nya kassören skall ej hållas  ansvarig för brister i denna ekonomiska redovisning. 

	\item Instanser inom sektionen rangordnas enligt följande, med avseende på misstroendevotum:
  	  \begin{itemize}
  	    \item[-] Sektionsmötet är högsta instans
  		\item[-] Sektionsstyrelsen, förtroendevalda i kommittéer, sektionsföreningar och nämnder, valberedningen, studerandearbetsmiljöombud, talmanspresidiet samt revisorer och förtroendevalda funktionärer lyder direkt under sektionsmötet
  		\item[-] Ledamöter i kommittéer, sektionsföreningar och nämnder och funktionärer som ej nämns ovan lyder under sektionsstyrelsen
  	  \end{itemize}
	
  	\item Ifall misstroendevotum av talmanspresidiet, eller enskild medlem av talmanspresidiet, lyfts tillses kallelse av sektionsmötet av sektionsordförande. Sektionsmötet väljer ett temporärt talmanspresidie till sektionsmötet.

  \switchcolumn
    \subsubsection*{Misstroendeförklaring}
    \item \emph{En högre instans enligt avsnitt~\ref{4.2:ansvar} kan förklara att en sektionsaktiv inte har dess förtroende.}

    \item \emph{Yrkande om misstroendeförklaring kan väckas av antingen}
      styrelseledamot, revisor eller 25 medlemmar \emph{med förslagsrätt på sektionsmötet}.

    \item \emph{Sådant yrkande ska prövas inom 15 läsdagar.
    Om sektionsmötet ska pröva förtroendet ska kallelsen tydligt ange att en förtroendefråga behandlas.} \label{4.3:misstroende}

    \item \emph{Svarande vars förtroende behandlas ska under alla omständigheter beredas möjlighet att närvara och tala för sin sak.}

    \item \emph{Entledigande till följd av misstroende sker med sluten omröstning med minst \sfrac{2}{3} majoritet.} \label{maj:mi} 
     
    \item \emph{Om yrkande om missförtroendeförklaring} avser \emph{talmanspresidiet i del eller helhet sker utlysning av och kallelse till extra} sektionsmöte av sektionsordförande.
      Sektionsmötet väljer ett temporärt presidium för mötet.
    
    \item Om ekonomiskt ansvarig \emph{entledigas} genom misstroende\emph{förklaring} beslutar \emph{sektionsmötet} om \emph{bokslutet enligt \S\ref{4.2:enteko}} upprättas av avgående eller nyvald kassör. De nyvalda \emph{ekonomiskt ansvariga anses fria från ansvar för föregående redovisning.}
\end{lydelse}
\setcounter{subsection}{4}    

\subsection{Överklagan}
Syftet är att konsolidera bestämmelser om överklagan från övriga kapitel till ett enda ställe.

Härigenom föreskrivs i fråga om stadgan
\begin{dels}
  \item att ett nytt avsnitt 4.5 införs,
  \item att avsnittets rubrik ska lyda ''Överklagan'',
  \item att avsnittet i sin helhet ska ha följande lydelse.
\end{dels}

\begin{lydelse}
  \item[] (se \S 5.10.1, \S 7.7.1)
  
  \switchcolumn
  \item \emph{Beslut av sektionsmötet eller sektionsstyrelsen som strider mot kårens eller sektionens stadga, reglemente, beslut eller övriga styrdokument får undanröjas efter överklagan.}
  \item \emph{Överklagan gällande kårens styrdokument framställs av kårmedlem till kårfullmäktige.}
  \item \emph{Överklagan gällande sektionens styrdokument framställs av sektionsmedlem till kårfullmäktige, och om det gäller sektionsstyrelsebeslut även sektionsmötet.}
\end{lydelse}

\section{Sektionsmötet}
Syftet är redaktionellt, samt att uppdatera bestämmelser i enlighet med övriga förändringar.
En bättre skillnad på detaljnivå i stadga kontra reglemente har gjorts.

\subsection{Allmänt}
Härigenom föreskrivs i fråga om stadgans avsnitt 5.1
\begin{dels}
    \item att avsnittets rubrik ska lyda ''Allmänt'',
    \item att avsnittet i sin helhet ska ha följande lydelse.
\end{dels}

\begin{lydelse}
  \setcounter{subsection}{1}
    \item Sektionsmötet är sektionens högsta beslutande organ i vilket samtliga medlemmar äger rätt att delta och har rösträtt.

    \item[] ---
  \switchcolumn
  \setcounter{subsection}{1}
  \setcounter{enumi}{0}
    \item Sektionsmötet är sektionens högsta beslutande organ, i vilket samtliga medlemmar har rätt att delta.

    \item \emph{Fråga av större principiell betydelse ska avgöras av sektionsmötet.}

    %\item \emph{Sektionsmötet regleras utöver sektionens stadgar av reglemente och mötesordning.}
\end{lydelse}

\setcounter{subsection}{1}
\subsection{Åligganden}
Syftet är eliminera redundans.
Valberedningen och revisorer får särskild status i åligganden.

Härigenom föreskrivs i fråga om stadgans avsnitt 5.4
\begin{dels}
    \item att avsnittet omnumreras till 5.2,
    \item att avsnittet i sin helhet ska ha följande lydelse.
\end{dels}

\begin{lydelse}
  \setcounter{subsection}{4}
  \item[] (se \S 5.2.1)

  \item Det åligger sektionsmötet att innan utgången av varje läsperiod
  \begin{itemize}
    \item behandla verksamhets- och revisionsberättelse samt ansvarsfrihet för eventuella sektionsstyrelse, studienämnd, kommittéer och sektionsföreningar som gått av efter föregående ordinarie sektionsmöte.
  \end{itemize}

    \item Det åligger sektionsmötet att innan utgången av läsperiod 1
    \begin{itemize}
    \item fastställa budget för sektionen.
    \item fastslå/avslå de ledamöter till sektionen tillhörande programråd som sektionsstyrelsen preliminärt har valt
    \item fastställa verksamhetsplan för sektionsstyrelsen innevarande läsår
    \item välja sektionsaktiva enligt reglementet
    \item var tredje år besluta om förlängt mandat för inspektor.
    \end{itemize}

  \item Det åligger sektionsmötet att innan utgången av läsperiod 2
    \begin{itemize}
    \item välja sektionsaktiva enligt reglementet.
    \end{itemize}

  \item Det åligger sektionsmötet att innan utgången av läsperiod 3
    \begin{itemize}
    \item välja sektionsaktiva enligt reglementet.
    \end{itemize}

  \item Det åligger sektionsmötet att innan utgången av läsperiod 4
    \begin{itemize}
    \item välja sektionsordförande
    \item välja sektionskassör
    \item välja övriga ledamöter av sektionsstyrelsen enligt reglementet
    \item välja talman
    \item välja sektionsaktiva enligt reglementet.
    \end{itemize}
  
  \switchcolumn
  \setcounter{subsection}{2}

  \item \emph{Ordinarie} sektionsmöte ska sammanträda minst en gång per läsperiod.

  \item Det åligger sektionsmötet att innan utgången av varje läsperiod
  \begin{itemize}
    \item behandla \emph{aktuella årsredovisningar och ansvarsfriheter enligt avsnitt \ref{14.x:redovisning}},
    \item välja sektionsaktiva enligt reglemente.
  \end{itemize}

  \item Det åligger sektionsmötet att innan utgången av läsperiod 1
  \begin{itemize}
    \item fastställa verksamhetsplan för sektionsstyrelsen,
    \item fastställa budget för \emph{innevarande räkenskapsår},
    \item \emph{fastställa programrådsledamöter utsedda av sektionsstyrelsen},
    \item vart tredje år välja inspektor.
  \end{itemize}

  \item Det åligger sektionsmötet att innan utgången av läsperiod 4
  \label{S:SektmoteLP4Alagg}
  \begin{itemize}
    \item välja sektionsordförande, \emph{vice sektionsordförande} och sektionskassör,
    \item välja talmans\emph{presidium},
    \item \emph{välja valberedning},
    \item \emph{välja revisorer}.
  \end{itemize}
\end{lydelse}

\setcounter{subsection}{2}
\subsection{Utlysning} \label{5.2:utlysning}
Syftet är redaktionellt.

Härigenom föreskrivs i fråga om stadgans avsnitt 5.2 och 5.3
\begin{dels}
\item att avsnitten slås samman till ett avsnitt 5.3,
\item att avsnittets rubrik ska lyda ''Utlysning'',
\item att avsnittet i sin helhet ska ha följande lydelse.
\end{dels}

\begin{lydelse}
  \item[] (se \S 5.2.2)
  \item Rätt att hos talmannen begära utlysande av
    sek\-tions\-möte tillkommer styrelseledamot, inspektor,  Chalmers Studentkårs styrelse, sek\-ti\-ons\-revisor  eller minst 25 medlemmar. Sådant möte skall hållas inom femton läsdagar.

  \item Ordinarie sektionsmöte skall utlysas minst 10 läsdagar i
    förväg genom att preliminär föredragningslista och kallelse an\-slås enligt
    reglemente. 
    
  \item Slutlig föredragningslista, enligt reglemente, anslås minst 3 läsdagar före ordinarie möte. 
    
  \item Inkomna motioner och propositioner skall anslås
    minst 3 läsdagar i förväg.
    
  \item Extra sektionsmöte skall utlysas minst 5 läsdagar före mötet
    och åt\-följ\-as av slutlig föredragningslista.

\switchcolumn
  \item Sektionsmötet sammanträder på kallelse av talman, \emph{eller vid vakans på kallelse av sektionsstyrelsen.}

  \item Ordinarie sektionsmöte \emph{utlyses senast} 10 läsdagar i förväg genom \emph{anslag} av preliminär föredragningslista och kallelse enligt reglemente.

  \item Slutlig föredragningslista\emph{, tillsammans med inkomna handlingar,} anslås 3 läsdagar före \emph{sammanträdet} enligt reglemente.

  \item \emph{Extra sektionsmöte får begäras av} styrelseledamot, inspektor,
    Chalmers studentkårs styrelse, revisor eller minst 25 medlemmar \emph{med förslagsrätt}.
    Sådant möte ska hållas inom 15 läsdagar.
  
  \item Extra sektionsmöte \emph{utlyses senast} 5 läsdagar i förväg \emph{på samma sätt som ordinarie sektionsmöte}.

  \item \emph{Särskilda bestämmelser om kallelse finns i \S\ref{4.3:misstroende}, \S\ref{15.1:ändring}, och \S\ref{16.0:upplösning}.}
\end{lydelse}

\subsection{Beslutsförhet}
Syftet är redaktionellt.

Härigenom föreskrivs i fråga om stadgans avsnitt 5.5
\begin{dels}
  \item att avsnittet omnumreras till 5.4,
  \item att avsnittets rubrik ska lyda ''Beslutsförhet'',
  \item att avsnittet i sin helhet ska ha följande lydelse.
\end{dels}

\begin{lydelse}
  \setcounter{subsection}{5}
  \item Sektionsmötet är beslutsmässigt om mötet är behörigt utlyst enligt stadgans avsnitt 5.3 samt om fler än 15 röstberättigade medlemmar, exklusive sektionsstyrelsen och presidiet, är närvarande.
  \item Om färre än 25 medlemmar, exklusive sektionsstyrelsen och presidiet, är närvarande då beslut skall fattas, kan detta ske om ingen yrkar på bordläggning.

\switchcolumn
  \setcounter{subsection}{4}
  \item Sektionsmötet är \emph{behörigt} om det är utlyst enligt avsnitt \ref{5.2:utlysning} samt fler än 15 röstberättigade medlemmar, \emph{ej räknat} sektionsstyrelsen och \emph{talmans}presidiet, är närvarande.
  \item Om färre än 25 \emph{röstberättigade} medlemmar, \emph{ej räknat} sektionsstyrelsen och \emph{talmans}presidiet, är närvarande, \emph{kan beslut endast fattas om ingen yrkar på bordläggning.}
\end{lydelse}

\setcounter{subsection}{4}
\subsection{Sammanträden}
Syftet är att konsolidera bestämmelser om själva sektionsmötet när det väl sammanträder.
Rösträtten för särskild medlem införs rent tekniskt här.
Detaljregler om personval sänks ned till reglemente.
Motionsrätten specificeras.

Härigenom föreskrivs i fråga om stadgans avsnitt 5.6--5.8, 5.11--5.13
\begin{dels}
  \item att avsnitten utgår,
  \item att ett nytt avsnitt 5.5 införs med rubrik ''Sammanträden'',
  \item att avsnittet i sin helhet ska ha följande lydelse.
\end{dels}
\begin{lydelse}
  \setcounter{subsection}{6}
  \item Närvaro-, yttrande-, förslags- och rösträtt tillkommer sektionsmedlem.
  
  \item Närvaro- yttrande- och förslagsrätt tillkommer inspektor, sektionsrevisor samt av talmanspreseidiet.
  
  \item Närvaro- och yttranderätt tillkommer
  heders\-med\-lem, seniormedlem, phatriark/mathriark, särskild ledamot, kårledningen, kårens inspektor, samt av mötet adjungerade icke-medlemmar.

  \item Rösträtt kan endast tillkomma sektionsmedlemmar.
  
  \switchcolumn
  \setcounter{subsection}{5}
  \subsubsection*{Rättigheter}
  \item Närvaro-, yttrande-, förslags- och rösträtt tillkommer \emph{ordinarie medlem} samt \emph{särskild medlem}. \label{5.x:rösträtt}
  \item Närvaro-, yttrande-, och förslagsrätt tillkommer inspektor, revisor samt talmanspresidiet.
  \item Närvaro- och yttranderätt tillkommer \emph{övriga medlemmar}, Chalmers studentkårs ledning, Chalmers studentkårs inspektor samt adjungerade. \label{5.x:grundrätt}

  \switchcolumn*
  \setcounter{subsection}{11}
  \setcounter{enumi}{0}
  \item Röstning med fullmakt får ej ske.

  \item Omröstning skall ske öppet, utom vid personval då stadgans avsnitt 5.12 gäller, och vid misstroendevotum då stadgans paragraf 4.3.5 gäller.

  \item Vid lika röstetal i sakfråga avgörs frågan av talmannen.
  
  \switchcolumn
  \subsubsection*{Beslut och votering}
  \item \emph{Beslut fattas i regel med enkel majoritet. Vid lika röstetal i sakfråga avgör talmannen.}
  \item \emph{Undantag från beslutsordningen finns i \S\ref{maj:hm1}, \S\ref{maj:hm2}, \S\ref{maj:sm}, \S\ref{maj:in}, \S\ref{maj:mi}, \S\ref{maj:eä}, \S\ref{maj:sä}, \S\ref{maj:rä}, och \S\ref{maj:up}.}
  \item \emph{Votering sker i regel öppet. Undantag finns i \S\ref{maj:mi}, samt i reglemente och i mötesordning.}
  \item \emph{Röst får ej avläggas av ombud.}

  \switchcolumn*
  \setcounter{subsection}{12}
  \setcounter{enumi}{0}
  \item I det fall då det finns fler sökande än antalet platser skall personval ske med sluten omröstning. I annat fall skall personval ske öppet om annat ej begärs av röstberättigad sektionsmötesdeltagare.
  
  \item Vid lika röstetal vid personval företas en ny omröstning mellan de kandidater som fått lika röstetal. Vid lika röstetal i den andra omröstningen skiljer lotten.
  
  \item En person kan endast väljas om denne är närvarande eller har gett talmannen sitt skriftliga bifall till att bli vald.
  
  \setcounter{subsection}{13}
  \setcounter{enumi}{0}
  \item Vid vakantsatt post har sektions\-styr\-elsen rätt att preliminärt tillsätta posten. Fastställande sker på näst\-komm\-ande sektionsmöte.

  
  \switchcolumn
  \item[] \emph{(ersätts av reglemente)}
  \item[] \emph{(ersätts av \S\ref{4.x:styretval}, \S\ref{4.x:samtycke})}

  \switchcolumn*
  \setcounter{subsection}{17} 
  \setcounter{enumi}{0}
  \item Motionsrätt tillkommer endast sektionsmedlem.

  \item Medlem som önskar ta upp frågor på föredragningslistan skall
  anmäla detta till sektionsstyrelsen och talmanspresidiet senast 6 läsdagar i förväg. Sektionsstyrelsen skall ges möjlighet att yttra sig kring dessa.
    
  \switchcolumn
  \subsubsection*{Motion}
  \item \emph{Motion kan inlämnas till talmanspresidiet av innehavare av förslagsrätt senast 6 läsdagar i förväg, eller senast 11 läsdagar i förväg om ärendet ska anges i kallelse.}
  \item \emph{Sektionsstyrelsen ges möjlighet att yttra sig i ärendet.}
 
  \switchcolumn*
  \setcounter{subsection}{18} 
  \setcounter{enumi}{0}
  \item Vid sektionsmöte får ärende som inte angivits på slutgiltig
  före\-drag\-nings\-lista endast tas upp om sektionsmötet med minst 2/3 majoritet så beslutar.
  
  \switchcolumn
  \subsubsection*{Extra ärende}
  \item Ärende som inte angivits på slutgiltig föredragningslista \emph{får} tas upp om sektionsmötet med \sfrac{2}{3} majoritet så beslutar \emph{och ärendet ej behöver anges i kallelse}. \label{maj:eä}
\end{lydelse}

\setcounter{subsection}{5}
\subsection{Protokoll}
Syftet är redaktionellt.

Härigenom föreskrivs i fråga om stadgans avsnitt 5.14 och 5.15
\begin{dels}
\item att avsnitten slås samman till ett avsnitt 5.6,
\item att avsnittets rubrik ska lyda ''Protokoll'',
\item att avsnittet i sin helhet ska ha följande lydelse.
\end{dels}

\begin{lydelse}
  \setcounter{subsection}{14}
  \item Sektionsmötesprotokoll skall justeras av två av mötet utvalda
  justeringspersoner. Justerat protokoll skall anslås senast tre
  läsveckor efter mötet.

  \setcounter{subsection}{15}
  \setcounter{enumi}{0}
  \item Sektionsmötesbeslut som ej förtecknas i stadga, reglemente eller bland Fysikteknologsektionens övriga styrdokument, skall sammanställas i särskillt dokument tillgängligt för sektionens medlemmar.

  \item Fysikteknologsektionens övriga styrdokument ska listas i reglementet.
  
  \switchcolumn
  \setcounter{subsection}{5}
  \setcounter{enumi}{0}

  \item Sektionsmötesprotokoll \emph{förs enligt avsnitt \ref{15.2:proto}} och justeras av två av mötet valda \emph{justerare}.

  \item Justerat protokoll ska anslås senast tre läsveckor efter mötet.

  \item Sektionsmötesbeslut ej \emph{förtecknat i annat styrdokument sammanställs i särskild förteckning.}
\end{lydelse}

\section{Sektionsstyrelsen}
Syftet är redaktionellt.
Många tidigare bestämmelser har införlivats i övriga kapitel.
Definitionen av förtroendeposter lyfts från reglemente.
Stadgat krav på antal möten per läsperiod anses förlegat.

Härigenom föreskrivs i fråga om stadgans kapitel 7
\begin{dels}
  \item att kapitlet omnumreras till 6,
  \item att kapitlet i sin helhet, inklusive avsnitt, ska ha följande lydelse.
\end{dels}

\subsection{Definition}
\begin{lydelse}
  \setcounter{section}{7}
  \setcounter{subsection}{1}
  \item Sektionsstyrelsen handhar i överensstämmelse med denna stadga, befintligt reglemente samt beslut tagna av sektionsmötet den verkställande ledningen av sektionens verksamhet. Sektionsstyrelsen är sektionsmötets ställföreträdare.

  \setcounter{subsection}{3}
  \setcounter{enumi}{0}
  \item Sektionsstyrelsen ansvarar inför sektionsmötet för sektionens verksamhet och sektionens ekonomi.

  \setcounter{section}{7}
  \setcounter{subsection}{2}
  \item Sektionsstyrelsen består av sektionsordförande, vice sektionsordförande, sektionskassör samt övriga ledamöter enligt reglemente.

  \setcounter{subsection}{9}
  \item Sektionsordförande utövar i brådskande fall sektionsstyrelsens befogenheter. Ordförandebeslut skall prövas på följande styrelsemöte.

  \item I ordförandes frånvaro utövar vice sektionsordförande dennes befogenheter och fullgör dennes plikter.
\switchcolumn
  \item \emph{Sektionsstyrelsen handhar den verkställande ledningen av sektionens verksamhet och ansvar för sektionens ekonomi}.

  \item Sektionsstyrelsen består av sektionsordförande, vice sektionsordförande, sektionskassör samt övriga ledamöter enligt reglemente.
    \emph{Samtliga är förtroendeposter.}

  \item Sektionsordförande utövar i \emph{nödfall} sektionsstyrelsens befogenheter.
    Ordförandebeslut \emph{ska fastställas snarast av sektionsstyrelsen}.

  \item \emph{Vice sektionsordförande övertar i sektionsordförandens frånvaro} dennes befogenheter och plikter.
  \label{S:ViceLaRevolution}

  \item \emph{Sektionsordförande och sektionskassör är ekonomiskt ansvariga.}
  \label{S:SektionsstyrelseEkonomisktAnsvariga}
\end{lydelse}
\setcounter{section}{6}
\setcounter{subsection}{1}

%\subsection{Sammansättning}
%\begin{lydelse}
%\switchcolumn
%\end{lydelse}
%\setcounter{section}{6}
%\setcounter{subsection}{1}

\subsection{Styrelsemöte}
\begin{lydelse}
  \setcounter{section}{7}
  \setcounter{subsection}{5}
  \item Sektionsstyrelsen sammanträder minst tre gånger per läsperiod.
  \item Sektionsstyrelsen sammanträder på kallelse av sektionsordförande.
  \setcounter{subsection}{6}
  \setcounter{enumi}{0}
  \item Sektionsstyrelsen är beslutsmässigt om sektionsordförande eller vice sektionsordförande och sammanlagt mer än hälften av sektionsstyrelsens ledamöter är närvarande.
  \setcounter{subsection}{8}
  \setcounter{enumi}{0}
  \item Protokoll skall föras vid styrelsemöte, justeras av två
    styrelseledamöter och anslås senast två läsveckor efter mötet.
  \switchcolumn  

  \item[] \emph{(utgår)}
  \item Sektionsstyrelsen sammanträder på kallelse av sektionsordförande. \label{S:Styretkallelse}
  \item Sektionsstyrelsen är \emph{beslutsför} om \emph{fler än hälften av sektionsstyrelsens ledamöter är närvarande inklusive sektionsordförande eller vice sektionsordförande}.
  \item Protokoll förs vid styrelsemöte, justeras av två styrelseledamöter och anslås senast två läsveckor efter mötet.
\end{lydelse}
\setcounter{section}{6}
\setcounter{subsection}{2}

\section{Studienämnden}
Syftet är detsamma som för föregående kapitel.
Dessutom förtydligas rättigheterna angående studienämndsmöte.

Härigenom föreskrivs i fråga om stadgans kapitel 8
\begin{dels}
\item att kapitlet omnumreras till 7,
\item att kapitlet i sin helhet, inklusive avsnitt, ska ha följande lydelse.
\end{dels}

\subsection{Definition}
\begin{lydelse}
  \setcounter{section}{8}
  \setcounter{subsection}{1}
  \item Studienämnden vid Fysikteknologsektionen, även kallad SNF, har till uppgift att inom sektionen övervaka tillståndet och utvecklingen beträffande studiefrågor, aktivt verka för god kurslitteratur, främja kontakten med lärarna samt hålla god kontakt med sektionens medlemmar och styrelse.

  \setcounter{section}{8}
  \setcounter{subsection}{2}
  \item Studienämnden består av studienämndsordförande, studienämndskassör och medlemmar enligt reglemente angående studienämnden.
\switchcolumn
  \setcounter{section}{7}
  \item Studienämnden \emph{(SNF)} har till uppgift att inom sektionen övervaka tillståndet och utvecklingen beträffande studiefrågor, aktivt verka för god kurslitteratur samt främja kontakten med lärarna.
  
  \item Studienämnden består av studienämndsordförande, studienämndskassör och övriga \emph{ledamöter} enligt reglemente.

  \item \emph{Studienämndsordförande och studienämndskassör är förtroendeposter och ekonomiskt ansvariga.}
  \label{S:StudienamndEkonomiskt}
\end{lydelse}
\setcounter{section}{7}
\setcounter{subsection}{1}

\subsection{Studienämndsmöte}
\begin{lydelse}
  \setcounter{section}{8}
  \setcounter{subsection}{4}
  \item[] ---
  \item Medlem i studienämnden har närvaro-, yttrande-, förslags- och rösträtt på studienämndsmöte.
  \item Sektionsmedlem har närvaro-, yttrande- och förslagsrätt på studienämndsmöte.
  \setcounter{subsection}{3}
  \setcounter{enumi}{0}   
  \item Protokoll skall föras vid studienämndsmöte, justeras av en studienämndsledamot och anslås senast två läsveckor efter mötet.
  \switchcolumn
  \setcounter{section}{7}
  \setcounter{subsection}{2}
  \item \emph{Studienämnden sammanträder på kallelse av studienämndsordförande.}
  \item \emph{Ordinarie och särskilda medlemmar har} närvaro-, yttrande- och förslagsrätt på studienämndsmöte. \label{7.x:SNFrätt}
  \item \emph{Endast studienämndens ledamöter har rösträtt på studienämndsmöte.}
  \item Protokoll \emph{förs} vid studienämndsmöte, justeras av en studienämndsledamot och anslås senast två läsveckor efter mötet.
  \label{S:StudienamndProtokoll}
\end{lydelse}
\setcounter{section}{7}
\setcounter{subsection}{2}

\section{Kommittéer}
Syftet är redaktionellt.
Alla avsnitt utöver definitionen har införlivats i övriga kapitel.
Definitionen av förtroendeposter lyfts från reglemente.

Härigenom föreskrivs i fråga om stadgans kapitel 9
\begin{dels}
\item att kapitlet omnumreras till 8,
\item att kapitlet i sin helhet, inklsuive avsnitt, ska ha följande lydelse.
\end{dels}

\subsection{Definition}
\begin{lydelse}
\setcounter{section}{9}
\setcounter{subsection}{1}
  \item Sektionskommitté på sektionen skall ha ett i reglemente fastställt antal förtroendeposter.
  \item Sektionskommitté på sektionen kan ha ett i reglementet fastställt antal övriga medlemmar.
  \item Förtroendeposter tillsätts av sektionsmötet på förslag av valberedningen.
  \item Övriga medlemmar fastslås i enlighet med reglementet.
  \item Sektionskommitté skall verka för sektionens bästa och ha en i reglementet fastslagen uppgift.
\switchcolumn
  \setcounter{section}{8}
  \item \emph{Kommittéer verkar för att uppfylla sektionens ändamål och ska vara dess representanter.}
  \item \emph{Kommittéer fastställs och förtecknas i reglemente, med syfte, sammansättning och åligganden.}
  \item \emph{Sammansättningen ska åtminstone innehålla ordförande och kassör, vilka är förtroendeposter och ekonomiskt ansvariga.}
  \label{S:KomitteEkonomisktAnsvar}
  \item[] \emph{(ersätts av \S\ref{4.x:tillsätt} och \S\ref{13.x:valBfp})}
\end{lydelse}
\setcounter{section}{8}

\section{Sektionsföreningar}
Syftet är redaktionellt.
Alla avsnitt utöver definitionen har införlivats i övriga kapitel.

Härigenom föreskrivs i fråga om stadgans kapitel 10
\begin{dels}
\item att kapitlet omnumreras till 9,
\item att kapitlet i sin helhet, inklsuive avsnitt, ska ha följande lydelse.
\end{dels}

\subsection{Definition}
\begin{lydelse}
\setcounter{section}{10}
\setcounter{subsection}{1}
  \item Sektionsförening på sektionen skall ha ett i reglemente fastställt antal förtroendeposter.
  \item Övriga medlemmar tecknas i reglementet.
  \item Medlemmar tillsätts enligt reglementet.
  \item Sektionsförening skall verka för sektionens bästa och ha en i reglementet fastslagen uppgift.
\switchcolumn
  \setcounter{section}{9}
  \item \emph{Sektionsföreningar verkar för att uppfylla sektionens ändamål.}
  \item \emph{Sektionsföreningar fastställs och förtecknas i reglemente, med sammansättning och åligganden.}
  \item[] \emph{(ersätts av \S\ref{4.x:tillsätt})}
\end{lydelse}
\setcounter{section}{9}

\section{Funktionärer}
Syftet är mestadels redaktionellt.
Alla avsnitt utöver definitionen har införlivats i övriga kapitel.
Stadgan öppnar för direkt tillsättning av funktionärer om reglemente så föreskriver (och gör dumvästen stadgeenlig!).

Härigenom föreskrivs i fråga om stadgans kapitel 11
\begin{dels}
\item att kapitlet omnumreras till 10,
\item att kapitlet i sin helhet, inklsuive avsnitt, ska ha följande lydelse.
\end{dels}

\subsection{Definition}
\begin{lydelse}
  \setcounter{section}{11}
  \setcounter{subsection}{1}
\item Sektionsfunktionär på sektionen ska finnas enligt reglemente.
\item Sektionsfunktionär tillsätts av sektionsmötet.
\item Sektionsförening skall verka för sektionens bästa och ha en i reglementet fastslagen uppgift.
\item Sektionsfunktionärer betraktas ej som förtroendevalda om ej annorlunda specificerats i regle-
mentet.
\switchcolumn
\setcounter{section}{10}
\item \emph{Funktionärer verkar för att uppfylla sektionens ändamål genom enskild uppgift.}
\item \emph{Funktionärer fastställs och förtecknas i reglemente, med sammansättning och åligganden.}
\item \emph{Om reglemente så föreskriver kan funktionär tillsättas på annat sätt än enligt \S\ref{4.x:tillsätt}.}
\item[] \emph{(redundant)}
\end{lydelse}
\setcounter{section}{10}

\section{Intresseföreningar}
Syftet är rent redaktionellt.
Många bestämmelser har införlivats i övriga kapitel.

Härigenom föreskrivs i fråga om stadgans kapitel 12
\begin{dels}
\item att kapitlet omnumreras till 11,
\item att kapitlet i sin helhet, inklusive avsnitt, ska ha följande lydelse.
\end{dels}

\subsection{Definition}
\begin{lydelse}
  \setcounter{section}{12}
  \setcounter{subsection}{1}
  \item En intresseförening på sektionen är en sammanslutning av sektionsmedlemmar med ett gemensamt intresse.
  \item Intresseföreningen skall verka för sektionens bästa och ha en i reglementet fastslagen uppgift.
\switchcolumn
  \setcounter{section}{11}
  \item Intresseförening är en sammanslutning av sektionsmedlemmar med gemensamt intresse \emph{som verkar enligt sektionens ändamål}.
\switchcolumn*
  \setcounter{subsection}{3}
  \setcounter{enumi}{0}
  \item Intresseförenings status beviljas och fråntas av sektionsmötet.
  
  \setcounter{subsection}{8}
  \setcounter{enumi}{0}
  \item Sektionens intresseföreningar är listade i reglementet.
\switchcolumn
  \item Intresseföreningsstatus beviljas och fråntas av sektionsmötet.
  \item Intresseföreningar \emph{förtecknas} i reglemente.
\end{lydelse}
\setcounter{section}{11}
\setcounter{subsection}{1}

\subsection{Krav}
\begin{lydelse}
  \setcounter{section}{12}
  \setcounter{subsection}{2}
  \item Intresseförening skall ha en av sektionsstyrelsen godkänd stadga.
  \item Intresseförening skall ha en styrelse. Minst två tredjedelar av styrelsledamöterna skall vara sektionsmedlemmar.
  \setcounter{subsection}{7}
  \item Varje sektionsmedlem skall ha rätt till medlemskap. Dock kan föreningsmedlem som motverkar föreningens syfte uteslutas. 
  \item Styrelsen kan besluta om medlemskap för någon som ej är medlem i Fysikteknologsektionen så länge dylika medlemmar ej utgör mer än hälften av föreningens medlemmar.
  \switchcolumn
  \setcounter{section}{11}
  \item Intresseförening \emph{ska} ha en av sektionsstyrelsen godkänd stadga.
  \item Intresseförening \emph{ska} ha en styrelse \emph{bestående till minst \sfrac{2}{3} av} sektionsmedlemmar.
  \item Varje sektionsmedlem \emph{ska} ha rätt till medlemskap i \emph{intresseförening}.
    Dock \emph{får} medlem som motverkar \emph{intresse}föreningens syfte uteslutas. \label{11.x:intrrätt}
  \item \emph{Sektionsmedlemmar ska utgöra minst hälften av intresseförenings medlemmar.}
\end{lydelse}
\setcounter{section}{11}
\setcounter{subsection}{2}

\subsection{Ekonomi och revision}
\begin{lydelse}
  \setcounter{section}{12}
  \setcounter{subsection}{6}
  \item Intresseförening skall ha en fristående ekonomi.
  \item Intresseförenings verksamhet och ekonomi granskas, utöver deras egna revisorer, av sektionens revisorer.
  \item Senast tre veckor efter att årsmötet godkänt verksamhetsberättelsen för föregående år skall denna lämnas till sektionsstyrelsen och frågan om ansvarsfrihet för intresseföreningens aktuella styrelse skall behandlas.
  \item Senast tre veckor efter att intresseföreningens revisors revisionsberättelse godkänts av årsmötet skall av sektionsrevisorerna begärt material lämnas till dessa.
\switchcolumn
  \setcounter{section}{11}
  \item Intresseförening \emph{ska} ha \emph{ekonomi fristående från sektionen}.
  \item Intresseförening\emph{ar}s verksamhet och ekonomi granskas, utöver \emph{egen revision}, av sektionens revisorer.
  \item \emph{Efter att intresseförenings årsmöte behandlat förgående verksamhetsår ska alla handlingar inom tre veckor tillsändas sektionsstyrelsen och sektionens revisorer.}
  \item \emph{Sektionsstyrelsen prövar frågan om ansvarsfrihet för intresseföreningens styrelse efter fullgjort verksamhetsår.}
\end{lydelse}
\setcounter{section}{11}
\setcounter{subsection}{3}

\section{Talmanspresidiet}
Syftet är att ge Talmanspresidiet som det organ det är ett eget kapitel.
Det tydliggör hur stadga och reglemente speglar sektionens struktur.
Presidiets sammansättning lyfts till stadgan, eftersom reglerna om organisation och ansvar ändå nämner vice talman.
Notera att valbarhetshindren flyttats upp till tidigare kapitel.

Härigenom föreskrivs i fråga om stadgans avsnitt 5.9
\begin{dels}
  \item att avsnittet ska omformas och flyttas till kapitel 12,
  \item att kapitlet i sin helhet, inklsuive avsnitt, ska ha följande lydelse.
\end{dels}

\subsection{Definition}
\begin{lydelse}
  \setcounter{section}{5}
  \setcounter{subsection}{9}
  \item Talmannen väljs av sektionsmötet. Talmannen får ej vara medlem av sektionsstyrelsen.
  \item Talmannen ska leda sektionsmötet i överensstämmelse med stadgan, reglementet samt av sektionsmötet fastslagen mötesordning.
  \item Talmannen behöver ej vara medlem av sektionen.
  \item Talmanspresidiet består av talman och övriga ledamöter i enlighet med reglementet.
\switchcolumn
  \setcounter{section}{12}
  \item \emph{Talmanspresidiet leder sektionsmötet och förvaltar sektionens demokratiska grund.}
  \item Talmanspresidiet består av talman\emph{, vice talman och sekreterare.}
  \label{S:TalPStrukt}
  \item \emph{Vice talman övertar i talmans frånvaro dennes befogenheter och plikter.}
  \label{S:TalPViceTOrdf}
\end{lydelse}
\setcounter{section}{12}

\section{Valberedningen}
Syftet är redaktionellt.
En del redundans (t.ex. valprocess, ansvarsbeskrivning) tas bort.
Jävsfrågor sänks ned till reglemente.
Proceduren för val av ordförande och vice ordförande sänks ned till arbetsordning om det skulle behövas.

Härigenom föreskrivs i fråga om stadgans kapitel 6
\begin{dels}
  \item att kapitlet omnumreras till 13,
  \item att kapitlet i sin helhet, inklusive avsnitt, ska ha följande lydelse.
\end{dels}

\subsection{Definition}
\begin{lydelse}
  \setcounter{section}{6}
  \setcounter{subsection}{1}
  \item Sektionens valberedning skall väljas av sektionsmötet. Valberedningen skall agera oberoende.
  \item Ledamot i valberedningen skall ej vara jävig gentemot dem den valbereder.
  \setcounter{subsection}{2}
  \setcounter{enumi}{0}
  \item Valberedningen består av 3---7 ledamöter varav två internt väljs till ordförande respektive vice ordförande.
    % dagens trigger att det är ett långt tankstreck >:(
  \item Ledamot i valberedningen får ej vara medlem i sektionsstyrelsen eller någon kommitté eller nämnd som har representant i sektionsstyrelsen.
  \item I det fall då valberedningen ej är fulltalig och sektionsstyrelsen så anser lämpligt har sektionsstyrelsen möjlighet att temporärt välja kvarstående valberedningsposter inför varje valberedningsprocess. Dessa temporära medlemmar får inneha post i sektionsstyrelsen.
  \setcounter{subsection}{4}
  \setcounter{enumi}{0}  
  \item Valberedningen ansvarar för nomineringar till samtliga poster i sektionsstyrelsen. Därutöver ansvarar valberedningen för nomineringar till förtroendeposter och övriga poster på sektionen enligt reglementet.

  \switchcolumn
  \item Valberedningen ansvarar för \emph{oberoende} nomineringar till \emph{samtliga förtroendeposter, inklusive sektionsstyrelsen,} och övriga poster enligt reglemente. \label{13.x:valBfp}
  \item Valberedningen består av 7 ledamöter varav två internt väljs till ordförande respektive vice ordförande.
  \item \emph{Vid vakans kan sektionsstyrelsen tillförordna valberedningsledamöter tills nästa sektionsmöte. }

  \item[] \emph{(ersätts av \S\S\ref{4.x:valbar.dubbel}--\ref{4.x:valbar.ober})}
\end{lydelse}
\setcounter{section}{13}
\setcounter{subsection}{1}

\subsection{Nomineringsbeslut}
\begin{lydelse}
  \setcounter{section}{6}
  \setcounter{subsection}{3}
  \item Vid valberedningens första sammanträde skall ordförande och vice ordförande väljas. Minst 3 av valberedningens ledamöter måste då vara närvarande.
  \item När valberedningen sammanträder har max två medlemmar ur berörd styrelse, nämnd, kommitté, sektions\-förening eller funktionär närvaro-, för\-slags-, yttrande- och rösträtt. Valberedningens ordförande är ordförande samt sammankallande för valberedningen.
  \item Valberedningen är beslutsförig om dess ordförande, eller vice ordförande, minst två ytterligare
ledamöter samt minst en representant för berörd styrelse, nämnd, kommitté eller funktionär
är närvarande.
  \setcounter{subsection}{5}
  \setcounter{enumi}{0}
  \item Valberedningens nomineringar skall anslås i enlighet med reglementet.
  \switchcolumn
  \item[] \emph{(ersätts av reglemente)}
  \item \emph{Valberedningen sammanträder för nominering på kallelse av ordförande. Upp till två representanter för organ till vilket ska nomineras får adjungeras.}
  \item Valberedningen är beslutsför \emph{om minst tre ledamöter är närvarande inklusive ordförande eller vice ordförande}, samt minst \emph{en adjungerad representant för organ till vilket ska nomineras.}
  \item \emph{Nominering ska beslutas senast 7 dagar innan sektionsmötet då val äger rum.} 
  \label{S:ValBAnslaNomTid}
  \item Valberedningens nomineringar anslås i enlighet med reglemente \emph{samt tillsänds sektionsstyrelsen och talmanspresidiet}.
  \label{S:ValBAnslaNom}
\end{lydelse}
\setcounter{section}{13}
\setcounter{subsection}{2}

\section{Ekonomi och revision}
Syftet är att konsolidera bestämmelser om ekonomi och revision till ett enda lättillgängligt ställe.
Detta görs genom att omstrukturera kapitlet för revision.
Många bestämmelser överförs från andra kapitel.
Dessutom förenklas förfarandet med verksamhetsplaner och verksamhetsberättelser som tagit mycket tid på sektionsmöten; den kontrollerande funktion tas över av styret när det gäller andra organ än sektionsstyrelsen och studienämnden.

Härigenom föreskrivs i fråga om stadgans kapitel 13
\begin{dels}
  \item att kapitlet omnumreras till 14,
  \item att kapitlets rubrik ska lyda ''Ekonomi och revision'',
  \item att kapitlet i sin helhet, inklusive avsnitt, ska ha följande lydelse.
\end{dels}

\subsection{Allmänt}
\begin{lydelse}
  \item[] ---

  \item[] (se \S 7.4.1)
  
  \switchcolumn

  \item \emph{Sektionens organisationsnummer är 857208-8477.}

  \item \emph{Sektionens firma tecknas av sektionsordförande och sektionskassör var för sig.}
  \label{S:EkonomiSektFirma}
\end{lydelse}

\subsection{Redovisning och ansvarsfrihet} \label{14.x:redovisning}
\begin{lydelse}
  \item[] (se \S 8.7.2, \S 9.3.4, \S 9.4.2, \S 10.3.3, \S 10.4.2, \S 11.4.1)
  \item[] (se \S 13.1.9)
  \item[] (se \S 13.3.1)
  \switchcolumn
  \setcounter{enumi}{0}  
  \item \emph{Sektionsstyrelsen upprättar gemensamt bokslut för sektionen, innefattande alla organ utom intresseföreningar.}
  \item \emph{Studienämnden, kommittéer och övriga organ enligt reglemente upprättar dessutom egen löpande redovising och delbokslut.}
  \label{S:RedovisningKont}
  \item \emph{Sektionsstyrelsen handhar annars den löpande redovisningen.}
  \item \emph{Gemensam årsredovisning enligt detta avsnitt fastställs vid första ordinarie sektionsmöte efter sektionens verksamhetsårs slut.
      Handlingarna tillställs revisorer, sektionsstyrelsen och talmanspresidiet senast 11 läsdagar innan mötet.} \label{14.x:tillställa}
  \item \emph{Frågan om ansvarsfrihet för förtroendeposter behandlas vid första ordinarie sektionsmöte efter berört organs verksamhetsår av sektionsmötet på grundval av revisionsberättelsen.}
\end{lydelse}
% budget
% bokslut

\subsection{Revisor}
\begin{lydelse}
  \setcounter{section}{13}
  \setcounter{subsection}{1}
  \item Sektionsmötet utser två lekmannarevisorer med uppgift att granska sektionens verksamhet och ekonomi.
  \item Revisor skall vara myndig.
  \item Revisor skall ej vara jävig gentemot de de granskar.
  \item Revisor kan ej vara medlem av sektionsstyrelsen under sitt verksamhetsår.
  \item Revisor kan ej inneha ekonomiskt ansvar inom sektionen.
  \item En revisor kan ej granska ekonomin för en kommitté, sektionsförening, nämnd eller sektionsstyrelse för ett år då denne var medlem av denna eller direkt påföljande år.
  \item I de fall då samtliga verksamheter inte kan granskas av ordinarie revisorer kan extra revisor väljas för att granska dessa verksamheter.
  \item Revisor behöver ej vara medlem av sektionen.
 
  \switchcolumn
  \setcounter{enumi}{0}  
  \item \emph{Lekmannarevisorerna granskar oberoende och kontinuerligt sektionens verksamhet och ekonomi.}

  \item \emph{Revisor får ej granska verksamhetsår för organ, då denna var ledamot det året eller föregående, eller då annat jäv föreligger.} \label{14.x:revisorjäv}

  \item \emph{Två ordinare revisorer väljs. Om revisorer enligt \S\ref{14.x:revisorjäv} inte kan granska samtliga organ väljs extra revisor för dessa.}

  \item[] \emph{(ersätts av \S\S\ref{4.x:valbar.dubbel}--\ref{4.x:valbar.revisormyndig})}

  \switchcolumn*
  \item Räkenskaper och övriga handlingar skall tillställas revisorerna senast 10 läsdagar före ordinarie sektionsmöte.
    \setcounter{subsection}{1}
    \setcounter{enumi}{0}  
  \item Det åligger revisorerna att anslå revisionsberättelser senast tre läsdagar före ordinarie sektionsmöte.
  \item Revisionsberättelsen skall innehålla yttrande i fråga om ansvarsfrihet för berörda personer.
  \item Det åligger revisorerna att under året kontinuerligt granska räkenskaper och förvaltning.
 
  \switchcolumn
  \item[] \emph{(ersätts av \S\ref{14.x:tillställa})}
 
  \item \emph{Revisor åligger att avge och anslå revisionberättelse, inklusive yttrande angående ansvarsfrihet, senast 4 läsdagar innan sektionsmöte.}

\end{lydelse}
\setcounter{section}{14}
\setcounter{subsection}{2}
\setcounter{enumi}{0}

\section{Styrdokument}
Syftet är dels redaktionellt, dels att förtydliga ändringsförfarandet.
En gammal tvist om huruvida ändringsyrkanden kan göras och påverkar stageändringen löses.
Eftersom talmannen har rätt att förordna vilka sektionsmöten är ordinarie (det är de vilka inte kallas på särskild anmodan) läggs tidskrav till för stadgeändring.

Härigenom föreskrivs i fråga om stadgans kapitel 14
\begin{dels}
  \item att kapitlet omnumreras till 15,
  \item att kapitlet i sin helhet, inklusive avsnitt, ska ha följande lydelse.
\end{dels}

\subsection{Allmänt}
\begin{lydelse}
  \setcounter{section}{14}
  \setcounter{subsection}{1}
  \item Förutom denna stadga finns reglemente och övriga styrdokument i enlighet med reglementet.
  \setcounter{subsection}{6}
  \setcounter{enumi}{0}  
  \item Originalstadgar tillhandahas av sektionsstyrelsen.
  \switchcolumn
  \setcounter{subsection}{1}
  \item \emph{Utöver denna stadga ska finnas reglemente och övriga styrdokument förtecknade däri.}
  \item Originalstadgar tillhandahas av sektionsstyrelsen.
\end{lydelse}
\setcounter{section}{15}
\setcounter{subsection}{1}

\subsection{Ändring}
\begin{lydelse}
  \setcounter{section}{14}
  \setcounter{subsection}{2}
  \item Ändring av eller tillägg till denna stadga inklusive dess bilagor kan endast göras av sektionsmötet och om minst 2/3 av de närvarande är om beslutet ense under två på varandra följande ordinarie sektionsmöten och den föreslagna lydelsen varit anslagen tillsammans med kallelsen. \label{15.1:ändring}
  \item Ändring av eller tillägg till denna stadga skall godkännas av Chalmers Studentkårs styrelse.
  \setcounter{subsection}{3}
  \setcounter{enumi}{0}  
  \item Ändring av eller tillägg till sektionens reglemente inklusive dess bilagor kan endast göras av sektionsmötet och om minst 2/3 av de närvarande är om beslutet ense.
  \switchcolumn
  \item \emph{Ändring av stadga eller reglemente görs av sektionsmötet. Kallelsen ska tydligt ange att fråga om ändring behandlas tillsammans med fullständigt förslag.}
  \item \emph{Beslut om stadgeändring fattas med \sfrac{2}{3} majoritet.
      Fastställande av beslutet görs med \sfrac{2}{3} majoritet på nästkommande sektionsmöte, dock tidigast efter fyra läsveckor förflutit.
      Lydelsen får ej ändras vid fastställandet.} \label{maj:sä}
  \item \emph{Stadgeändring ska fastställas av Chalmers studentkårs styrelse.}
  \item \emph{Beslut om reglementesändring fattas med \sfrac{2}{3} majoritet.} \label{maj:rä}
\end{lydelse}
\setcounter{section}{15}
\setcounter{subsection}{2}

\subsection{Protokoll och tillkännagivande} \label{15.2:proto}
\begin{lydelse}
  \setcounter{section}{14}
  \setcounter{subsection}{5}
  \item Protokoll som förs i sektionens olika organ skall innehålla anteckningar om ärendenas art, samtliga ställda och ej återtagna yrkanden, beslut samt särskilda yttranden och reservationer.
  \item Beslut som fattas inom sektionen och berör Chalmers Studentkår i dess helhet skall meddelas Chalmers Studentkårs styrelse.
  \item Sektionens officiella kommunikationskanaler utgörs av sektionens hemsida samt sektionens officiella anslagstavla.
  \item Meddelanden och beslut är behörigt anslagna då de anslås via någon av sektionens officiella kommunikationskanaler.
  \item[] (se \S 2.2.5, \S 2.8.2) 
  \switchcolumn
  \setcounter{subsection}{3}
  \item Protokoll som förs i sektionens organ ska innehålla anteckningar om ärendenas art, samtliga ställda och ej återtagna yrkanden, beslut samt särskilda yttranden och reservationer.
  \item Protokoll och beslut \emph{tillkännages genom behörigt anslag enligt reglemente.}
  \item Beslut \emph{fattade} inom sektionen som berör Chalmers studentkår i dess helhet \emph{meddelas} Chalmers studentkårs styrelse.
  \item Medlem \emph{har} rätt att ta del av mötesprotokoll och sektionens övriga handlingar, undantaget de dokument som listas som icke offentliga i reglemente. \label{15.x:offentlighet}
\end{lydelse}
\setcounter{section}{15}
\setcounter{subsection}{3}

\subsection{Tolkningstvister}
\begin{lydelse}
  \setcounter{section}{14}
  \setcounter{subsection}{4}
  \item Uppstår tolkningstvist om dessa stadgars tolkning skall frågan hänskjutas till sektionens inspektor.
  \item Vid konflikt med reglemente eller Fysikteknologsektionens övriga styrdokument har stadgan företräde.
  \switchcolumn
  \item \emph{Vid konflikt med annat styrdokument har stadga och reglemente företräde. Vid konflikt med reglemente har stadga företräde.}
  \label{S:TolkningStadgaFore}
  \item \emph{Vid tvist om stadgans tolkning avgörs frågan av inspektor.}
\end{lydelse}
\setcounter{section}{15} 

\section{Upplösning}
Syftet är redaktionellt.
Kraven för upplösning justeras enligt ovanstående observationer.
En ideell förening är en juridisk person med begränsat personligt ansvar; det finns ingen anledning att påtvinga ChS att åta sig sektionens skulder om den t.ex. skulle upplösas till följd av förestående konkurs.
Det är också i linje med Skatteverkets instruktioner om avveckling av ideell förening.
Med det sagt hindrar inget oss från att stadga disposition av eventuellt överskott.
Dessutom bör vi bevara vår relativt långa historia sett till övriga sektioner och se till att arkivera handlingar om det värsta händer.

Härigenom föreskrivs i fråga om stadgans kapitel 15
\begin{dels}
  \item att kapitlet omnumreras till 16,
  \item att kapitlets rubrik ska lyda ''Upplösning'',
  \item att kapitlet i sin helhet, inklusive avsnitt, ska ha följande lydelse.
\end{dels}

\subsection{Upplösningsbeslut}
\begin{lydelse}
  \setcounter{section}{15}
  \setcounter{subsection}{0}
  \item Sektionen upplöses genom beslut på två på varandra följande sektionsmöten med minst 3/4 majoritet och minst 25 bifallande medlemmar.
  \switchcolumn
  \setcounter{subsection}{1}
  \item Sektionen upplöses genom beslut på två sektionsmöten, \emph{mellan vilka minst 4 läsveckor förflutit}, med \sfrac{3}{4} majoritet och minst 25 bifallande medlemmar. \label{maj:up}
  \emph{Kallelsen ska tydligt ange att fråga om upplösning behandlas.} \label{16.0:upplösning}
\end{lydelse}
\setcounter{section}{16}
\setcounter{subsection}{1}

\subsection{Avveckling}
\begin{lydelse}
  \setcounter{section}{15}
  \setcounter{subsection}{0}
  \setcounter{enumi}{1}
  \item Om sektionsmötet beslutar att upplösa sektionen skall samtliga dess tillgångar och skulder, som framgår av upprättad balansräkning, i och med upplösning tillfalla Chalmers Studentkår.
  \item I det fall medlen utgörs av tillgångar skall Chalmers Studentkår fondera och förvalta dessa till ny sektion bildats för studerande på utbildningsprogrammet för Teknisk fysik och/eller Teknisk matematik eller motsvarande.
  \switchcolumn
  \setcounter{subsection}{2}
  \item \emph{Vid upplösning avvecklar sektionsstyrelsen snarast sektionen.
      Eventuellt överskott tillfaller Chalmers studentkår.
      Sektionens handlingar arkiveras på Göteborgs föreningsarkiv.}
  \item \emph{Chalmers studentkår förvaltar eventuellt överskott i fond fram tills} ny sektion bildats för studerande på utbildningsprogrammet för Teknisk fysik och/eller Teknisk matematik eller motsvarande.
\end{lydelse}
\setcounter{section}{16}
\setcounter{subsection}{2}

\clearpage

%%%%%%%%%%%%%%%%%%%%%%%%%%%%%%%%%%%%%%%%%%

\part{Reglemente}

\setcounter{section}{0}
\section{Medlemskap}
Syftet är redaktionellt.

Härigenom föreskrivs i fråga om reglementets kapitel 1
\begin{dels}
    \item att kapitlets rubrik ska lyda ''Medlemskap'',
    \item att nuvarande avsnitt 1.1 och 1.2 ska utgå.
\end{dels}

\begin{lydelse}
    \setcounter{subsection}{1}
    \setcounter{enumi}{0}
    \item Medlem som går i första årskursen på programmet Teknisk fysik eller Teknisk matematik
skall städa sektionslokalen.

    \setcounter{subsection}{2}
    \setcounter{enumi}{0}
    \item Följande dokument har medlem inte rätt att ta del av, såvida det inte behövs för sektionens
verksamhet:
    \begin{itemize}
        \item Incidenthanteringsprotokoll
        \item Dokument innehållande personuppgifter
    \end{itemize}
    
    \setcounter{subsection}{0}
    \switchcolumn
    
    \item[] \emph{(ersätts av \S\ref{R:Nollanstad})}\vspace{2.4em}
    \item[] \emph{(ersätts av avsnitt \ref{R:sekretess})}
\end{lydelse}

\subsection{Hedersmedlemmar}
Syftet är redaktionellt.

Härigenom föreskrivs i fråga om reglementets avsnitt 1.3
\begin{dels}
    \item att avsnittet ska omnumreras 1.1,
    \item att avsnittet i sin helhet ska ha följande lydelse.
\end{dels}

\begin{lydelse}
    \setcounter{subsection}{3}
    \item Fysikteknologsektionens hedersmedlemmar är
    \begin{itemize}
        \item Valen Åke som strandade i Träslövsläge.
        \item Schrödingers katt, kanske.
    \end{itemize}
    \setcounter{subsection}{1}
    
    \switchcolumn
    
    \item  Sektionens hedersmedlemmar är
\begin{itemize}
    \item Valen Åke som strandade i Träslövsläge.
    \item Schrödingers katt, kanske.\\
    {\footnotesize\itshape Under sektionsmötet 2013–05–07 genomfördes en omröstning om nämnda katts medlemskap. Omröstningen skedde enligt mötets önskan genom sluten votering, och rösterna placerades i ett kuvert, som sedan förseglades.
    Schrödingers katt är därmed kanske hedersmedlem.}
\end{itemize}
\end{lydelse}
\section{Sektionsmötet}
Syftet är redaktionellt samt att förteckna praxis som använts de senaste åren.

Härigenom föreskrivs i fråga om reglementets kapitel 2

\begin{dels}
    \item att kapitlets rubrik ska lyda ''Sektionsmötet''
\end{dels}

\subsection{Åligganden}
Syftet är redaktionellt.
Åligganden angående verksamhetsplaner som inte finns i stadga kodifieras.
Flera åligganden angående val har flyttat till stadgan.

Härigenom föreskrivs i fråga om reglementets avsnitt 2.4

\begin{dels}
    \item att avsnittets rubrik ska lyda ''Åligganden'',
    \item att avsnittet ska omnumreras 2.1,
    \item att avsnittet i sin helhet ska ha följande lydelse.
\end{dels}

\begin{lydelse}
    \item[] \emph{(se \S 2.3.1, \S 2.5.2)}
    \setcounter{subsection}{4}
    
    \item[]
    \item[]
    \vspace{12em}
    \item Det åligger sektionsmötet att innan utgången av läsperiod 1 utöver de i stadgarna definierade åliggandena välja:
    \begin{itemize}
			\item Årskursrepresentant till studienämnden
			\item Balnågonting
			\item Kräldjursvårdare  %flyttad enl sektmötesbeslut lp1 21/22
			\item Sektionsnörd      %se ovan
		\item Frisörer.         %se ovan
	\end{itemize}
    
    \item Det åligger sektionsmötet att innan utgången av läsperiod 2 utöver de i stadgarna definierade åliggandena välja:
	\begin{itemize}
		\item Förtroendeposter i FARM
		\item Ledamöter i FARM
		\item Förtroendeposter i FnollK
		\item Ledamöter i FnollK
		\item FIF.
    \end{itemize}

	\item Det åligger sektionsmötet att innan utgången av läsperiod 3 utöver de i stadgarna definierade åliggandena välja:
	\begin{itemize}
		\item Bilnissar
		\item Blodgrupp
		\item Fanfareri
		\item Finform
		\item Spidera
		\item Sångförmän
		\item Bakisclubben (BC)
		\item Game Boy
		\item Piff och Puff
		\item Foton
		\item Fabiola
		\item Mastermottagningsansvarig.
	\end{itemize}

	\item Det åligger sektionsmötet att innan utgången av  läsperiod 4 utöver de i stadgarna definierade åliggandena välja:
	\begin{itemize}
		\item Dragos
		\item Sektionsstyrelse
		\item Valberedning
		\item Studienämnden, förutom årskursrepresentant
		\item Förtroendeposter i Djungelpatrullen
		\item Adjutanter i Djungelpatrullen
		\item Förtroendeposter i F6
		\item Ledamöter i F6
		\item Förtroendeposter i Focumateriet
		\item Ledamöter i Focumateriet
		\item Revisorer
		\item Talmanspresidiet
		\item Fristående ledamöter i JämF.
	\end{itemize}
    \setcounter{subsection}{1}
    \switchcolumn
    
    \item \emph{Det åligger dessutom sektionsmötet att innan utgången av läsperiod 1}
    \begin{itemize}
        \item \emph{anta sammanträdesordning på förslag av föregående års talman},
        \item \emph{fastställa verksamhetsplan för studienämnden}.
    \end{itemize}
    
    \item \emph{Det åligger dessutom sektionsmötet att innan utgången av läsperiod 4}
    \begin{itemize}
        \item \emph{fastställa preliminär verksamhetsplan för sektionsstyrelsen,}
        \item \emph{fastställa preliminär budget för sektionen.}
    \end{itemize}
    
    \vspace{0.6em}
    \item Det åligger \emph{dessutom} sektionsmötet att innan utgången av läsperiod 1 välja
    \begin{itemize}
        \item Balnågonting
        \item Frisörer
        \item Kräldjursvårdare
        \item Sektionsnörd
        \item årskursrepresentant \emph{i} studienämnden
    \end{itemize}
    
    \vspace{1.2em}
    \item Det åligger \emph{dessutom} sektionsmötet att innan utgången av läsperiod 2 välja
    \begin{itemize}
        \item FARM
        \item FIF
        \item FnollK
    \end{itemize}

    \vspace{4.8em}
    \item Det åligger \emph{dessutom} sektionsmötet att innan utgången av läsperiod 3 välja
    \begin{itemize}
        \item Bakisclubben
        \item Bilnissar
        \item Blodgrupp
        \item Fabiola
        \item Fanfareri
        \item Finform
        \item Foton
        \item Game Boy
        \item Mastermottagningsansvarig
        \item Piff och Puff
        \item Spidera
        \item Sångförmän
    \end{itemize}

    \item[]
    \item Det åligger \emph{dessutom} sektionsmötet att innan utgången av läsperiod 4 välja
    \begin{itemize}
        \item Djungelpatrullen
        \item Dragos
        \item F6
        \item Focumateriet
        \item Studienämnden, \emph{exkl}. årskursrepresentant
        \item fristående ledamöter i JämF
        \item \emph{resterande ledamöter i Sektionsstyrelsen}
        \item []\emph{(valberedning, revisorer samt talmanspresidium flyttat till stadga \S \ref{S:SektmoteLP4Alagg})}
    \end{itemize}
\end{lydelse}

\subsection{Utlysning}
Syftet är redaktionellt.

Härigenom föreskrivs i fråga om reglementets avsnitt 2.1 och 2.2

\begin{dels}
    \item att avsnitten sammanfogas till ett nytt avsnitt,
    \item att avsnittets rubrik ska lyda ''Utlysning'',
    \item att avsnittet ska omnumreras 2.2,
    \item att avsnittet i sin helhet ska ha följande lydelse.
\end{dels}

\begin{lydelse}
    \setcounter{subsection}{1}
    \setcounter{enumi}{1}
    \item Kallelse till sektionsmöte skall tillsändas sektionsmedlemmar, revisorer, inspektor och kårledningen.
    \setcounter{enumi}{0}
    \item Kallelse till sektionsmöte  skall innehålla uppgifter om datum, tid och plats, samt preliminär föredragningslista. Denna skall anslås via sektionens officiella kommunikationskanaler.
    
    \item[] (se \S 2.5.8)
    
    \setcounter{subsection}{0}
    
    \switchcolumn
    \item Kallelse \emph{samt slutgiltig föredragningslista ska utöver anslag} tillsändas sektionsmedlemmar, revisorer, inspektor och kårledningen.
    \label{R:SektionsMAnslag}
    
    \item[] \emph{(ersätts av \S \ref{R:SektionsMAnslag}, avsnitt \ref{R:Tillkannagivande})}

    \vspace{9ex}
    \item Datum för ordinarie sektionsmöten fastställs av talmanspresidiet på ett styrelsemöte i samråd med styrelsen.
\end{lydelse}

\setcounter{subsection}{2}
\subsection{Personval}
Syftet är konsolidera bestämmelserna som berör personval under ett avsnitt.

Härigenom föreskrivs i fråga om reglementets kapitel 2

\begin{dels}
    \item att ett nytt avsnitt införs,
    \item att avsnittets rubrik ska lyda ''Personval'',
    \item att avsnittet ska numreras 2.3,
    \item att avsnittet i sin helhet ska ha följande lydelse.
\end{dels}

\begin{lydelse}
    \setcounter{subsection}{1}

    \item[] (se stadga \S 5.12.1)
    
    \vspace{1.2em}
    \item[] (se stadga \S 5.12.2)
    
    \vspace{3.4em}
    \item[] (se \S3.3.3, \S3.3.4)
    
    \setcounter{subsection}{0}
    
    \switchcolumn
    \item \emph{Vid personval där det finns fler kandidater än poster ska sluten votering användas.}
    
    \item \emph{Vid personval där två eller fler kandidater får lika röstetal som hade lett till val ska ny votering göras med dessa kandidater.
    Skulle på nytt lika röstetal uppkomma skiljer lotten.}
    
    \item \emph{Vid personval där godkänd gruppnominering enligt \ref{R:valb:grupp} avgetts ska nominerade anses valda enligt denna om sektionsmötet enligt valberedningens nominering väljer motsvarande förtroendeposter.
    I annat fall sker personval till dessa poster enligt vanlig ordning.}
    \label{R:SektmoteGruppnom}
\end{lydelse}


\section{Sektionsstyrelsen}

Härigenom föreskrivs i fråga om reglementets kapitel 4

\begin{dels}
    \item att kapitlets rubrik ska lyda ''Sektionsstyrelsen''
    \item att kapitlet ska omnumreras 3
\end{dels}

\subsection{Sammansättning}
Syftet är redaktionellt samt att eliminera redundans.
Villkoren för valbarhet ryms nu i stadga.

Härigenom föreskrivs i fråga om reglementets avsnitt 4.1

\begin{dels}
    \item att avsnittets rubrik ska lyda ''Sammansättning'',
    \item att avsnittet ska omnumreras 3.1,
    \item att avsnittet i sin helhet ska ha följande lydelse.
\end{dels}

\begin{lydelse}
    \setcounter{section}{4}
    
    \item Sektionsstyrelsen består av följande förtroendeposter:
	\begin{itemize}
		\item Sektionsordförande
		\item Vice sektionsordförande
		\item Sektionskassör
		\item Sekreterare
		\item Skyddsombud
		\item Informationsansvarig
		\item Ordförande i SNF
		\item Ordförande i FnollK
		\item Ordförande i F6
		\item Ordförande i FARM
		\item Ordförande i Focumateriet
		\item Ordförande i Djungelpatrullen
	\end{itemize}

    \vspace{1ex}
	\item Villkor för ledamöter i sektionsstyrelsen:
	\begin{itemize}
		\item En kan ej inneha fler än en position i sektionsstyrelsen.
		\item Revisor, talmanspresidiet eller ledamot av valberedningen kan ej vara medlem av sektionsstyrelsen.
		\item Ledamot av sektionsstyrelsen kan ej inneha annan förtroendepost.
	\end{itemize}

	\item Suppleanter
	\label{R:StyretViceSuppleant}
	\begin{itemize}
		\item Vice ordförande i respektive kommitté samt studienämnden är suppleant i sektionsstyrelsen.
		\item Suppleant övertar ordförandens befogenheter vid styrelsemöten vid ordförandens bortfall, undantaget då styrelsemötet hålls bakom stängda dörrar.
	\end{itemize}
	
    \setcounter{section}{3}
    \switchcolumn
    
    \item Sektionsstyrelsen består av följande poster:
    \begin{itemize}
        \item Sektionsordförande
    	\item Vice sektionsordförande
    	\item Sektionskassör
    	\item Sekreterare
    	\item Skyddsombud
    	\item Informationsansvarig
    	\item Ordförande i \emph{studienämnden}
    	\item Ordförande i FnollK
    	\item Ordförande i F6
    	\item Ordförande i FARM
    	\item Ordförande i Focumateriet
    	\item Ordförande i Djungelpatrullen
    \end{itemize}
    \emph{varav samtliga är förtroendeposter.}
    
    \vspace{0.4em}
    \item[] \emph{(ersätts av stadga \S \ref{S:ValbarhetMaxEnFortroende}, stadga \S \ref{4.x:valbar.dubbel})}
    
    \item[]
    \item[]
    \item[]
    \item[]
    \item[]
    
    \vspace{3ex}
    \item Vice ordförande i respektive kommitté samt studienämnden är suppleant i sektionsstyrelsen.
    
    \vspace{1.2em}
    \item Suppleant övertar \emph{sin ordinaries} befogenheter vid styrelsemöten vid bortfall, undantaget då styrelsemötet hålls bakom stängda dörrar.
\end{lydelse}


\subsection{Styrelsemöten}
Syftet är mestadels redaktionellt. 
Kravet på minst 3 sammanträden per läsperiod är förlegat och tas bort.
Det förtydligas att mötesdeltagare ej kan adjungeras in med rösträtt.

Härigenom föreskrivs i fråga om reglementets avsnitt 4.2

\begin{dels}
    \item att avsnittets rubrik ska lyda ''Styrelsemöten'',
    \item att avsnittet ska omnumreras 3.2,
    \item att avsnittet i sin helhet ska ha följande lydelse.
\end{dels}

\begin{lydelse}
    \setcounter{section}{4}
    \setcounter{subsection}{2}
    \setcounter{enumi}{0}
    \item Sektionsstyrelsen skall sammanträda minst 3 gånger per läsperiod.
    
    \item Kallelse till styrelsemöte anslås av sektionsordföranden.

    \item Kallelse till styrelsemöte ska senast två dagar innan mötet skickas till ordinarie ledamöter av sektionsstyrelsen, medlem av kommitté och nämnd, revisorer, talmanspresidiet, valberedningen samt övriga berörda sektionsföreningsmedlemmar och funktionärer.
    
    \item Protokoll från styrelsemöten anslås via någon av sektionens officiella kommunikationskanaler.
	\setcounter{enumi}{8}	
    \item Varje medlem har rätt att få fråga behandlad på styrelsemöte.
    Sådan fråga skickas till styrelsen senast tre dagar innan mötet.
    \switchcolumn
    \setcounter{section}{3}
    \setcounter{subsection}{2}
    \setcounter{enumi}{0}
    \item[] ---
    
	\item[] \emph{(ersätts av stadga \S\ref{S:Styretkallelse})}

	\item Kallelse till styrelsemöte skall senast två dagar innan mötet tillsändas ordinarie ledamöter av sektionsstyrelsen, medlem av kommitté och nämnd, revisorer, talmanspresidiet, valberedningen samt övriga berörda sektionsföreningsmedlemmar och funktionärer.
	
	\item[] \emph{(ersätts av \S \ref{R:Tillkannagivande})}
	
	\item Varje medlem har rätt att få fråga behandlad på styrelsemöte.\emph{ Denna ska då} skickas till styrelsen senast tre dagar innan mötet.
\end{lydelse}
\textbf{Rättigheter}
\begin{lydelse}
    \setcounter{section}{4}
    \setcounter{subsection}{2}
    \setcounter{enumi}{4}
    \item Ordinarie ledamöter av sektionsstyrelsen har \\ närvaro-, yttrande-, förslags- och rösträtt. 
    \item Talmannen har närvaro-, yttrande- och förslagsrätt. Vid frågor som direkt berör sektionsmöten har talmannen även rösträtt. Vid talmannens frånvaro övertar vice talman dennes befogenheter. 
    \item Revisor och ledamöter av valberedningen har \\ närvaro-, yttrande- och förslagsrätt.
    \item Ledamöter av kommittéer och nämnd, samt sektionens inspektor, har närvaro- och yttranderätt.
    \item[] ---
    \setcounter{section}{3}
    \switchcolumn
    \setcounter{section}{3}
    \setcounter{subsection}{2}
    \setcounter{enumi}{3}
    \item Ordinarie ledamöter av sektionsstyrelsen har \\ närvaro-, yttrande-, förslags- och rösträtt. 

    \item Talman, \emph{revisor och ledamöter av valberedningen} har närvaro-, yttrande- och förslagsrätt.
    \label{R:StyretMoteTRV}
    
    \vspace{2.4em}
    \item[] \emph{(ersätts av \S \ref{R:StyretMoteTRV})}
    
    \vspace{1.2em}
    \item Ledamot av kommitté och \emph{studienämnden} har närvaro- och yttranderätt. %Note: saknas sektionens inspektor?
    		
    \item \emph{Mötesdeltagare kan ej adjungeras in med rösträtt.}
\end{lydelse}
\textbf{Bakom stängda dörrar}
\begin{lydelse}
    \setcounter{section}{4}
    \setcounter{enumi}{9}
    \item Bakom stängda dörrar
    \begin{itemize}
		\item Sektionsstyrelsen kan, om synnerliga skäl föreligger, för visst ärende med minst 2/3 majoritet besluta att överläggning sker bakom stängda dörrar.
		\item Enbart ordinarie ledamöter av sektionsstyrelsen äger närvarorätt vid möte bakom stängda dörrar.
		\item Sektionsstyrelsen kan adjungera in övriga deltagare.
		\item Endast beslutsprotokoll fört då mötet hålls bakom stängda dörrar skall anslås.
		\item Det som diskuteras bakom stängda dörrar får ej föras vidare till tredje part. 
	\end{itemize}
    \setcounter{section}{3}
    \switchcolumn
    
    \vspace{4ex}
    \item Sektionsstyrelsen kan, om synnerliga skäl föreligger, för visst ärende med minst \sfrac{2}{3} majoritet besluta att överläggning sker bakom stängda dörrar.

    \item Enbart ordinarie ledamöter av sektionsstyrelsen äger närvarorätt vid möte bakom stängda dörrar.
    \emph{Övriga deltagare kan adjungeras in.}
    \label{R:BSDAdjungering}
    
    \item[]
    
    \vspace{3ex}
    \item Endast beslutsprotokoll förs då mötet hålls bakom stängda dörrar ska anslås. %Note: Stavning
    
    \item Det som diskuteras bakom stängda dörrar får ej föras vidare till tredje part. 
\end{lydelse}


\subsection{Åligganden}
Syftet är eliminera redundans.

Härigenom föreskrivs i fråga om reglementets avsnitt 4.3

\begin{dels}
    \item att avsnittets rubrik ska lyda ''Åligganden'',
    \item att avsnittet ska omnumreras 3.3,
    \item att avsnittet i sin helhet ska ha följande lydelse.
\end{dels}

\begin{lydelse}
    \setcounter{section}{4}
    \item Det åligger sektionsstyrelsen:
	\begin{description}
		\item[att] verka för sammanhållningen mellan sektionsmedlemmarna och verka för deras gemensamma intressen.
		\item[att] leda sektionens arbete.
		\item[att] verkställa och övervaka genomförandet av sektionsmötesbeslut.
		\item[att] framlägga budget till sektionsmötet.
		\item[att] planera Fysikteknologsektionens framtida inriktning och verksamhet.
		\item[att] fatta beslut i de ärenden som framlägges till sektionsstyrelsen.
		\item[att] varje år tillsammans med studienämnden utse sektionens representanter i styrelser och kommittéer inom högskolan. Dock krävs för ledamöterna i Programråden godkännande från sektionsmötet.
		\item[att] delta i de av Djungelpatrullen anordnade städdagarna två gånger per läsår. Om detta uppfylls får de närvarande gå gratis på nästa sektionsaktivafest.
		\item[att] i samråd med de som önskar söka sektionsstyrelsen ta fram en preliminär verksamhetsplan och presentera denna på sektionsmöte innan utgången av läsperiod 4.
		\item[att] följa de av sektionen upprättade arbetsordningarna.
	\end{description}

    \item Det åligger sektionsstyrelsens ordförande:
	\begin{description}
		\item[att] tillse att sektionens beslut verkställs.
		\item[att] föra sektionens talan då något annat ej stadgats eller beslutats.
		\item[att] teckna sektionens firma.
		\item[att] leda och övervaka arbetet inom sektionsstyrelsen.
		\item[att] vara Fysikteknologsektionens representant i kårledningsutskottet.
		\item[att] tillse att det finns representanter från sektionsstyrelsen i F:s och TM:s programråd.
		\item[att] tillsammans med sektionskassören ansvara för Fysikteknologsektionens ekonomi.
	\end{description}
    \noindent Sektionsordförande har full insyn i Fysikteknologsektionens alla organ och äger rätt att deltaga i deras möten med yttranderätt.

    \item Det åligger sektionsstyrelsens vice ordförande:
	\begin{description}
		\item[att] i ordförandes frånvaro överta dennes åligganden.
		\item[att] biträda ordföranden.
		\item[att] i samråd med styrelsen och övriga sektionsaktiva upprätta sektionens verksamhetsberättelse.
	\end{description}

    \item Det åligger sektionsstyrelsens  kassör:
	\begin{description}
		\item[att] sköta och ansvara för Fysikteknologsektionens ekonomi tillsammans med ordföranden.
		\item[att] fortlöpande kontrollera kommittéernas samt eventuella sektionföreningars räkenskaper och bokföring.
		\item[att] teckna sektionens firma.
		\item[att] genom Chalmers Studentkår uppbära sektionsavgiften.
		\item[att] i samråd med sektionsstyrelsen upprätta preliminärt budgetförslag till första ordinarie höstmötet.
		\item[att] till varje sektionsmöte kunna redogöra för sektionens ekonomiska ställning.
	\end{description}

    \item Det åligger sektionsstyrelsens sekreterare:
	\begin{description}
		\item[att] föra protokoll vid styrelsemöten och senast två läsdagar efter möte överräcka renskrivet protokoll till ordföranden.
		\item[att] tillse att protokoll från styrelsemöten anslås.
		\item[att] tillse att sektionens stadgar, reglemente och förordningar är aktuella och efterlevs.
	\end{description}

    \item Det åligger sektionsstyrelsens skyddsombud:
	\begin{description}
		\item[att] vara sektionens studerandearbetsmiljöombud samt vara Fysikteknologsektionens jämlikhetsansvarige. SAMO har i sitt uppdrag tystnadsplikt.
		\item[att] tillvarata sektionsmedlemmarnas intressen i skyddsfrågor, jämlikhets- och jämställdhetsfrågor och samarbeta med Chalmers Studentkårs kontaktperson samt högskolans skyddsombud.
		\item[att] deltaga på studienämndens möten då arbetsmiljöfrågor behandlas.
		\item[att] leda arbetet i sektionens jämlikhetsråd, JämF.
	\end{description}

    \item Det åligger sektionsstyrelsens informationsansvariga:  
	\begin{description}
		\item[att] sköta kontakt med sektionen samt uppdatera sektionens hemsida. 
		\item[att] tillse att material som inkommer till sektionen anslås eller på annat sätt förmedlas  till den/dem det berör.
		\item[att] vara sektionsstyrelsens representant i Spidera.
	\end{description}    

    \item Det åligger sektionsstyrelsens övriga ledamöter:
	\begin{description}
		\item[att] bistå sektionsstyrelsen med information.
		\item[att] aktivt deltaga i beslutsprocessen.
		\item[att] redogöra för sin kommittés/studienämnds löpande verksamhet vid styrelsemöten.
	\end{description}
	\setcounter{section}{3}
    \switchcolumn
    \setcounter{enumi}{0}
    \item Det åligger sektionsstyrelsen
    \begin{aligganden}
        \item verka för sammanhållningen mellan sektionsmedlemmarna och deras gemensamma intressen.
    	\item leda sektionens arbete.
    	\item verkställa och övervaka genomförandet av sektionsmötesbeslut.
    	\item framlägga budget till sektionsmötet.
    	\item planera \emph{sektionens} framtida inriktning och verksamhet.
    	\item fatta beslut i de ärenden som framlägges till sektionsstyrelsen.
    	\item varje år tillsammans med studienämnden utse sektionens representanter i styrelser och kommittéer inom högskolan.
    	    \emph{Ledamöterna i programråden fastställs av sektionsmötet.} \vspace{1.6em}
    	\item[] \emph{(redundant med arbetsordning)}\vspace{8.7ex}
    	\item i samråd med de som önskar söka sektionsstyrelsen ta fram en preliminär verksamhetsplan och presentera denna på sektionsmöte innan utgången av läsperiod 4.\vspace{0.9ex}
    	\item[] \emph{(implicit från \ref{R:ForteckningStyrdokument})}
    \end{aligganden}

    \vspace{0.9em}
    \item Det åligger sektionsstyrelsens ordförande:
    \begin{aligganden}
        \item tillse att sektionens beslut verkställs.
        \item föra sektionens talan då något annat ej stadgats eller beslutats.
        \item[] \emph{(ersätts av stadga \S \ref{S:EkonomiSektFirma})}\vspace{0.3em}
        \item leda och övervaka arbetet inom sektionsstyrelsen.\vspace{0.2em}
        \item vara \emph{sektionens} representant i kårledningsutskottet.
        \item tillse att det finns representanter från sektionsstyrelsen i \emph{programråden för sektionens program}.
        \item[] \emph{(ersätts av stadga \S \ref{S:SektionsstyrelseEkonomisktAnsvariga})}\vspace{1.2em}
        \item[] \emph{(redundant efter ansvarsfördelning i stadgans avsnitt \ref{4.2:ansvar})}
    \end{aligganden}

    \vspace{0.5em}
    \item Det åligger sektionsstyrelsens vice ordförande:
    \begin{aligganden}
        \vspace{-0.2em}
        \item[] \emph{(ersätts av stadga \S \ref{S:ViceLaRevolution})}\vspace{1.2em}
        \item[] \emph{(redundant)}\vspace{0.2em}
        \item i samråd med styrelsen och övriga sektionsaktiva upprätta sektionens verksamhetsberättelse.
    \end{aligganden}


    \item Det åligger sektionsstyrelsens  kassör:
    \begin{aligganden}
        \vspace{-0.2em}
        \item[] \emph{(ersätts av stadga \S \ref{S:SektionsstyrelseEkonomisktAnsvariga})}\vspace{1.4em}
        \item fortlöpande kontrollera kommittéernas samt eventuella sektionföreningars räkenskaper och bokföring.\vspace{0.2em}
        \item[] \emph{(ersätts av stadga \S \ref{S:EkonomiSektFirma})}\vspace{0.2em}
        \item genom Chalmers studentkår uppbära sektionsavgiften.
        \item i samråd med sektionsstyrelsen upprätta preliminärt budgetförslag till första ordinarie höstmötet.
        \item till varje sektionsmöte kunna redogöra för sektionens ekonomiska ställning.
    \end{aligganden}
    
    \vspace{-0.1em}
    \item Det åligger sektionsstyrelsens sekreterare:\vspace{-0.4em}
    \begin{aligganden}
        \item föra protokoll vid styrelsemöten och senast två läsdagar efter möte överräcka renskrivet protokoll till ordföranden.\vspace{0.3em}
        \item tillse att protokoll från styrelsemöten anslås.\vspace{0.2em}
        \item tillse att sektionens stadgar, reglemente och förordningar är aktuella och efterlevs.
    \end{aligganden}
    
    \vspace{-0.2em}
    \item Det åligger sektionsstyrelsens skyddsombud:
    \begin{aligganden}
        \item \emph{vara sektionens studerandearbetsmiljöombud samt vara sektionens jämlikhetsansvarige.}
        \item \emph{lyda under tystnadsplikt i sitt uppdrag.}\vspace{0.8em}
        \item tillvarata sektionsmedlemmarnas intressen i skyddsfrågor, jämlikhets- och jämställdhetsfrågor och samarbeta med Chalmers studentkårs kontaktperson samt högskolans skyddsombud.\vspace{0.2em}
        \item deltaga på studienämndens möten då arbetsmiljöfrågor behandlas.
        \item vara sektionsstyrelsens representant \emph{i} JämF.
    \end{aligganden}
    
    \item Det åligger sektionsstyrelsens informationsansvariga:  
    \begin{aligganden}
        \item sköta kontakt med sektionen samt uppdatera sektionens hemsida. 
        \item tillse att material som inkommer till sektionen anslås eller på annat sätt förmedlas till \emph{berörda parter.}\vspace{0.2em}
        \item vara sektionsstyrelsens representant i Spidera.
    \end{aligganden}
    
    \item[] \emph{(ersätts av \S\ref{R:KommitteOrdfRepr})}
\end{lydelse}


\section{Studienämnden}
Syftet är redaktionellt samt att eliminera redundans.
I synnerhet tas begreppet ''nämnder'' bort eftersom studienämnden är den enda s.k. nämnden sedan länge.

Härigenom föreskrivs i fråga om reglementets kapitel 5

\begin{dels}
    \item att kapitlets rubrik ska lyda ''Studienämnden'',
    \item att avsnitt 5.1 ska uppgå i kapitlet,
    \item att \S 5.1.1.1 -- \S 5.1.1.3, \S 5.1.3.1 ska utgå,
    \item att kapitlet ska omnumreras 4
\end{dels}

\begin{lydelse}

    \item[5.1.1.1] En person kan ej inneha två poster i studienämnden samtidigt.
	\item[5.1.1.2] Ekonomiskt ansvarig i studienämnden skall vara myndig.
	\item[5.1.1.3] En person kan ej inneha ordförandepost i en kommitté eller nämnd samtidigt som denne innehar vice ordförandepost i annan kommitté eller nämnd.
	\item[5.1.3.1] Studienämnden har en verksamhet som går från 1 juli till 30 juni.
    \switchcolumn
    \item[] \emph{(ersätts av stadga \S\ref{4.x:valbar.dubbel})} \vspace{1.3em}
    \item[] \emph{(ersätts av stadga \S\ref{S:ValbarEkonomiskMyndig})} \vspace{1em}
    \item[] \emph{(ersätts av stadga \S\ref{4.x:valbar.dubbel})}\vspace{2.5em}
    \item[] \emph{(ersätts av \S\ref{R:VanligtVerksamhetsar})}
    
\end{lydelse}

\subsection{Sammansättning}
Syftet är redaktionellt.

Härigenom föreskrivs i fråga om reglementets avsnitt 5.1.2

\begin{dels}
    \item att avsnittets rubrik ska lyda ''Sammansättning'',
    \item att avsnittet ska omnumreras 4.1,
    \item att avsnittet i sin helhet ska ha följande lydelse.
\end{dels}

\begin{lydelse}

	\item[5.1.2.1] Studienämnden består av följande poster:
	\begin{itemize}
		\item Ordförande
		\item Vice ordförande
		\item Kassör
		\item Sekreterare
		\item Kandidatansvarig
		\item Masteransvarig
		\item Årskursrepresentant åk. 1
		\item Veckobladerist
		\item Matansvarig. 
	\end{itemize}
    Där ordförande, vice ordförande och kassör anses vara förtroendevalda.

    \switchcolumn
    \item Studienämnden består av följande poster:
    \begin{itemize}
        \item Ordförande
    	\item Vice ordförande
    	\item Kassör
    	\item Sekreterare
    	\item Kandidatansvarig
    	\item Masteransvarig
    	\item Årskursrepresentant åk. 1
    	\item Veckobladerist
    	\item Matansvarig %Note: ta bort punkt här
    	\end{itemize}
   \emph{varav ordförande, vice ordförande och kassör är förtroendeposter.}

\end{lydelse}

\subsection{Åligganden}
Syftet är mestadels redaktionellt.
Kravet på minst 3 sammanträden per läsperiod slopas eftersom det är en kopia från styrelsemötena.


Härigenom föreskrivs i fråga om reglementets avsnitt 5.1.4

\begin{dels}
    \item att avsnittets rubrik ska lyda ''Åligganden'',
    \item att avsnittet ska omnumreras 4.2,
    \item att avsnittet i sin helhet ska ha följande lydelse.
\end{dels}

\begin{lydelse}

    \item[5.1.4.1] Det åligger studienämnden:
	\begin{description}
		\item[att]  sammanträda minst tre gånger per läsperiod.
		\item[att] ansvara för utvecklandet av utbildningsbevakningen på Fysikteknologsektionen.
		\item[att] inför Fysikteknologsektionen svara för att F- och TM-teknologernas intressen i studiefrågor och studiemiljö bevakas på ett tillfredsställande sätt.
		\item[att] tillse att det finns representant från studienämnden i programråden för F och TM.\vspace{1em}
		\item[att] på verksamhetsårets första sektionsmöte presentera en verksamhetsplan för det kommande läsåret.   
		\item[att] följa åligganden enligt arbetsbeskrivning.
	\end{description}

    \item[5.1.4.2] Det åligger studienämndens ordförande:
	\begin{description}
		\item[att] tillse att studienämndens ålägganden utförs.
		\item[att] leda studienämndens verksamhet.
		\item[att] kalla studienämnden till sammanträde.
		\item[att] tillsammans med kassören ansvara för studienämndens ekonomi.
		\item[att] representera studienämnden i sektionsstyrelsen.
		\item[att] i studie- och studiemiljöfrågor representera Fysikteknologsektionen och föra dess talan.
		\item[att] representera F och TM i Utbildningsutskottet, UU.
		\item[att] inför Fysikteknologsektionen svara för att F- och TM-teknologernas intressen i studiefrågor och studiemiljö bevakas på ett tillfredsställande sätt.
	\end{description}

    \item[5.1.4.3] Det åligger studienämndens vice ordförande:
	\begin{description}
		\item[att] assistera ordföranden i dennes åligganden.
		\item[att] ersätta ordföranden när denne inte är närvarande.
		\item[att] ansvara för studiesociala evenemang.
		\item[att] vara studienämndens suppleant i sektionsstyrelsen och Utbildningsutskottet.
	\end{description}

    \item[5.1.4.4] Det åligger studienämndens kassör:
	\begin{description}
		\item[att] sköta och ansvara för studienämndens ekonomi tillsammans med ordförande.
		\item[att] kontinuerligt föra en granskningsbar redovisning gällande studienämnden:s ekonomi.
		\item[att] mot revisorerna och sektionskassören kontinuerligt redovisa den ekonomiska situationen. 
	\end{description}

    \item[5.1.4.5] Det åligger studienämndens sekreterare:
	\begin{description}
 		\item[att] tillse att protokoll förs på studienämndens möten.
		\item[att] anslå nämndens protokoll enligt stadgarna.
	\end{description}

    \item[5.1.4.6] Det åligger studienämndens medlemmar:
	\begin{description}
		\item[att] vara ordföranden behjälplig.
	\end{description}
	\setcounter{section}{4}
    \switchcolumn
    \setcounter{enumi}{0}
    
    \item Det åligger studienämnden:
    \begin{aligganden}
        \item[] --- \vspace{1.2em}
        \item ansvara för utvecklandet av utbildningsbevakningen på \emph{sektionen}.
        \item inför \emph{sektionen} svara för att \emph{teknologerna på sektionens programs} intressen i studiefrågor och studiemiljö bevakas på ett tillfredsställande sätt.\vspace{0.2em}
        \item tillse att det finns representant från studienämnden i programråden för sektionens program.\vspace{0.2em}
        \item på verksamhetsårets första sektionsmöte presentera en verksamhetsplan för det kommande läsåret.\vspace{0.2em}
        \item[] \emph{(implicit från \ref{R:ForteckningStyrdokument})}
    \end{aligganden}
    
    \vspace{-0.2em}
    \item Det åligger studienämndens ordförande:
    \begin{aligganden}
        \vspace{-0.2em}
        \item[] \emph{(redundant)}\vspace{0.2em}
        \item leda studienämndens verksamhet.
        \item kalla studienämnden till sammanträde. \vspace{0.5em}
        \item[] \emph{(ersatt av stadga \S \ref{S:StudienamndEkonomiskt})}\vspace{1.4em}
        \item representera studienämnden i sektionsstyrelsen.\vspace{0.2em}
        \item i studie- och studiemiljöfrågor representera \emph{sektionen} och föra dess talan.\vspace{0.2em}
        \item representera \emph{sektionen} i Utbildningsutskottet, UU.
        \item[] \emph{(redundant)} %Note: Finns ett sådant krav för kommittéer men ej snf
    \end{aligganden}
    
    \item[]
    \item[]
    \vspace{3em}
    \item[] \vspace{-0.5em}
    
    \item Det åligger studienämndens vice ordförande:
    \begin{aligganden}
        
        \item[] \emph{(redundant)}
        \item[] \emph{(redundant)}\vspace{1.4em}
        \item ansvara för studiesociala evenemang.
        \item[] \emph{(redundant)}
    \end{aligganden}
    
    \vspace{1.2em}
    \item Det åligger studienämndens kassör:
    \begin{aligganden}
        \item[] \emph{(ersätts av stadga \S \ref{S:StudienamndEkonomiskt})}\vspace{1.2em}
        \item[] \emph{(ersätts av stadga \S \ref{S:RedovisningKont})}\vspace{1.3em}
        \item mot revisorerna och sektionskassören kontinuerligt redovisa den ekonomiska situationen. 
    \end{aligganden}
    
    \item Det åligger studienämndens sekreterare:
    \begin{aligganden}
         \item tillse att protokoll förs på studienämndens möten.
         \item[] \emph{(ersätts av stadga \S \ref{S:StudienamndProtokoll})}
    \end{aligganden}
    \begin{aligganden}
        \item[] \emph{(redundant)}
    \end{aligganden}
\end{lydelse}

\section{Kommittéer}
Syftet är redaktionellt samt att eliminera redundans.
Kraven på valbarhet till poster ersätts av avsnittet om valbarhet i stadgan.

Härigenom föreskrivs i fråga om reglementets kapitel 6

\begin{dels}
    \item att kapitlets rubrik ska lyda ''Kommittéer''
    \item att avsnitt 6.1 ska utgå,
    \item att kapitlet ska omnumreras 5
\end{dels}

Förklaring av strukna punkter:
\begin{lydelse}
    \setcounter{section}{6}
    \setcounter{subsection}{1}
    \item En person kan ej inneha två ledamotsposter i samma kommitté.
	\item Ekonomiskt ansvarig i kommitté skall vara myndig.
	\item En person kan ej inneha ordförandepost i en kommitté eller nämnd samtidigt som denne innehar vice ordförandepost i annan kommitté eller nämnd.
	\setcounter{section}{5}
    \switchcolumn
    
    \item[] \emph{(ersätts av stadga \ref{S:ValbarhetMaxEnFortroende})}\vspace{1.2em}
    \item[] \emph{(ersätts av stadga \ref{S:ValbarEkonomiskMyndig})}\vspace{1.2em}
    \item[] \emph{(ersätts av stadga \ref{S:ValbarhetMaxEnFortroende})}
    
\end{lydelse}

\subsection{Förteckning}
Syftet är redaktionellt.
Kommittéernas beskrivande texter blir egna punkter under respektive kommittées avsnitt.

Härigenom föreskrivs i fråga om reglementets avsnitt 6.2

\begin{dels}
    \item att avsnittets rubrik ska lyda ''Förteckning'',
    \item att avsnittet ska omnumreras 5.1,
    \item att avsnittet i sin helhet ska ha följande lydelse.
\end{dels}

\begin{lydelse}
    \setcounter{section}{6}
    \setcounter{subsection}{2}
    \item Fysikteknologsektionens sektionskommittéer är:
	\begin{itemize}
		\item Fysikteknologsektionens arbetsmarknadsgrupp, FARM.
		\item Fysikteknologsektionens mottagningskommitté, FnollK.
		\item Fysikteknologsektionens sexmästeri, F6.
		\item Fysikteknologsektionens PR-förening och rustmästeri, Djungelpatrullen.
		\item Fysikteknologsektionens focumateri, Focumateriet. 
	\end{itemize}
	\setcounter{section}{5}
	\setcounter{subsection}{1}
    \switchcolumn
    
    \item \emph{Sektionens} kommittéer är:
    \begin{itemize}
        \item \emph{FARM}
    	\item \emph{FnollK}
    	\item \emph{F6}
    	\item \emph{Djungelpatrullen}
    	\item \emph{Focumateriet}
    \end{itemize}
\end{lydelse}

\subsection{Allmänna åligganden}
Syftet är att konsolidera allmänna åliggnaden som varit utspridda.

Härigenom föreskrivs i fråga om reglementets avsnitt 6.4

\begin{dels}
    \item att avsnittets rubrik ska lyda ''Allmänna åligganden'',
    \item att avsnittet ska omnumreras 5.2,
    \item att avsnittet i sin helhet ska ha följande lydelse.
\end{dels}

\begin{lydelse}
    \setcounter{section}{6}
    \setcounter{subsection}{3}
    
    \item Det åligger varje sektionskommitté:
	\begin{description}
		\item[att] på det ordinarie sektionsmöte som följer på det att kommittens verksamhetsår har börjat presentera en verksamhetsplan för det kommande verksamhetsåret.     
		\item[att] följa åligganden enligt arbetsbeskrivning. 
	\end{description}
    
    \vspace{8em}
    \item Det åligger kassören i varje sektionskommitté:
	\begin{description}
		\item[att]  mot revisorerna och sektionskassören kontinuerligt redovisa för den ekonomiska situationen.
	\end{description}
	
    \setcounter{section}{5}
    \setcounter{subsection}{2}
    \switchcolumn
    
    \item Det åligger varje \emph{kommitté}:
    \begin{aligganden}
        \vspace{-0.4em}
        \item \emph{inom en läsperiod efter dess inval presentera en verksamhetsplan för det kommande verksamhetsåret för sektionsstyrelsen.}
        \item[] \item[] \emph{(implicit från \ref{R:ForteckningStyrdokument})}
    \end{aligganden}
    
    \vspace{-0.3em}
    \item \emph{Det åligger ordföranden i varje kommitté:}
    \begin{aligganden}
        \item \emph{leda kommitténs arbete.}
        \label{R:KommitteOrdfLeda}
        \item \emph{fungera som en kontaktlänk mellan kommittén och andra organ samt företräda kommittén i sektionsstyrelsen.}
        \label{R:KommitteOrdfRepr}
    \end{aligganden}
    
    \item Det åligger kassören i varje \emph{kommitté}:
    \begin{aligganden}
        \vspace{-0.3em}
        \item mot revisorerna och sektionskassören kontinuerligt redovisa för den ekonomiska situationen.
    \end{aligganden}
\end{lydelse}
\subsection{FARM}
Syftet är redaktionellt samt att eliminera redundans.
Många implicita åligganden stryks.

Härigenom föreskrivs i fråga om reglementets avsnitt 6.4

\begin{dels}
    \item att avsnittets rubrik ska lyda ''FARM'',
    \item att avsnittet ska omnumreras 5.3,
    \item att avsnittet i sin helhet ska ha följande lydelse.
\end{dels}

\begin{lydelse}
    \setcounter{section}{6}
    \setcounter{subsection}{4}
    \item[] \emph{(Se \S6.2.1)}
    
    \item FARM består av följande poster:
	\begin{itemize}
		\item Ordförande
 		\item Vice ordförande
		\item Kassör
		\item 0--5 ledamöter
	\end{itemize}
    där ordförande, vice ordförande och kassör är förtroendeposter.
    \item FARM har en verksamhet som går från 1 januari till 31 december.

	\item Vid räkenskapsårets slut skall tillgångar över 0,659 basbelopp tillfalla sektionen. Visst utrymme för representation får förekomma.
    \item Det åligger FARM:
	\begin{description}
		\item[att] arrangera studiebesök och branschkvällar.   
		\item[att] informera företag om programmen Teknisk fysik samt Teknisk matematik, och deras fördelar.
	\end{description}
	
	\item Det åligger FARM:s ordförande:
	\begin{description}
		\item[att] tillse att kommitteens åligganden utförs.
		\item[att] leda FARM:s arbete.
		\item[att] tillsammans med kassören ansvara för FARM:s ekonomi.
		\item[att] fungera som kontaktlänk mellan FARM och övriga kommittéer samt företräda FARM i sektionsstyrelsen.
	\end{description}

    \item Det åligger FARM:s vice ordförande:
	\begin{description}
		\item[att] vara kommitténs suppleant i sektionsstyrelsen.
		\item[att] hjälpa FARM-ordföranden i dennes uppgifter så att de utförs på bästa sätt.
	\end{description}

    \item Det åligger FARM:s kassör:
	\begin{description}
		\item[att] tillsammans med ordförande ansvara för och sköta FARM:s ekonomi.
		\item[att] kontinuerligt föra en granskningsbar redovisning gällande FARM:s ekonomi.
	\end{description}

    \item Det åligger FARM:s övriga medlemmar:
	\begin{description}
		\item[att] hjälpa FARM- ordföranden i dennes uppgifter så att de utförs på bästa sätt.
	\end{description}
	\setcounter{section}{5}
    \setcounter{subsection}{3}
    \switchcolumn 
    \item \emph{FARM är sektionens arbetsmarknadsgrupp.}

    \item FARM består av följande poster:
    \begin{itemize}
        \item Ordförande
     	\item Vice ordförande
    	\item Kassör
    	\item \emph{5 ledamöter}
    \end{itemize}
    \emph{varav} ordförande, vice ordförande och kassör är förtroendeposter.
    
    \item[] \emph{(ersätts av \S \ref{R:BrutnaVerksamhetsar})} \vspace{1.2em}
    
    \item[] \emph{(redundant)}\vspace{2.4em}
    
    \item Det åligger FARM:
    \begin{aligganden}
        \vspace{-0.3em}
        \item arrangera studiebesök och branschkvällar. \vspace{0.2em}
        \item informera företag om \emph{sektionens program} och deras fördelar.
    \end{aligganden}
    
    \vspace{1.2em}
    \item[] \emph{(ersätts av \S \ref{R:KommitteOrdfLeda}, \S \ref{R:KommitteOrdfRepr}, stadga \S \ref{S:KomitteEkonomisktAnsvar})} 
    
    \vspace{12.5em}
    
    \item[] \emph{(ersätts av \S \ref{R:StyretViceSuppleant})}
    
    \vspace{6em}
    \item[] \emph{(ersätts av \S \ref{S:KomitteEkonomisktAnsvar})}\vspace{1.2em}
    \item[] \emph{(ersätts av \S \ref{S:RedovisningKont})}
    
    \vspace{1.2em}
    \item[] \emph{(redundant)}
    
\end{lydelse}

\subsection{FnollK}
Syftet är redaktionellt samt att eliminera redundans.
Många implicita åligganden stryks.
Schrödingers katt-memen rättas i sann fysikeranda.

Härigenom föreskrivs i fråga om reglementets avsnitt 6.5

\begin{dels}
    \item att avsnittets rubrik ska lyda ''FnollK'',
    \item att avsnittet ska omnumreras 5.4,
    \item att avsnittet i sin helhet ska ha följande lydelse.
\end{dels}

\begin{lydelse}
    
    \item[] \emph{(se \S6.2.1)}
    
    \setcounter{section}{6}
    \setcounter{subsection}{5}
    
    \item FnollK består av följande poster:
	\begin{itemize}
		\item Ordförande
		\item Vice ordförande
		\item Kassör
		\item 0--4 ledamöter
	\end{itemize}
    där ordförande, vice ordförande och  kassör förtroendeposter.
    
    \item FnollK har en verksamhet som går från 1 januari till 31 december.

	\item Vid räkenskapsårets slut skall tillgångar över 0,659 basbelopp tillfalla sektionen. Visst utrymme för representation får förekomma.
    
    \item Det åligger FnollK:
	\begin{description}
		\item[att] genomföra en värdig mottagning i enlighet med kårens och sektionens intentioner.\vspace{0.8em}
		\item[att] arrangera aktiviteter för F- och TM-nollan som syftar till att införliva Nollan i livet som F- respektive TM-teknolog både vad gäller studier och det studiesociala livet.
		\item[att] under ledning av FnollK:s  ordförande planera och leda dessa aktiviteter i samråd med berörda organ.
		\item[att] tillse att valen Åke bär nollbricka om gamble så tycker.
		\item[att] kanske tillse att Schrödingers katt kanske bär nollbricka.
	\end{description}
    
    \vspace{0.3em}
    \item Det åligger FnollK:s ordförande:
	\begin{description}
		\item[att] tillse att kommittéens åligganden utförs.
		\item[att] leda och ansvara för FnollK:s arbete.
		\item[att] fungera som kontaktlänk mellan FnollK och övriga kommittéer samt företräda FnollK i sektionsstyrelsen.
		\item[att] tillsammans med FnollK:s kassör ansvara för FnollK:s ekonomi.
		\item[att] representera Fysikteknologsektionen i Chalmers Studentkårs samarbetsorgan för mottagningen, MoS.
	\end{description}
	
	\item Det åligger FnollK:s vice ordförande:
			\begin{description}
				\item[att] vara kommitténs suppleant i sektionsstyrelsen
				\item[att] vid ordförandens frånvaro överta dennes åligganden
			\end{description}

		\item Det åligger FnollK:s kassör:
			\begin{description}
				\item[att] sköta och tillsammans med FnollK:s ordförande ansvara för FnollK:s ekonomi.
				\item[att] kontinuerligt föra en granskningsbar redovisning gällande FnollK:s ekonomi.
			\end{description}

		\item Det åligger FnollK:s övriga ledamöter:
			\begin{description}
				\item[att] hjälpa FnollK-ordföranden i dennes uppgifter så att de utförs på bästa sätt.
			\end{description}
    
    \setcounter{section}{5}
    \setcounter{subsection}{4}
    
    \switchcolumn 
    
    \item \emph{FnollK är sektionens mottagningskommitté.}

    \item FnollK består av följande poster:
    \begin{itemize}
        \item Ordförande
    	\item Vice ordförande
    	\item Kassör
    	\item \emph{4 ledamöter}
    \end{itemize}
    \emph{varav} ordförande, vice ordförande och kassör är förtroendeposter.
    
    \item[] \emph{(ersätts av \S \ref{R:BrutnaVerksamhetsar})}\vspace{1.2em}
    
    \item[] \emph{(redundant)}\vspace{2.4em}
    
    \item Det åligger FnollK:\vspace{-0.4em}
    \begin{aligganden}
        \item genomföra en värdig mottagning i enlighet med kårens och sektionens \emph{anda i samråd med relevanta organ.}
        \item arrangera aktiviteter för \emph{Nollan} som syftar till att införliva \emph{dem} i livet som \emph{teknolog vid sektionens program} både vad gäller studier och det studiesociala livet.\vspace{0.2em}
        \item[] \emph{(redundant)}\vspace{2.6em}
        \item tillse att valen Åke bär nollbricka om gamble så tycker.
        \item \emph{tillse att Schrödingers katt bär nollbricka.}
        \item \emph{tillse att Schrödingers katt ej bär nollbricka.}
    \end{aligganden}
    
   
    \item Det åligger FnollK:s ordförande: \vspace{-0.2em}
    \begin{aligganden}
        \item[] \emph{(redundant)}
        \item[] \emph{(ersätts av \S \ref{R:KommitteOrdfLeda})}
        \item[] \emph{(ersätts av \S \ref{R:KommitteOrdfRepr})}\vspace{3em}
        \item[] \emph{(ersätts av stadga \S \ref{S:KomitteEkonomisktAnsvar})}\vspace{1.2em}
        \item representera \emph{sektionen} i Chalmers studentkårs samarbetsorgan för mottagningen, MoS.
    \end{aligganden}
    
    \vspace{2.4em}
    \begin{aligganden}
        \item[] \emph{(ersätts av \S \ref{R:StyretViceSuppleant})} \vspace{6.4em}
        \item[] \emph{(ersätts av stadga \S \ref{S:KomitteEkonomisktAnsvar})}\vspace{1.6em}
        \item[] \emph{(ersätts av stadga \S \ref{S:RedovisningKont})}\vspace{1.6em}
        \item[] \emph{(redundant)}
    \end{aligganden}
\end{lydelse}

\subsection{F6}
Syftet är redaktionellt samt att eliminera redundans.
Många implicita åligganden stryks.

Härigenom föreskrivs i fråga om reglementets avsnitt 6.6

\begin{dels}
    \item att avsnittets rubrik ska lyda ''F6'',
    \item att avsnittet ska omnumreras 5.5,
    \item att avsnittet i sin helhet ska ha följande lydelse.
\end{dels}

\begin{lydelse}

    \item[] \emph{(Se \S6.2.1)}
    
    \setcounter{section}{6}
    \setcounter{subsection}{6}
    
    \item F6 består av följande poster:
	\begin{itemize}
		\item Ordförande, Sexmästare
		\item Vice ordförande, Sexreterare
		\item Kassör
		\item 0--6 ledamöter
	\end{itemize}
    där ordförande, vice ordförande och kassör är förtroendeposter.

    \item F6 har en verksamhet som går från 1 juli till 30 juni.

	\item Verksamheten skall drivas i icke vinstdrivande syfte. Vid räkenskapsårets slut skall kommitténs tillgångar upp till 0,659 basbelopp övergå till nästkommande års kommitté. Tillgångar därutöver tillfaller sektionen. Visst utrymme för representation får förekomma.

    \item Det åligger F6:
	\begin{description}
		\item[att] minst en gång per läsperiod anordna gasque.
		\item[att] ansvara för kalas- och tentamensfestlighetsverksamhet på sektionen.
		\item[att] vara ett komplement till FnollK under mottagningen.
	\end{description}
	
	\item Det åligger F6:s ordförande, Sexmästaren:
	\begin{description}
		\item[att] tillse att kommitténs åligganden utförs.
		\item[att] leda F6:s arbete.
		\item[att] fungera som kontaktlänk mellan F6 och övriga kommittéer samt företräda F6 i sektionsstyrelsen.
		\item[att] att vara Fysiktteknologsektionens representant i Gasquerådet, om F6 beslutar att vara medlemmar i Gasquerådet.
		\item[att] tillsammans med kassören ansvara för F6:s ekonomi.
	\end{description}
	
	\item Det åligger F6:s vice ordförande, Sexreteraren:
	\begin{description}
		\item[att] vid sexmästarens frånvaro utföra dennes uppgifter.
		\item[att] vara kommitténs suppleant i sektionsstyrelsen
	\end{description}

    \item Det åligger F6:s kassör:
	\begin{description}
		\item[att] tillsammans med Sexmästaren ansvara för F6:s ekonomi.
		\item[att] kontinuerligt föra en granskningsbar redovisning gällande F6:s ekonomi.
	\end{description}

    \item Det åligger F6:s övriga medlemmar:
	\begin{description}
		\item[att] hjälpa de förtroendevalda i F6:s verksamhet.
	\end{description}

    \setcounter{section}{5}
    \setcounter{subsection}{5}

    \switchcolumn

    \item \emph{F6 är sektionens sexmästeri.}

    \item F6 består av följande poster:
    \begin{itemize}
    	\item Ordförande, Sexmästare
    	\item Vice ordförande, Sexreterare
    	\item Kassör
    	\item \emph{6 ledamöter}
    \end{itemize}
    \emph{varav} ordförande, vice ordförande och kassör är förtroendeposter.
    
    \item[] \emph{(ersätts av \S \ref{R:VanligtVerksamhetsar})}\vspace{1.2em}
    
    \item[] \emph{(redundant)}\vspace{5.8em}
    
    \item Det åligger F6:
    \begin{aligganden}
        \item minst en gång per läsperiod anordna gasque.
        \item ansvara för kalas- och tentamensfestlighetsverksamhet på sektionen.
        \item vara ett komplement till FnollK under mottagningen.
    \end{aligganden}
    
    \item Det åligger F6:s ordförande, Sexmästaren:
    \begin{aligganden}
        \item[] \emph{(redundant)}
        \item[] \emph{(ersätts av \S \ref{R:KommitteOrdfLeda})}
        \item[] \emph{(ersätts av \S \ref{R:KommitteOrdfRepr})}\vspace{3em}
        \item vara \emph{sektionens} representant i Gasquerådet \emph{om F6 så beslutar.}\vspace{1.4em}
        \item[] \emph{(ersätts av stadga \S \ref{S:KomitteEkonomisktAnsvar})}
    \end{aligganden}
    
    \vspace{2.6em}
    \begin{aligganden}
        \item[] \emph{(ersätts av \S \ref{R:StyretViceSuppleant})} \vspace{6.4em}
        \item[] \emph{(ersätts av stadga \S \ref{S:KomitteEkonomisktAnsvar})}\vspace{1.6em}
        \item[] \emph{(ersätts av stadga \S \ref{S:RedovisningKont})}\vspace{1.6em}
        \item[] \emph{(redundant)}
    \end{aligganden}
    
\end{lydelse}

\subsection{Djungelpatrullen}
Syftet är redaktionellt samt att eliminera redundans.
Många implicita åligganden stryks.

Härigenom föreskrivs i fråga om reglementets avsnitt 6.7

\begin{dels}
    \item att avsnittets rubrik ska lyda ''Djungelpatrullen'',
    \item att avsnittet ska omnumreras 5.6,
    \item att avsnittet i sin helhet ska ha följande lydelse.
\end{dels}

\begin{lydelse}
    \item[] \emph{(Se \S6.2.1)}
    
    \setcounter{section}{6}
    \setcounter{subsection}{7}
    
    \vspace{1.2em}
    \item Djungelpatrullens poster utgörs av:
	\begin{itemize}
		\item Ordförande, Överste
		\item Vice ordförande, Rustmästare
		\item Kassör, Skattmästare
		\item 0--7 adjutanter
	\end{itemize}
    där Ordförande, Vice ordförande och kassör är förtroendeposter

    \item Djungelpatrullen har en verksamhet som går från 1 juli till 30 juni.

    \item Verksamheten skall drivas i icke vinstdrivande syfte. Vid räkenskapsårets slut skall kommitténs tillgångar upp till 0,659 basbelopp övergå till nästkommande års kommitté. Tillgångar därutöver tillfaller sektionen. Visst utrymme för representation får förekomma.

    \item Det åligger Djungelpatrullen:
	\begin{description}
		\item[att] tillse att sektionshelgonet vördas på ett hedersamt sätt av alla Fysikteknologsektionens medlemmar.   
		\item[att] ansvara för att det ordnas arrangemang för medlemmarna på Fysikteknologsektionen. Dessa skall hållas i en anda som ökar sammanhållningen på Fysikteknologsektionen och ökar kontakten över årskursgränserna.
		\item[att] vara ett komplement till FnollK under mottagningen. 
		\item[att] sköta det löpande underhållet av Fysikteknologsektionens lokaler och egendom.
		\item[att] vårda sektionens traditioner.
		\item[] (se §1.1.1) 
	\end{description}
	
	\vspace{1.15em}
	\item Det åligger Djungelpatrullens ordförande, \\ Översten:
	\begin{description}
		\item[att] tillse att kommitténs åligganden utförs. 
		\item[att] leda Djungelpatrullens arbete.
		\item[att] fungera som kontaktlänk mellan Djungelpatrullen och övriga kommittéer samt företräda Djungelpatrullen i sektionsstyrelsen.
		\item[att] tillsammans med Skattmästaren ansvara för Djungelpatrullens ekonomi.
	\end{description}
	
	\item Det åligger Djungelpatrullens vice ordförande, \\ Rustmästaren:
	\begin{description}
		\item[att] i överstens frånvaro leda Djungelpatrullens arbete.
		\item[att] leda renoverings- och underhållsarbeten i sektionens lokaler.
		\item[att] ansvara för uthyrning av Focus inventarier.
		\item[att] vara kommitténs suppleant i sektionsstyrelsen.
	\end{description}
	
	\item Det åligger Djungelpatrullens kassör, Skattmästaren:
	\begin{description}
		\item[att] tillsammans med ordförande ansvara för och sköta Djungelpatrullens ekonomi.
		\item[att] kontinuerligt föra en granskningsbar redovisning gällande Djungelpatrullens ekonomi.
	\end{description}

    \item Det åligger Djungelpatrullens adjutanter:
	\begin{description}
		\item[att] hjälpa de förtroendevalda i DP:s verksamhet.
	\end{description}
	
	\setcounter{section}{5}
    \setcounter{subsection}{6}
    
    \switchcolumn
    
    \item \emph{Djungelpatrullen är sektionens rustmästeri och PR-förening.}

    \item Djungelpatrullen \emph{består av följande poster}:
    \begin{itemize}
        \item Ordförande, Överste
    	\item Vice ordförande, Rustmästare
    	\item Kassör, Skattmästare
    	\item \emph{7 adjutanter}
    \end{itemize}
    \emph{varav} ordförande, vice ordförande och kassör är förtroendeposter.
    
    \item[] \emph{(ersätts av \S \ref{R:VanligtVerksamhetsar})}\vspace{1.2em}
    
    \item[] \emph{(redundant)}\vspace{6em}
    
    \item Det åligger Djungelpatrullen:\vspace{-0.2em}
    \begin{aligganden}
        \item tillse att sektionshelgonet vördas på ett hedersamt sätt av alla \emph{sektionens} medlemmar.\vspace{1.4em}
        \item ansvara för att det ordnas arrangemang för medlemmarna på \emph{sektionen}.
            Dessa ska hållas i en anda som ökar sammanhållningen på \emph{sektionen} och ökar kontakten över årskursgränserna.\vspace{0.2em}
        \item vara ett komplement till FnollK under mottagningen. \vspace{0.2em}
        \item sköta det löpande underhållet av \emph{sektionens} lokaler och egendom.\vspace{0.2em}
        \item vårda sektionens traditioner.
        \item tillse att sektionens förstaårsstudenter städar sektionslokalen.
        \label{R:Nollanstad}
    \end{aligganden}
    
    \vspace{1.2em}
    \begin{aligganden}
        \item[] \emph{(redundant)}\vspace{0.4em}
        \item[]
        \item[] \emph{(ersätts av \S \ref{R:KommitteOrdfLeda})}
        \item[] \emph{(ersätts av \S \ref{R:KommitteOrdfRepr})}\vspace{0.7em}
        \item[] \emph{(ersätts av stadga \S \ref{S:KomitteEkonomisktAnsvar})}
    \end{aligganden}
    
    \vspace{1.15em}
    \item Det åligger Djungelpatrullens vice ordförande, \\ Rustmästaren:\vspace{-0.2em}
    \begin{aligganden}
        \item[] \emph{(redundant)} \vspace{1.2em}
        \item leda renoverings- och underhållsarbeten i sektionens lokaler.\vspace{0.2em}
        \item ansvara för uthyrning av Focus inventarier.\vspace{0.2em}
        \item[] \emph{(ersätts av \S \ref{R:StyretViceSuppleant})}
    \end{aligganden}
    
    \vspace{3.8em}
    \begin{aligganden}
        \item[] \emph{(ersätts av stadga \S \ref{S:KomitteEkonomisktAnsvar})}\vspace{1.6em}
        \item[] \emph{(ersätts av stadga \S \ref{S:RedovisningKont})}\vspace{1.6em}
        \item[] \emph{(redundant)}
    \end{aligganden}
    
    
\end{lydelse}

\subsection{Focumateriet}
Syftet är redaktionellt samt att eliminera redundans.
Många implicita åligganden stryks.

Härigenom föreskrivs i fråga om reglementets avsnitt 6.8

\begin{dels}
    \item att avsnittets rubrik ska lyda ''Djungelpatrullen'',
    \item att avsnittet ska omnumreras 5.7,
    \item att avsnittet i sin helhet ska ha följande lydelse.
\end{dels}

\begin{lydelse}

    \item[] \emph{(Se \S6.2.1)}
    
    \setcounter{section}{6}
    \setcounter{subsection}{8}
    
    \item Focumateriets utgörs av följande poster:
	\begin{itemize}
		\item Ordförande, kapten
		\item Vice ordförande, automatpirat
		\item Kassör, kistväktare
		\item 0--5 övriga ledamöter
	\end{itemize}
    där ordförande, vice ordförande och kassör är förtroendeposter.
    
    \item Focumateriet har en verksamhet som går från 1 juli till 30 juni.

	\item Verksamheten skall drivas i icke vinstdrivande syfte. Vid räkenskapsårets slut skall kommitténs tillgångar upp till 0,659 basbelopp övergå till nästkommande års kommitté. Tillgångar därutöver tillfaller sektionen. Visst utrymme för representation får förekomma.
    
    \item Det åligger Focumateriet:
	\begin{description}
		\item[att] handha Focumaten, samt, efter sektionsstyrelsens bestämmande, av sektionen ägda automater samt av sektionen ägd elektronisk utrustning.
	\end{description}
	
	\item Det åligger Focumateriets ordförande:
	\begin{description}
		\item[att] leda Focumateriets arbete.
		\item[att] tillse att kommitténs åligganden efterföljs.
		\item[att] tillsammans med kassören ansvara för kommitténs ekonomi.
	\end{description}

    \item Det åligger Focumateriets vice ordförande:
	\begin{description}
		\item[att] vara kommitténs suppleant i sektionsstyrelsen
		\item[att] vid ordförandens frånvaro överta dennes åligganden
	\end{description}

    \item Det åligger Focumateriets kassör:
	\begin{description}
		\item[att] tillsammans med ordföranden ansvara för och sköta Focumateriets ekonomi.
		\item[att] kontinuerligt föra en granskningsbar redovisning gällande Focumateriets ekonomi.
	\end{description}

    \item Det åligger Focumateriets övriga ledamöter:
	\begin{description}
		\item[att] hjälpa Focumateriordföranden att efterfölja Focumateriets åligganden.
	\end{description}
    
    \setcounter{section}{5}
    \setcounter{subsection}{7}
    
    \switchcolumn
    
    \item \emph{Focumateriet är sektionens focumateri.}

    \item Focumateriet består av följande poster:
    \begin{itemize}
        \item Ordförande, Kapten
    	\item Vice ordförande, Automatpirat
    	\item Kassör, Kistväktare
    	\item \emph{5 ledamöter}
    \end{itemize}
    \emph{varav} ordförande, vice ordförande och kassör är förtroendeposter.
    
    \item[] \emph{(ersätts av \S \ref{R:VanligtVerksamhetsar})}\vspace{1.2em}
    
    \item[] \emph{(redundant)}\vspace{6em}
    

    \item Det åligger Focumateriet:
    \begin{aligganden}
        \item \emph{handha Focumaten.}
        \item \emph{efter sektionsstyrelsens bestämmande handha sektionens automater och annan elektronisk utrustning.}
    \end{aligganden}
    
    \vspace{1em}
    \begin{aligganden}
        \item[] \emph{(redundant)}\vspace{0.2em}
        \item[] \emph{(ersätts av \S \ref{R:KommitteOrdfLeda})}\vspace{0.2em}
        \item[] \emph{(ersätts av stadga \S \ref{S:KomitteEkonomisktAnsvar})}
    \end{aligganden}
    
    \vspace{2.4em}
    \begin{aligganden}
        \item[] \emph{(ersätts av \S \ref{R:StyretViceSuppleant})}\vspace{1.4em}
        \item[] \emph{(redundant)}
        
    \end{aligganden}
    
    \vspace{2.8em}
    \begin{aligganden}
        \item[] \emph{(ersätts av stadga \S \ref{S:KomitteEkonomisktAnsvar})}\vspace{1.6em}
        \item[] \emph{(ersätts av stadga \S \ref{S:RedovisningKont})}\vspace{1.6em}
        \item[] \emph{(redundant)}
    \end{aligganden}
    
\end{lydelse}


\section{Sektionsföreningar}
Syftet är redaktionellt.

Härigenom föreskrivs i fråga om reglementets kapitel 7

\begin{dels}
    \item att kapitlets rubrik ska lyda ''Sektionsföreningar'',
    \item att kapitlet ska omnumreras 6,
    \item att avsnitt 7.1 ska utgå.
\end{dels}

\begin{lydelse}
    \setcounter{section}{7}
    \setcounter{subsection}{1}
    
    \item En person kan ej inneha två ledamotsposter i samma sektionsförening.
	\item Ekonomisk ansvarig ska vara myndig.
	
	\setcounter{section}{6}
    \setcounter{subsection}{0}
    \switchcolumn
    
    \item[] \emph{(ersätts av stadga \S \ref{S:ValbarhetMaxEnFortroende})}\vspace{1.2em}
    \item[] \emph{(ersätts av stadga \S \ref{S:ValbarEkonomiskMyndig})}
    
\end{lydelse}

\subsection{Förteckning}
Syftet är redaktionellt.

Härigenom föreskrivs i fråga om reglementets avsnitt 7.2

\begin{dels}
    \item att avsnittets rubrik ska lyda ''Förteckning'',
    \item att avsnittet ska omnumreras 6.1,
    \item att avsnittet i sin helhet ska ha följande lydelse.
\end{dels}

\begin{lydelse}
    
    \setcounter{section}{7}
    \setcounter{subsection}{2}
    
    \item Fysikteknologsektionens Sektionsföreningar är:
	\begin{itemize}
		\item Fabiola
		\item Fysikteknologsektionens Idrottsförening, FIF
		\item Foton
		\item Game Boy.
	\end{itemize}
	
	\setcounter{section}{6}
    \setcounter{subsection}{1}
    
    \switchcolumn
    
    \item \emph{Sektionens} sektionsföreningar är:
    \begin{itemize}
        \item Fabiola
    	\item FIF
    	\item Foton
    	\item Game Boy
    \end{itemize}
    
\end{lydelse}
\subsection{Allmänna åligganden}
Syftet är redaktionellt.
Lägg till krav på ordförande likt för kommittéer.

Härigenom föreskrivs i fråga om reglementets avsnitt 7.3

\begin{dels}
    \item att avsnittets rubrik ska lyda ''Allmänna åligganden'',
    \item att avsnittet ska omnumreras 6.2,
    \item att avsnittet i sin helhet ska ha följande lydelse.
\end{dels}

\begin{lydelse}
    
    \setcounter{section}{7}
    \setcounter{subsection}{3}
    
    \item Det åligger varje sektionsförening
	\begin{description}
		\item[att] följa åligganden enligt arbetsbeskrivning.
	\end{description}

    \item[] ---
    
    \vspace{5em}
    \item Det åligger ekonomiskt ansvarig:
	\begin{description}
	    \vspace{1.5em}
		\item[att] mot revisorerna och sektionskassören kontinuerligt redovisa för den ekonomiska situationen.
	\end{description}
    
    \setcounter{section}{6}
    \setcounter{subsection}{2}
    
    \switchcolumn
    
    \item[]
    \item[] \emph{(implicit från \S \ref{R:ForteckningStyrdokument})}
    
  
    \item \emph{Det åligger ordförande i sektionsförening:}
    \label{R:SektForeningOrdf}
    \begin{aligganden}
        \item \emph{leda sektionsföreningens arbete.}
        \item \emph{fungera som en kontaktlänk mellan sektionsföreningen och andra organ.}
    \end{aligganden}
    
    \item Det åligger ekonomiskt ansvarig \emph{i sektionsförening}:
    \label{R:SektForeningEkonomi}
    \begin{aligganden}
        \item mot revisorerna och sektionskassören kontinuerligt redovisa för den ekonomiska situationen.
    \end{aligganden}
    
\end{lydelse}
\subsection{Fabiola}
Syftet är enbart redaktionellt.

Härigenom föreskrivs i fråga om reglementets avsnitt 7.4

\begin{dels}
    \item att avsnittets rubrik ska lyda ''Fabiola'',
    \item att avsnittet ska omnumreras 6.3,
    \item att avsnittet i sin helhet ska ha följande lydelse.
\end{dels}

\begin{lydelse}

    \setcounter{section}{7}
    \setcounter{subsection}{4}

    \item[] ---\vspace{1.2em}
    
    \item Fabiola består av 1--10 ledamöter varav en väljs internt till ordförande. Fabiola innehar inga förtroendeposter.
    
	\item Det åligger Fabiola:
	\begin{description}
		\item[att] arrangera evenemang riktade mot kvinnor och icke-binära på sektionen.
	\end{description}   %ändrad enl sektmöte lp4 20/21
    
    \setcounter{section}{6}
    \setcounter{subsection}{3}
    
    \switchcolumn
    
    \item \emph{Fabiola är sektionens förening för kvinnor och ickebinära.}

    \item Fabiola består av \emph{10 ledamöter} varav en väljs internt till ordförande. Fabiola har inga förtroendeposter.
    		
    \item Det åligger Fabiola:
    \begin{aligganden}
        \item arrangera evenemang riktade mot kvinnor och \emph{ickebinära} på sektionen.
    \end{aligganden}   %ändrad enl sektmöte lp4 20/21
    
\end{lydelse}
\subsection{FIF}
Syftet är mestadels redaktionellt.
Bestämmelsen om ekonomiskt ansvar förtydligas.

Härigenom föreskrivs i fråga om reglementets avsnitt 7.5

\begin{dels}
    \item att avsnittets rubrik ska lyda ''FIF'',
    \item att avsnittet ska omnumreras 6.4,
    \item att avsnittet i sin helhet ska ha följande lydelse.
\end{dels}

\begin{lydelse}

    \setcounter{section}{7}
    \setcounter{subsection}{5}
    
    
    \item[] ---
    
    \item FIF består av följande poster:
    \begin{itemize}
    	\item Ordförande
    	\item Vice ordförande
    	\item Kassör
    	\item 0--6 ledamöter.
    \end{itemize}
    Där ordförande, vice ordförande och kassör är förtroendeposter.
    
    \vspace{1.2em}
    \item FIF har en verksamhet som går från 1 januari till 31 december.

	\item Verksamheten skall drivas i icke vinstdrivande syfte. Vid räkenskapsårets slut skall föreningens tillgångar upp till 0,659 basbelopp övergå till nästkommande års förening. Tillgångar därutöver tillfaller sektionen. Visst utrymme för representation får förekomma.
    
    \item Det åligger FIF:
	\begin{description}
		\item[att] främja idrottskulturen på sektionen genom att arrangera regelbundna träningar.
		\item[att] handha idrottsmaterial samt sköta utlåning av denna till sektionsmedlemmar.
	\end{description}

	\item Det åligger FIF:s ordförande:
	\begin{description}
  		\item[att] tillse att FIF:s åligganden utförs.
  		\item[att] leda FIF:s arbete.
  		\item[att] fungera som kontaktlänk mellan FIF och andra föreningar på sektionen samt tillse att kontakt med andra sektioners idrottsföreningar uppehålls.
  		\item[att] tillsammans med kassören ansvara för FIF:s ekonomi.
  		\item[att] tillsammans med kassören presentera ett bokslut vid det första sektionsmötet efter verksamhetsårets slut.
	\end{description}
	
	\setcounter{enumi}{6}
	
	\item Det åligger FIF:s kassör:
	\begin{description}
		\item[att] tillsammans med ordföranden ansvara för FIF:s ekonomi.
		\item[att] kontinuerligt föra en granskningsbar redovisning gällande FIF:s ekonomi.
		\item[att] tillsammans med ordföranden presentera ett bokslut vid det första sektionsmötet efter verksamhetsårets slut.
	\end{description}
    
    \setcounter{section}{6}
    \setcounter{subsection}{4}
    
    \switchcolumn
    
    \item \emph{FIF är sektionens idrottsförening.}

    \item FIF består av följande poster:
    \begin{itemize}
        \item Ordförande
    	\item Vice ordförande
    	\item Kassör
    	\item \emph{6 ledamöter}
    \end{itemize}
    \emph{varav} ordförande, vice ordförande och kassör är förtroendeposter.
    \emph{Ordförande och kassör är ekonomiskt ansvariga.} \label{R:FIFekansvar}
    
    \item[] \emph{(ersätts av \S \ref{R:BrutnaVerksamhetsar})}\vspace{1.2em}
    
    \item[] \emph{(redundant)}\vspace{6em}
    
    \item Det åligger FIF:
    \begin{aligganden}
        \item främja idrottskulturen på sektionen genom att arrangera regelbundna träningar.
        \item handha idrottsmaterial samt sköta utlåning av denna till sektionsmedlemmar.
    \end{aligganden}
    
    \item Det åligger FIF:s ordförande:\vspace{-0.4em}
    \begin{aligganden}
        \item[] \emph{(ersätts av \S \ref{R:SektForeningOrdf})}
        \item[] \emph{(ersätts av \S \ref{R:SektForeningOrdf})}\vspace{0.4em}    
        \item tillse att kontakt \emph{med andra sektioners idrottsföreningar} uppehålls.\vspace{2.6em}
        \item[] \emph{(ersätts av \S \ref{R:FIFekansvar})}%Note: Tror inte det gäller för sektionsföreningar per default
        \vspace{1em}
        \item[] \emph{(ersätts av \S \ref{R:FIFred})}%Tror inte det gäller för sektionsföreningar per default
    \end{aligganden}
    
    \vspace{2.2em}
    \item Det åligger FIF:s kassör:
    \begin{aligganden}
        \item[] \emph{(ersätts av \S \ref{R:FIFekansvar})}\vspace{1.2em}
        \item[] \emph{(ersätts av \S \ref{R:SektForeningEkonomi})}\vspace{1.4em}
        \item[] \emph{(ersätts av \S \ref{R:SektForeningEkonomi})}
    \end{aligganden}
    
\end{lydelse}
\subsection{Foton}
Syftet är enbart redaktionellt.

Härigenom föreskrivs i fråga om reglementets avsnitt 7.6

\begin{dels}
    \item att avsnittets rubrik ska lyda ''Foton'',
    \item att avsnittet ska omnumreras 6.5,
    \item att avsnittet i sin helhet ska ha följande lydelse.
\end{dels}

\begin{lydelse}
    
    \setcounter{section}{7}
    \setcounter{subsection}{6}
    
    \item[] ---
    
    \item Foton består av 1--6 ledamöter varav en väljs internt till ordförande. Foton innehar inga förtroendeposter.
	
	\item Det åligger Foton:
	\begin{description}
		\item[att] löpande dokumentera sektionens arrangemang.
		\item[att] i samband med arbetsmarknadsmässor arrangera porträttfotografering.
		\item[att] stödja sektionens kommittéer och nämnder vid behov av filmning eller fotografering.
	\end{description}
		
    \setcounter{section}{6}
    \setcounter{subsection}{5}
    
    \switchcolumn
    
    \item \emph{Foton är sektionens fotoförening.}

    \item Foton består av \emph{6 ledamöter} varav en väljs internt till ordförande. Foton har inga förtroendeposter.
    
    \vspace{1.2em}
    \item Det åligger Foton:
    \begin{aligganden}
        \vspace{-0.2em}
        \item löpande dokumentera sektionens arrangemang.\vspace{0.2em}
        \item i samband med arbetsmarknadsmässor arrangera porträttfotografering.\vspace{0.2em}
        \item stödja sektionens kommittéer och nämnder vid behov av filmning eller fotografering.
    \end{aligganden}
    
\end{lydelse}
\subsection{Gameboy}
Syftet är enbart redaktionellt.

Härigenom föreskrivs i fråga om reglementets avsnitt 7.7

\begin{dels}
    \item att avsnittets rubrik ska lyda ''Gameboy'',
    \item att avsnittet ska omnumreras 6.6,
    \item att avsnittet i sin helhet ska ha följande lydelse.
\end{dels}

\begin{lydelse}
    \setcounter{section}{7}
    \setcounter{subsection}{7}
    
    \item[] ---
    
    \item Game Boy består av 0--6 Game Boys. Game Boy har inga förtroendeposter.
    
    \item Det åligger Game Boy:
    \begin{description}
        \item[att] ta hand om de sällskapsspel som finns på Focus.
        \item[att] främja brädspelsverksamheten på Fysikteknologsektionen.
    \end{description}
		
    \setcounter{section}{6}
    \setcounter{subsection}{6}
    \switchcolumn
    
    \item \emph{Game Boy är sektionens brädspelsförening.}

    \item Game Boy består av \emph{6 Game Boys}. Game Boy har inga förtroendeposter.
    	    
    \item Det åligger Game Boy:
    \begin{aligganden}
        \vspace{-0.4em}
        \item ta hand om de sällskapsspel som finns på Focus.\vspace{0.2em}
    	\item främja brädspelsverksamheten på \emph{sektionen}.
    \end{aligganden}
    
\end{lydelse}

\section{Funktionärer}
Syftet är mestadels redaktionellt. Några mindre ändringar för individuella funktionärer förekommer.

Härigenom föreskrivs i fråga om reglementets kapitel 8

\begin{dels}
    \item att kapitlets rubrik ska lyda ''Funktionärer'',
    \item att kapitlet ska omnumreras 7,
    \item att avnitt 8.2 ska utgå.
\end{dels}


\subsection{Förteckning}
Syftet är enbart redaktionellt. De längre beskrivande titlarna flyttas till respektive förenings avsnitt.

Härigenom föreskrivs i fråga om reglementets avsnitt 8.1

\begin{dels}
    \item att avsnittets rubrik ska lyda ''Förteckning'',
    \item att avsnittet ska omnumreras 7.1,
    \item att avsnittet i sin helhet ska ha följande lydelse.
\end{dels}

\begin{lydelse}
    \setcounter{section}{8}
    
    \item Fysikteknologsektionens sektionsfunktionärer är:
	\begin{itemize}
		\item Fysikteknologsektionens informationsskrift, Finform
		\item Sångförmännen
		\item Revisorer
		\item Fysikteknologsektionens skyddshelgon, Dragos
		\item Fanfareriet
		\item Bilnissar
		\item Blodgruppen
		\item Kräldjursvårdare
		\item Dumvästinnehavare
		\item Bakisclubben (BC)
		\item Fysikteknologsektionens webbgrupp, Spidera
		\item Sektionsnörd
		\item Balnågonting
		\item Piff och Puff
		\item JämF
		\item Mastermottagningsansvarig
		\item Frisörer.
	\end{itemize}
    
    \setcounter{section}{7}
    \switchcolumn
    
    \item \emph{Sektionens} funktionärer är:
    \begin{itemize}
        \item Finform\vspace{1.2em}
    	\item Sångförmännen
    	\item Revisorer
    	\item \emph{Dragos}\vspace{1.2em}
    	\item Fanfareriet
    	\item Bilnissar
    	\item Blodgruppen
    	\item Kräldjursvårdare
    	\item Dumvästinnehavare
    	\item \emph{Bakisclubben}
    	\item \emph{Spidera} \vspace{1.2em}
    	\item Sektionsnörd
    	\item Balnågonting
    	\item Piff och Puff
    	\item JämF
    	\item Mastermottagningsansvarig
    	\item Frisörer.
    \end{itemize}
    
\end{lydelse}

\subsection{Finform}
Syftet är redaktionellt.

Härigenom föreskrivs i fråga om reglementets avsnitt 8.3

\begin{dels}
    \item att avsnittets rubrik ska lyda ''Finform'',
    \item att avsnittet ska omnumreras 7.2,
    \item att avsnittet i sin helhet ska ha följande lydelse.
\end{dels}

\begin{lydelse}
    \setcounter{section}{8}
    \setcounter{subsection}{3}
    
    
    \item Finform är Fysikteknologsektionens informationsskrift och skall på ett lättillgängligt sätt presentera intressanta fakta, skämt och skvaller. 
    
    \item Finforms redaktion består av chefredaktör tillika ansvarig utgivare, kassör samt 2--8 redaktörer.
    
    \vspace{7.1em}
    \item Ansvarig utgivare för Finform tillträder efter inregistrering enligt gällande lag.

    \item Det åligger Finform-redaktionen:
	\begin{description}
		\item[att] producera minst 4 nummer av Finform per läsår, med fördelningen 2 på hösten och 2 på våren.
		\item[att] ansvara för tryckning och distribution.
	\end{description}
    
    \vspace{2.5em}
    \item Det åligger Finforms chefredaktör:
	\begin{description}
		\item[att] vara ansvarig utgivare.
		\item[att] tillse att Finforms åliggande utförs.
		\item[att] leda Finforms arbete.
	\end{description}
	

	\item Det åligger Finforms kassör:
	\begin{description}
		\item[att] tillsammans med chefredaktören ansvara för Finforms ekonomi.
		\item[att] ansvara för att Finforms ekonomiska anslag används inom av sektionsstyrelsen fastställd ram.
	\end{description}

    \item Det åligger Finforms ansvariga utgivare:
	\begin{description}
		\item[att] kontrollera Finform så att Finform inte agerar olagligt, kränkande eller på annat sätt olämpligt.
		\item[att] Finform agerar på ett lämpligt sätt för att vara Fysikteknologsektionens officiella informationsskrift.
	\end{description}

    \setcounter{section}{6}
    \setcounter{subsection}{7}
    \switchcolumn
    
    \item Finform är sektionens informationsskrift och ska på ett lättillgängligt sätt presentera intressanta fakta, skämt och skvaller. 
    
    \item \emph{Finform består av följande poster:}
    \label{R:FinformPoster}
    \begin{itemize}
        \item \emph{Chefredaktör tillika ansvarig utgivare}
    	\item \emph{Kassör}
    	\item \emph{8 redaktörer}
    \end{itemize}
    
    \item Ansvarig utgivare för Finform tillträder efter inregistrering enligt gällande lag.
    
    \item Det åligger Finformredaktionen:
    \begin{aligganden}
        \vspace{-0.4em}
        \item producera minst 4 nummer av Finform per läsår, varav minst 2 på hösten och minst 2 på våren.\vspace{0.2em}
        \item ansvara för tryckning och distribution av Finform.
    \end{aligganden}
    
    \item Det åligger Finforms chefredaktör:
    \label{R:FinformChef}
    \begin{aligganden}
        \vspace{-0.2em}
        \item[] \emph{(ersätts av \S \ref{R:FinformPoster})}\vspace{0.2em}
        \item[] \emph{(redundant)}\vspace{0.2em}
        \item leda Finforms arbete.
        \item \emph{tillse att Finform inte agerar olagligt, kränkande eller på annat olämpligt sätt.}
        \item \emph{tillse att Finform agerar på ett lämpligt sätt för att vara sektionens officiella informationsskrift.}
    \end{aligganden}
    
    \item Det åligger Finforms kassör:
    \begin{aligganden}
        \vspace{-0.2em}
        \item \emph{tillsammans med chefredaktören} ansvara för att Finforms ekonomiska anslag används inom av sektionsstyrelsen fastställd ram.
    \end{aligganden}
    
    \begin{aligganden}
        \item[] \emph{(ersätts av \S \ref{R:FinformChef})}\vspace{2.6em}
        \item[] \emph{(ersätts av \S \ref{R:FinformChef})}
    \end{aligganden}
    
\end{lydelse}

\setcounter{section}{7}
\setcounter{subsection}{2}
\subsection{Sångförmän}
Syftet är enbart redaktionellt.

Härigenom föreskrivs i fråga om reglementets avsnitt 8.4

\begin{dels}
    \item att avsnittets rubrik ska lyda ''Sångförmän'',
    \item att avsnittet ska omnumreras 7.3,
    \item att avsnittet i sin helhet ska ha följande lydelse.
\end{dels}

\begin{lydelse}
    \setcounter{section}{8}
    \setcounter{subsection}{4}
    
    \item Sångförmännens syfte är att förvalta och bevara sektionens sångtraditioner.

	\item Sångförmännen är till antalet 0--6.
    
    \setcounter{section}{7}
    \setcounter{subsection}{3}
    \switchcolumn
    
    \item Sångförmännens syfte är att förvalta och bevara sektionens sångtraditioner.

    \item Sångförmännen är\emph{ 6 till antalet}.
    
\end{lydelse}

\subsection{Dragos}
Syftet är enbart redaktionellt.

Härigenom föreskrivs i fråga om reglementets avsnitt 8.5

\begin{dels}
    \item att avsnittets rubrik ska lyda ''Dragos'',
    \item att avsnittet ska omnumreras 7.4.
\end{dels}
\begin{lydelse}
    \setcounter{section}{8}
    \setcounter{subsection}{5}
    
    \item Dragos är sektionens högste beskyddare, och utövar Fanfareriets högsta befäl.
    
    \setcounter{section}{7}
    \setcounter{subsection}{3}
    \switchcolumn
    
    \item Dragos är sektionens högste beskyddare, och utövar Fanfareriets högsta befäl.
    
\end{lydelse}

\subsection{Fanfareriet}
Syftet är enbart redaktionellt.

Härigenom föreskrivs i fråga om reglementets avsnitt 8.6

\begin{dels}
    \item att avsnittets rubrik ska lyda ''Fanfareriet'',
    \item att avsnittet ska omnumreras 7.5,
    \item att avsnittet i sin helhet ska ha följande lydelse.
\end{dels}
\begin{lydelse}
    \setcounter{section}{8}
    \setcounter{subsection}{6}
    
    \item Fanfareriets syfte är att ta hand om sektionens fanor och flaggor.

	\item Fanfareriet består av en flaggmarskalk och 1-2 fanbärare.
    
    \setcounter{section}{7}
    \setcounter{subsection}{5}
    \switchcolumn
    
    \item Fanfareriets syfte är att ta hand om sektionens fanor och flaggor.

    \item Fanfareriet består av en flaggmarskalk och \emph{2 fanbärare}.
    
\end{lydelse}

\subsection{Bilnissar}
Syftet är enbart redaktionellt.

Härigenom föreskrivs i fråga om reglementets avsnitt 8.7

\begin{dels}
    \item att avsnittets rubrik ska lyda ''Bilnissar'',
    \item att avsnittet ska omnumreras 7.6.
\end{dels}
\begin{lydelse}
    \setcounter{section}{8}
    \setcounter{subsection}{7}
    
    \item Bilnissarna består av en ekonomisk bilnisse samt en mekanisk bilnisse.

	\item Det åligger Bilnissarna:
	\begin{description}
		\item[att] ansvara för de motorfordon som sektionsstyrelsen beslutat om, dock endast sådana som sektionen helt eller delvis förfogar över.
		\item[att] ansvara för uthyrning av dessa fordon, enligt taxa fastställd av sektionsstyrelsen.
	\end{description}

	\item Det åligger ekonomisk bilnisse:
	\begin{description}
		\item[att] vara sektionskassören behjälplig vid ekonomiska ärenden rörande bilnisse.
	\end{description}

	\item Det åligger mekanisk bilnisse:
	\begin{description}
		\item[att] tillse att ovannämnda fordon underhålls och repareras på ett tillfredsställande sätt.
	\end{description}
    
    \setcounter{section}{7}
    \setcounter{subsection}{6}
    \switchcolumn
    
    \item Bilnissarna består av en ekonomisk bilnisse samt en mekanisk bilnisse.

    \item Det åligger Bilnissarna:
    \begin{aligganden}
        \vspace{-0.4em}
        \item ansvara för de motorfordon som sektionsstyrelsen beslutat om, dock endast sådana som sektionen helt eller delvis förfogar över.\vspace{0.4em}
        \item ansvara för uthyrning av dessa fordon, enligt taxa fastställd av sektionsstyrelsen.
    \end{aligganden}
    
    \item Det åligger ekonomisk bilnisse:
    \begin{aligganden}
        \vspace{-0.4em}
        \item vara sektionskassören behjälplig vid ekonomiska ärenden rörande bilnisse.
    \end{aligganden}
    
    \item Det åligger mekanisk bilnisse:
    \begin{aligganden}
        \vspace{-0.4em}
        \item tillse att ovannämnda fordon underhålls och repareras på ett tillfredsställande sätt.
    \end{aligganden}
    
\end{lydelse}

\subsection{Blodgruppen}
Syftet är enbart redaktionellt.

Härigenom föreskrivs i fråga om reglementets avsnitt 8.8

\begin{dels}
    \item att avsnittets rubrik ska lyda ''Blodgruppen'',
    \item att avsnittet ska omnumreras 7.7,
    \item att avsnittet i sin helhet ska ha följande lydelse.
\end{dels}
\begin{lydelse}
    \setcounter{section}{8}
    \setcounter{subsection}{8}
    
    \item Blodgruppens syfte är att uppmuntra sektionsmedlemmarna till att lämna blod

	\item Blodgruppen består av en ansvarig och 1--4 ledamöter.
    
    \setcounter{section}{7}
    \setcounter{subsection}{7}
    \switchcolumn
    
    \item Blodgruppens syfte är att uppmuntra sektionsmedlemmarna till att lämna blod.

    \item Blodgruppen består av \emph{fem ledamöter varav en väljs internt till ordförande}.
    
\end{lydelse}

\subsection{Kräldjursvårdare}
Syftet är enbart redaktionellt.

Härigenom föreskrivs i fråga om reglementets avsnitt 8.9

\begin{dels}
    \item att avsnittets rubrik ska lyda ''Kräldjursvårdare'',
    \item att avsnittet ska omnumreras 7.8,
    \item att avsnittet i sin helhet ska ha följande lydelse.
\end{dels}
\begin{lydelse}
    \setcounter{section}{8}
    \setcounter{subsection}{9}
    
    \item Kräldjursvårdarens syfte är att ta hand om sektionens slang, Tilde.

	\item Det ska finnas en kräldjursvårdare på sektionen.

	\item Sektionen skall inte ha kräldjur på Focus.
    
    \setcounter{section}{7}
    \setcounter{subsection}{8}
    \switchcolumn
    
    \item Kräldjursvårdarens syfte är att ta hand om sektionens slang, Tilde.

    \item Det ska finnas en kräldjursvårdare på sektionen.
    
    \item Sektionen \emph{ska} inte ha kräldjur på Focus.
    
\end{lydelse}

\subsection{Dumvästinnehavare}
Syftet är enbart redaktionellt.

Härigenom föreskrivs i fråga om reglementets avsnitt 8.10

\begin{dels}
    \item att avsnittets rubrik ska lyda ''Dumvästinnehavare'',
    \item att avsnittet ska omnumreras 7.9.
\end{dels}
\begin{lydelse}
    \setcounter{section}{8}
    \setcounter{subsection}{10}
    
	\item Det ska väljas en Dumvästinnehavare på varje sektionsmöte.

	\item Den, som med avsikt att erhålla Dumvästen utfört dumheter bör ej vara kvalificerad till denna.
	
	\item Kriminella handlingar är ej kvalificerade till Dumvästen, såvida inte sektionsmötet anser detta.
	
	\item Om ej tillräckligt kvalificerad dumhet nomineras, kvarstår Dumvästen hos innehavaren för tillfället.
	
	\item Förteckning över Dumvästinnehavare genom tiderna tillhandahålls av sektionsstyrelsen.
    
    \setcounter{section}{7}
    \setcounter{subsection}{9}
    \switchcolumn
    
    \item Det ska väljas en Dumvästinnehavare på varje sektionsmöte.
    
    \item Den, som med avsikt att erhålla Dumvästen utfört dumheter bör ej vara kvalificerad till denna.
    		
    \item Kriminella handlingar är ej kvalificerade till Dumvästen, såvida inte sektionsmötet anser detta.
    		
    \item Om ej tillräckligt kvalificerad dumhet nomineras, kvarstår Dumvästen hos innehavaren för tillfället.
    		
    \item Förteckning över Dumvästinnehavare genom tiderna tillhandahålls av sektionsstyrelsen.
    
\end{lydelse}

\subsection{Bakisclubben}
Syftet är enbart redaktionellt.

Härigenom föreskrivs i fråga om reglementets avsnitt 8.11

\begin{dels}
    \item att avsnittets rubrik ska lyda ''Bakisclubben'',
    \item att avsnittet ska omnumreras 7.10,
    \item att avsnittet i sin helhet ska ha följande lydelse.
\end{dels}
\begin{lydelse}
    \setcounter{section}{8}
    \setcounter{subsection}{11}
    
    \item Bakisclubben (BC) ska främja bakverksamheten på sektionen.

	\item BC består av 0--6 bakisar.

	\item Det åligger Bakisclubben:
	\begin{description}
		\item[att] tillse att kanelbullar bakas inför samt att dessa kanelbullar är tillgängliga på Focus för alla sektionsmedlemmar att avnjuta i samband med Kanelbullens dag.
	\end{description}
    
    \setcounter{section}{7}
    \setcounter{subsection}{10}
    \switchcolumn
    
    \item \emph{Bakisclubbens} syfte är att främja bakverksamheten på sektionen.

    \item \emph{Bakisclubben} består av \emph{6 bakisar}.
    
    \item Det åligger Bakisclubben:
    \begin{aligganden}
        \item \emph{inför Kanelbullens dag tillse att kanelbullar bakas och görs tillgängliga på Focus för alla sektionsmedlemmar att avnjuta.}
    \end{aligganden}
    
\end{lydelse}

\subsection{Spidera}
Syftet är mestadels redaktionellt.
Åliggandet att nätmästaren är länken mellan Spidera och sektionsstyrelsen slopas då Informationsansvarig är mer naturlig.

Härigenom föreskrivs i fråga om reglementets avsnitt 8.12

\begin{dels}
    \item att avsnittets rubrik ska lyda ''Spidera'',
    \item att avsnittet ska omnumreras 7.11,
    \item att avsnittet i sin helhet ska ha följande lydelse.
\end{dels}
\begin{lydelse}
    \setcounter{section}{8}
    \setcounter{subsection}{12}
    
    \item[] ---
    
    \item Spidera består av en nätmästare, informationsansvarig från sektionsstyrelsen samt 1--9 nätmakare. Sektionsmötet väljer 2--10 teknologer till Spidera, som i samråd med sektionsstyrelsen utser en nätmästare.

	\item Det åligger Spidera:
	\begin{description}
		\item[att] administrera och utveckla Fysikteknologsektionens internetportal.
	\end{description}

	\item Det åligger Spideras nätmästare:
	\begin{description}
	    \item[] ---
		\item[att] tillse att ansvarig för av Spidera disponerad hårdvara är medlem i Spidera.
		\item[att] fungera som länk mellan sektionsstyrelsen och Spidera.
		\item[att] kalla till möte med Spidera.
	\end{description}
    
    \setcounter{section}{7}
    \setcounter{subsection}{11}
    \switchcolumn
    
    \item \emph{Spideras syfte är att ansvara för sektionens IT.}

    \item Spidera består av \emph{10 nätmakare samt informationsansvarig från sektionsstyrelsen.
    Av dessa väljs en internt till nätmästare i samråd med sektionsstyrelsen.}
    
    \vspace{1.2em}
    \item Det åligger Spidera:
    \begin{aligganden}
        \vspace{-0.4em}
        \item administrera och utveckla Fysikteknologsektionens internetportal.
    \end{aligganden}
    
    \item Det åligger Spideras nätmästare:
    \begin{aligganden}
        \vspace{-0.4em}
    	\item \emph{leda Spideras arbete.}\vspace{0.2em}
        \item tillse att ansvarig för av Spidera disponerad hårdvara är medlem i Spidera.\vspace{0.4em}
        \item[] \emph{(sköts av Informationsansvarig)}\vspace{1.4em}
        \item[] \emph{(redundant)}
    \end{aligganden}
    
\end{lydelse}

\subsection{Sektionsnörd}
Syftet är att eliminera redundans.
Funktionären har endast en post och därmed kan det bara väljas en till posten.
Sektionsmötet är därefter ålagd att välja in den.

Härigenom föreskrivs i fråga om reglementets avsnitt 8.13

\begin{dels}
    \item att avsnittets rubrik ska lyda ''Sektionsnörd'',
    \item att avsnittet ska omnumreras 7.12,
    \item att avsnittet i sin helhet ska ha följande lydelse.
\end{dels}
\begin{lydelse}
    \setcounter{section}{8}
    \setcounter{subsection}{13}
    
    \item Sektionsnördens syfte är att ta hand om allting som rör sektionens prenumeration av Fantomen

	\item Det ska väljas en sektionsnörd
    
    \setcounter{section}{7}
    \setcounter{subsection}{12}
    \switchcolumn
    
    \item Sektionsnördens syfte är att ta hand om allting som rör sektionens prenumeration av Fantomen.
    
    \item[] \emph{(redundant)}
    
\end{lydelse}

\subsection{Balnågonting}
Syftet är enbart redaktionellt.

Härigenom föreskrivs i fråga om reglementets avsnitt 8.14

\begin{dels}
    \item att avsnittets rubrik ska lyda ''Balnågonting'',
    \item att avsnittet ska omnumreras 7.13,
    \item att avsnittet i sin helhet ska ha följande lydelse.
\end{dels}
\begin{lydelse}
    \setcounter{section}{8}
    \setcounter{subsection}{14}
    
    \item Balnågontings syfte är att anordna bal, middag samt kringaktiviteter som tillhör balen

	\item Balnågonting har 0--5 medlemmar.
    
    \setcounter{section}{7}
    \setcounter{subsection}{13}
    \switchcolumn
    
    \item Balnågontings syfte är att anordna bal, middag samt kringaktiviteter som tillhör balen.

    \item Balnågonting har \emph{5 medlemmar}.
    
\end{lydelse}

\subsection{Piff och Puff}
Syftet är enbart redaktionellt.

Härigenom föreskrivs i fråga om reglementets avsnitt 8.15

\begin{dels}
    \item att avsnittets rubrik ska lyda ''Piff och Puff'',
    \item att avsnittet ska omnumreras 7.14,
    \item att avsnittet i sin helhet ska ha följande lydelse.
\end{dels}
\begin{lydelse}
    \setcounter{section}{8}
    \setcounter{subsection}{15}
    
    \item Piff och Puff är sektionens aktivitetsgrupp, och syftar till att hjälpa sektionen engagera och underhålla fler sektionsmedlemmar, samt hjälpa sektionsmedlemmar med vägledning och tips om hur aktiviteter för sektionsmedlemmar kan genomföras.

	\item Piff och Puff består av 0--4 Piffar.
    
    \setcounter{section}{7}
    \setcounter{subsection}{14}
    \switchcolumn
    
    \item Piff och Puff är sektionens aktivitetsgrupp, och syftar till att hjälpa sektionen engagera och underhålla fler sektionsmedlemmar, samt hjälpa sektionsmedlemmar med vägledning och tips om hur aktiviteter för sektionsmedlemmar kan genomföras.

    \item Piff och Puff består av \emph{4 piffar}.
    
\end{lydelse}

\subsection{JämF}
Syftet är enbart redaktionellt.

Härigenom föreskrivs i fråga om reglementets avsnitt 8.16

\begin{dels}
    \item att avsnittets rubrik ska lyda ''JämF'',
    \item att avsnittet ska omnumreras 7.15,
    \item att avsnittet i sin helhet ska ha följande lydelse.
\end{dels}
\begin{lydelse}
    \setcounter{section}{8}
    \setcounter{subsection}{16}
    
    \item JämF är sektionens jämlikhetsråd, och syftar till att arbeta för en mer jämlik sektion där alla känner sig välkomna.
	    
    \item JämF består av sektionsstyrelsens skyddsombud, en representant vardera från de kommittéer och sektionsföreningar där intresse för medlemskap finns, samt 4--6 fristående ledamöter varav en väljs internt till ordförande. %sektmöte 1 lp3 21/22
    
    \setcounter{section}{7}
    \setcounter{subsection}{15}
    \switchcolumn
    
    \item JämF är sektionens jämlikhetsråd, och syftar till att arbeta för en mer jämlik sektion där alla känner sig välkomna.
	    
    \item JämF består av sektionsstyrelsens skyddsombud, en representant vardera från de kommittéer och sektionsföreningar där intresse för medlemskap finns, \emph{samt 6 fristående ledamöter}.
    Av dessa väljs en internt till ordförande.
    
\end{lydelse}

\subsection{Mastermottagningsansvarig}
Syftet är att eliminera redundans.
Funktionären har endast en post och därmed kan det bara väljas en till posten.
Sektionsmötet är därefter ålagd att välja in den.

Härigenom föreskrivs i fråga om reglementets avsnitt 8.17

\begin{dels}
    \item att avsnittets rubrik ska lyda ''Mastermottagningsansvarig'',
    \item att avsnittet ska omnumreras 7.16,
    \item att avsnittet i sin helhet ska ha följande lydelse.
\end{dels}
\begin{lydelse}
    \setcounter{section}{8}
    \setcounter{subsection}{17}
    
    \item Mastermottagningsansvarigs syfte är att vara med och arrangera en mastermottagning vid utbildningsområdet som sektionens kandidatprogram tillhör.
        
    \item Mastermottagningsansvarig är en till antalet.
    
    \item Det åligger mastermottagningsansvarig
    \begin{description}
        \item[att] tillsammans med andra mastermottagningsansvariga vid utbildningsområdet för sektionens kandidatprogram arrangera en mottagning för nya studenter vid relaterade masterprogram.
    \end{description}
    
    \setcounter{section}{7}
    \setcounter{subsection}{16}
    \switchcolumn
    
    \item Mastermottagningsansvarigs syfte är att vara med och arrangera en mastermottagning vid utbildningsområdet som sektionens kandidatprogram tillhör.
        
    \item[] \emph{(redundant)}  
    
    \item Det åligger mastermottagningsansvarig
    \begin{aligganden}
        \vspace{-0.4em}
        \item tillsammans med andra mastermottagningsansvariga vid utbildningsområdet för sektionens kandidatprogram arrangera en mottagning för nya studenter vid relaterade masterprogram.
    \end{aligganden}
    
\end{lydelse}

\subsection{Frisörer}
Syftet är enbart redaktionellt.

Härigenom föreskrivs i fråga om reglementets avsnitt 8.18

\begin{dels}
    \item att avsnittets rubrik ska lyda ''Frisörer'',
    \item att avsnittet ska omnumreras 7.17,
    \item att avsnittet i sin helhet ska ha följande lydelse.
\end{dels}
\begin{lydelse}
    \setcounter{section}{8}
    \setcounter{subsection}{18}
    
    \item Frisörernas syfte är att sköta om F-sektionens sektionssten, Einsten.
        
    \item Frisörerna är till antalet 0--2.
    
    \setcounter{section}{7}
    \setcounter{subsection}{17}
    \switchcolumn
    
    \item Frisörernas syfte är att sköta om \emph{sektionens} sektionssten, Einsten.
        
    \item Frisörerna är till \emph{antalet 2}.
    
\end{lydelse}

\section{Intresseföreningar}
Syftet är enbart redaktionellt.

Härigenom föreskrivs i fråga om reglementets kapitel 9

\begin{dels}
    \item att kapitlets rubrik ska lyda ''Intresseföreningar'',
    \item att kapitlet ska omnumreras 8,
    \item att kapitlet i sin helhet ska ha följande lydelse.
\end{dels}

\subsection{Förteckning}
\begin{lydelse}
    
    \item[] (se \S9.1, \S9.2, \S9.3, \S9.4)
   
    \switchcolumn
    
    \item \emph{Sektionens intresseföreningar är}
    \begin{itemize}
        \item \emph{F-spexet} \\
            {\itshape F-spexets syftet är att årligen verka för att sätta upp ett spex, och på så sätt sprida spexkulturen i Göteborgsområdet allt medan spexets medlemmar har roligt.}
        \item \emph{3Dteamet} \\
            {\itshape 3Dteamets syfte är att verka för att till sina medlemmar tillgängliggöra 3D-skrivare och annan utrustning som föreningen har att tillgå samt att administrera och underhålla utrustning och hemsida.}
        \item \emph{FyS} \\
            {\itshape Fysikteknologsektionens spelförenings syfte är att främja intresset av digitala spel på sektionen.}
    \end{itemize}
    
\end{lydelse}

\section{Valberedningen}
Syftet är att eliminera redundans med övriga delar av reglemente och stadga. 

Härigenom föreskrivs i fråga om reglementets kapitel 3

\begin{dels}
    \item att kapitlets rubrik ska lyda ''Valberedningen''
    \item att kapitlet ska omnumreras 9
\end{dels}

\subsection{Åligganden}
Syftet är att eliminera redundans med övriga delar av reglemente och stadga. 

Härigenom föreskrivs i fråga om reglementets avsnitt 3.1

\begin{dels}
    \item att avsnittets rubrik ska lyda ''Åligganden'',
    \item att avsnittet ska omnumreras 9.1,
    \item att avsnittet i sin helhet ska ha följande lydelse.
\end{dels}

\begin{lydelse}
    \setcounter{section}{3}
    \setcounter{subsection}{1}

    \item Det åligger valberedningen att ansvara för nomineringen till följande poster:
	\begin{itemize}
		\item Samtliga poster i sektionsstyrelsen
		\item Samtliga förtroendeposter
		\item Övriga poster i SNF
		\item Övriga ledamöter i Djungelpatrullen
		\item Övriga ledamöter i F6
		\item Övriga ledamöter i Focumateriet
		\item Övriga ledamöter i FARM
		\item Övriga ledamöter i FnollK.
	\end{itemize}
	
	\item Det åligger valberedningen att anslå nomineringar till poster i sektionsstyrelsen, förtroendeposter och övriga medlemmar i sektionskommittéer senast 7 veckodagar före sektionsmöte då relevant inval äger rum. Nomineringarna skall anslås via någon av sektionens officiella informationskanaler.

    \item Vid varje sektionsmöte, där det sker inval av poster där valberedningen ansvarar för nomineringar, skall valberedningen vara representerad.
    
    \setcounter{section}{9}
    \setcounter{subsection}{0}
    
    \switchcolumn
    \item \emph{Valberedningen ansvarar utöver åliggande i stadga dessutom för nomineringar till}
    \item[] \emph{(ersätts av stadga \S \ref{13.x:valBfp})}
    \vspace{4ex}
    \begin{itemize}
        \item \emph{Studienämnden} %Note: Sortera efter namn som på andra ställen?
        \item \emph{Djungelpatrullen}
        \item \emph{F6}
        \item \emph{Focumateriet}
        \item \emph{FARM}
        \item \emph{FnollK}
    \end{itemize}
    
    \item[] \emph{(ersätts av stadga \S \ref{13.x:valBfp}, stadga \S \ref{S:ValBAnslaNomTid}, \\
    stadga \S \ref{S:ValBAnslaNom}, reglemente \S \ref{R:Tillkannagivande})}
    
    \vspace{14ex}
    \item[] \emph{(ersätts av stadga \S \ref{4.1:rep})}
\end{lydelse}


\setcounter{subsection}{1}
\subsection{Nomineringsbeslut}
Syftet är att eliminera redundans med övriga delar av reglemente och stadga. 

Härigenom föreskrivs i fråga om reglementets avsnitt 3.3

\begin{dels}
    \item att avsnittets rubrik ska lyda ''Nomineringsbeslut'',
    \item att avsnittet ska omnumreras 9.2,
    \item att avsnittet i sin helhet ska ha följande lydelse.
\end{dels}

\begin{lydelse}
    \setcounter{subsection}{3}
    \setcounter{section}{3}

    \item[] ---
    \vspace{5.5ex}
    \item[] (se \S3.3.2, \S3.3.1)
    
    \vspace{11.3ex}\setcounter{enumi}{2}
    \item[]
    \item[]\vspace{1.8em}
    \item Om sektionsmötet sedan väljer in alla förtroendeposter i berörd kommitté i enlighet med valberedningens nominering fastslås gruppen omedelbart och de övriga medlemmarna är att betrakta som invalda.

	\item Om sektionsmötet inte väljer in alla förtroendeposter i berörd kommitté i enlighet med valberedningens nominering förkastas den preliminärt fastslagna gruppen. Därefter skall val av övriga medlemmar istället företas på sektionsmöte.
    
    \item[] (se \S3.3.1, \S3.3.5)
    
    \setcounter{subsection}{0}
    \setcounter{section}{9}
    \switchcolumn
    \item \emph{Ledamot i valberedningen får inte delta i beslut som rör ledamoten själv eller annan närstående, eller om annat jäv föreligger.}
    
    \item \emph{Poster utöver förtroendeposter i organ kan nomineras i grupp av valberedningen.
    Sådan gruppnominering ska fastställas av sektionsstyrelsen senast tre läsdagar innan sektionsmötet då valet äger rum.}
    \label{R:valb:grupp}
    
    \item[] \emph{(ersätts av \S \ref{R:SektmoteGruppnom})}
    \vspace{12ex}

    \item[] \emph{(ersätts av \S \ref{R:SektmoteGruppnom})}
    
    \vspace{4.8em}
    \item \emph{Sektionsstyrelsen bedömer då valberedningens arbete. 
    Avslag av gruppnomineringen får ske om synnerliga skäl föreligger.
    Beslut om avslag anslås omedelbart med motivering.}
\end{lydelse}

\section{Talmanspresidiet}
Syftet är att bryta ut talmanspresidiet till ett eget kapitel samt att förtydliga åligganden för presidiet.

Härigenom föreskrivs i fråga om reglementets avsnitt 2.5

\begin{dels}
    \item att \S\S 2.5.1,2.5.3,2.5.5,2.5.7 ska utgå,
    \item att avsnittet ska bli eget kapitel,
    \item att kapitlets rubrik ska lyda ''Talmanspresidiet'',
    \item att kapitlet ska numreras 10,
\end{dels}

Förklaring till strukna punkter:
\begin{lydelse}
    \setcounter{section}{2}
    \setcounter{subsection}{5}
    \item Talmanspresidiet består av talman, vice talman och sekreterare.
    
    \setcounter{enumi}{2}
    \item Det åligger vice talman att stödja talmannen i dennes arbete.
    
    \setcounter{enumi}{4}
    \item Vid talmannens frånvaro övertar vice talman dennes åligganden och befogenheter.

    \setcounter{enumi}{6}    
    \item Vid vakant post i talmanspresidiet väljer sektionsmötet in en person för det aktuella sektionsmötet på förslag av styrelsen. Personen får vara medlem i sektionsstyrelsen förutom när det gäller talman samt vice talman när denne skall överta talmannens arbete. 
    
    \setcounter{section}{10}
    \setcounter{subsection}{0}
    \switchcolumn
    
    \item[] \emph{(ersätts av stadga \S \ref{S:TalPStrukt})}
    
    \vspace{1.2em}
    \item[] \emph{(redundant)}
    
    \vspace{1.2em}
    \item[] \emph{(ersätts av stadga \S \ref{S:TalPViceTOrdf})}
    
    \vspace{1.2em}
    \item[] \emph{(ersätts av mötesordning)}
    
\end{lydelse}

\subsection{Åligganden}
Syftet är redaktionellt

Härigenom föreskrivs i fråga om delar från reglementets avsnitt 2.5

\begin{dels}
    \item att avsnittets rubrik ska lyda ''Åligganden'',
    \item att avsnittet ska numreras 10.1,
    \item att avsnittet i sin helhet ska ha följande lydelse.
\end{dels}

\begin{lydelse}
    \setcounter{section}{2}
    \setcounter{subsection}{5}
    
    \item[] (se \S2.5.2, \S2.5.6, \S2.5.8)
    
    \vspace{14.3ex}\setcounter{enumi}{3}
    \item Det åligger sekreterare att föra protokoll under sektionsmöten samt tillse att protokollet justeras och anslås i tid. 
    
    \setcounter{section}{10}
    \setcounter{subsection}{1}
    \switchcolumn
    
   \item \emph{Det åligger talman:}
    \begin{aligganden}
        \item \emph{tillse att sektionsmöten utlyses och fortlöper stadgeenligt.}
        \item \emph{opartiskt leda sektionsmötet och särskilt bevaka demokratiska principer.}
    \end{aligganden}
    
   \item \emph{Det åligger talmanspresidiets sekreterare:}
    \begin{aligganden}
        \item \emph{föra protokoll under sektionsmötet och tillse att det justeras och anslås i tid.}
    \end{aligganden}

\end{lydelse}

\section{Ekonomi och revision}
Syftet är att samla alla ekonomiska punkter som föreligger många organ på ett och samma ställe.

Härigenom föreskrivs i fråga om reglementet

\begin{dels}
    \item att ett nytt kapitel införs
    \item att kapitlets rubrik ska lyda ''Ekonomi och revision''
    \item att kapitlet ska numreras 11
\end{dels}

\subsection{Verksamhetsår}
Syftet är att samla sektionens verksamhetsår under ett avsnitt istället för att individuellt under olika organs avsnitt förteckna dem.

Härigenom föreskrivs i fråga om reglementets nya kapitel 11.

\begin{dels}
    \item att ett nytt avsnitt införs
    \item att avsnittets rubrik ska lyda ''Verksamhetsår'',
    \item att avsnittet ska omnumreras 11.1,
    \item att avsnittet i sin helhet ska ha följande lydelse.
\end{dels}
\begin{lydelse}
    
    \item[] (se \S5.1.3.1, \S6.6.2, \S6.7.2, \S6.8.2)
    \item[]
    \item[] (se \S6.4.2, \S6.5.2, \S7.5.2)
    
    \switchcolumn
  
   \item \emph{Om inget annat anges har ett organ samma verksamhetsår som sektionen.}
    \label{R:VanligtVerksamhetsar}
    
   \item \emph{Följande organs verksamhetsår löper från 1 januari till 31 december:}
   \label{R:BrutnaVerksamhetsar}
    \begin{itemize}
        \item \emph{FARM}
        \item \emph{FnollK}
        \item \emph{FIF}
    \end{itemize} 


\end{lydelse}

\subsection{Redovisning och ansvarsfrihet}
Syftet är att samla kraven för sektionens organs ekonomiska redovisning under en punkt.
Krav på delbokslut och årsredovisning läggs på organen då det är ett krav för sektionsstyrelsens ekonomiska årsredovisning; det har tidigare varit ett krav.
Detta avsnitt verkställer också flytten av rapport-/dechargeärenden för flertalet organ till sektionsstyrelsen med undantag för studienämnden.

Härigenom föreskrivs i fråga om reglementets nya kapitel 11.

\begin{dels}
    \item att ett nytt avsnitt införs,
    \item att avsnittets rubrik ska lyda ''Redovisning och ansvarsfrihet'',
    \item att avsnittet ska omnumreras 11.2,
    \item att avsnittet i sin helhet ska ha följande lydelse.
\end{dels}
\begin{lydelse}
    
    \item[] (se \S6.3.2, \S7.3.2)
    
    \vspace{3em}
    \item[] ---
    
    \vspace{2.2em}
    \item[] ---
    
    \switchcolumn
  
   \item \emph{Följande organ, utöver studienämnden och kommittéer, upprättar egen löpande redovisning:} \label{R:FIFred}
    \begin{itemize}
        \item \emph{FIF}
    \end{itemize}
    
   \item \emph{Delbokslut och årsredovisning enligt detta avsnitt presenteras för sektionsstyrelsen inom fyra läsveckor efter organets verksamhetsårs slut.}
    
   \item \emph{Studienämnden presenterar dessutom verksamhetsberättelse vid första ordinarie sektionsmöte efter verksamhetsårets slut.
    Handlingarna tillställs enligt vad som stadgats för sektionsstyrelsens årsredovisning.}

\end{lydelse}
\subsection{Fonder}
Syftet är att samla punkterna om sektionens fonder under ett avsnitt samt att generalisera och gruppera dem där de är lika. 

Härigenom föreskrivs i fråga om reglementets gamla kapitel 11

\begin{dels}
    \item att kapitlet ska bli ett avsnitt
    \item att avsnittets rubrik ska lyda ''Fonder'',
    \item att avsnittet ska numreras 11.3,
    \item att avsnittet i sin helhet ska ha följande lydelse.
\end{dels}
\begin{lydelse}
    \setcounter{section}{11}
    \setcounter{subsection}{1}

    \item Fysikteknologsektionens fonder är:
	\begin{itemize}
		\item F-fonden
		\item Focus-fonden.
	\end{itemize}
	
	\item[] (se \S11.2.3, \S11.3.3)
	
	\item[] (se \S11.2.7, \S11.3.7)
    
    \setcounter{section}{11}
    \setcounter{subsection}{3}
    \switchcolumn
  
   \item \emph{Sektionens} fonder är:
    \begin{itemize}
        \item F-fonden
        \item \emph{Focusfonden}
    \end{itemize}

   \item \emph{Fonderna handhas av sektionsstyrelsen.}
   \label{R:FondHantSekt}
    
   \item \emph{Fondernas medel placeras i räntefonder med låg risk.
    Ränteavkastningen återinvesteras.}
    \label{R:FondRantAter}

\end{lydelse}
\textbf{F-fonden}
\begin{lydelse}
    
    \setcounter{subsection}{2}
    
    \item[] ---
    
    
    \vspace{1.3em}\setcounter{enumi}{1}
    \item F-Fondens medel kan användas till:
	\begin{itemize}
		\item Renovering av Fysikteknologsektionens lokaler.
		\item Införskaffande och renovering av sektionsbil.
		\item Inköp av inventarier.
		\item Att täcka eventuellt överdrag av budgeten.
		\item Sektionens oförutsedda utgifter.
		\item Annat som sektionsmötet finner lämpligt.
	\end{itemize}
	
	
	\item Fonden handhas av sektionsstyrelsen.
	
    \item Uttag ur F-fonden beslutas av sektionsmöte. Sådant uttag kan dock ej göras om ärendet inte upptagits på slutlig föredragningslista.
    
    \item Sektionsstyrelsen får för den löpande driften låna medel ur F-fonden. Vid sådant lån behöver fonden ej kompenseras för ränteförlusten.
    
    \setcounter{enumi}{0}
    \item Sektionens medel bör göras räntebärande genom placering i räntefond.
    
    \setcounter{enumi}{5}
    \item Överblivna medel inom Fysikteknologsektionen skall i möjligaste mån tillfalla fonden. Har styrelsen vid verksamhetsårets slut ej tagit samtliga medel inom budgeten i anspråk, skall överskjutande medel avsättas till F-fonden.
    
    \item F-fondens medel skall göras räntebärande genom placering i lågriskfonder. Räntan skall fonderas.
    
    \item När F-fonden understiger det under året gällande basbeloppet skall avsättning till F-fonden ske årligen med minst 15 \% av inbetalda sektionsavgifter tills fonden återigen uppgår till minst ett basbelopp.

    \setcounter{subsection}{3}
    \switchcolumn
    \setcounter{enumi}{3}
    
    \item \emph{F-fonden syftar till att säkerställa sektionens likviditet.}
    
    \item \emph{ F-fondens medel kan användas till det som sektionsmötet finner lämpligt, t.ex. renovering av sektionens lokaler, inköp och underhåll av inventarier så som sektionsbil, oförutsedda utgifter och täckande av underskott.}
    
    
    
    \vspace{7.3em}
    \item[]
    \item[] \emph{(ersätts av \S\ref{R:FondHantSekt})}
    
    \item Uttag ur F-fonden beslutas av sektionsmöte. \emph{Ärendet ska anges i kallelse.}
    
    \vspace{3ex}
    \item Sektionsstyrelsen får \emph{om nödvändigt kortvarigt} låna medel ur F-fonden.\emph{ I sådant fall} behöver fonden ej kompenseras för ränteförlusten.
    
    \item[] \emph{(ersätts av \S\ref{R:Fondoverskott})}
    
    \vspace{1.2em}
    \item \emph{Överskott i sektionens verksamhet tillfaller F-fonden.}
    \label{R:Fondoverskott}
    
    \vspace{8.5ex}
    \item[] \emph{(ersätts av \S\ref{R:FondRantAter})}
    
    \vspace{2.75ex}
    \item \emph{Om F-fondens värde understiger ett prisbasbelopp ska avsättning göras med minst 15 procent av inbetalda sektionsavgifter tills dess denna nivå uppnås.}
\end{lydelse}
\textbf{Focusfonden}
\begin{lydelse}
    
    \item Att fondera medel för framtida upprustning \\ av Focus.

	\item Focus-fondens medel kan användas till:
	\begin{itemize}
		\item Renovering av Focus
		\item Inköp av inventarier till Focus.
	\end{itemize}

	\item Fonden handhas av sektionsstyrelsen i \\ samarbete med DP.

	\item Uttag ur fonden beslutas av sektionsstyrelsen.

    \vspace{2.75ex}
	\item Överblivna medel inom budgetposten Focus \\ upprustning tillfaller Focus-fonden vid \\ verksamhetsårets slut.

	\item Samtliga hyresintäkter för uthyrning av Focus tillfaller Focus-fonden.

	\item Focus-fondens medel skall göras räntebärande genom placering i lågriskfonder. Räntan skall fonderas.
    
    \switchcolumn
    \setcounter{enumi}{9}
    
    \item \emph{Focusfonden syftar till att möjliggöra större upprustningar av Focus.}
    
    \item \emph{Focusfondens medel kan användas till renovering av Focus eller inköp av inventarier till Focus.}
    
    \vspace{6ex}
    \item[] \emph{(ersätts av \S\ref{R:FondHantSekt}, \S\ref{R:FocusFondUttag})}
    
    \vspace{2.5ex}
    \item Uttag ur \emph{Focusfonden} beslutas av sektionsstyrelsen \emph{i samråd med Djungelpatrullen}.
    \label{R:FocusFondUttag}
    
    \item \emph{Överskott inom budgetposten ''Focus upprustning'' samt samtliga hyresintäkter för Focus tillfaller Focusfonden.}
    \label{R:FocusFondOverskott}
    
    \item[] \emph{(ersätts av \S\ref{R:FocusFondOverskott})}
    
    \vspace{3ex}
    \item[] \emph{(ersätts av \S\ref{R:Fondoverskott})}

\end{lydelse}

\section{Styrdokument}
Syftet är att samla alla punkterna kring sektionens olika styrdokument under ett kapitel.

Härigenom föreskrivs i fråga om reglementets kapitel 10

\begin{dels}
    \item att kapitlets rubrik ska lyda ''Styrdokument''
    \item att kapitlet ska omnumreras 12
\end{dels}

\subsection{Förteckning}
Syftet är att befästa synnerligen viktiga dokument i reglementet. 

Härigenom föreskrivs i fråga om reglementets avsnitt 10.1

\begin{dels}
    \item att avsnittets rubrik ska lyda ''Förteckning'',
    \item att avsnittet ska omnumreras 12.1,
    \item att avsnittet i sin helhet ska ha följande lydelse.
\end{dels}
\begin{lydelse}
    
    \setcounter{section}{10}
    
    \item Följande styrdokument skall finnas på sektionen:
	\begin{itemize}
		\item Sammanträdesordning sektionsmötet
		\item Riktlinjer för verksamhetsberättelse och ansvarsfrihet
		\item Arbetsordningar
	\end{itemize}
    Därtill regleras verksamheten av övriga dokument som sektionsstyrelsen beslutat vara styrdokument. Komplett förteckning över övriga styrdokument skall tillhandahållas av sektionsstyrelsen.

    \vspace{1.4em}
    \item[] (se \S 10.1.1)
    
    \setcounter{section}{12}
    \switchcolumn
  
    \item Följande \emph{övriga} styrdokument \emph{ska} finnas på sektionen:
    \label{R:ForteckningStyrdokument}
    \begin{itemize}
        \item Arbetsordningar
        \item Sammanträdesordning \emph{för} sektionsmötet
        \item \emph{Förteckning över sektionsmötesbeslut}
        \item \emph{Ekonomisk policy}
        \item \emph{Integritetspolicy}
        \item \emph{Miljö- och hållbarhetspolicy}
        \item \emph{Lokalpolicy Focus}
    \end{itemize}
    
    \item Därtill \emph{får sektionsstyrelsen förklara ytterligare dokument} vara styrdokument.
    Komplett förteckning över övriga styrdokument ska tillhandahållas av sektionsstyrelsen.

\end{lydelse}
\subsection{Ändring}
Syftet är redaktionellt med tillägget att väsentliga ändringar i styrdokument ska anslås.

Härigenom föreskrivs i fråga om delar av reglementets avsnitt 10.2

\begin{dels}
    \item att avsnittets rubrik ska lyda ''Ändring'',
    \item att avsnittet ska omnumreras 12.2,
    \item att avsnittet i sin helhet ska ha följande lydelse.
\end{dels}
\begin{lydelse}
    \setcounter{section}{10}
    \setcounter{subsection}{2}
    
    \item Vad gäller ändring och tolkning av enskilda styrdokument ska detta anges i de individuella dokumenten. Om uteblivet sker ändringar och tolkningar av sektionsstyrelsen via enkel majoritet.
    
    \item[] ---
    
    \setcounter{section}{12}
    \setcounter{subsection}{2}
    \switchcolumn
  
    \item \emph{Ändring av övrigt styrdokument görs av sektionsstyrelsen med enkel majoritet om inget annat anges.}

    \item[]
    \item \emph{Sektionsstyrelsen ska anslå väsentliga ändringar gjorda i styrdokument.}

\end{lydelse}
\subsection{Tillkännagivande}
\label{R:Tillkannagivande}
Syftet är att gruppera och beteckna entydigt vad innebördan av att ''anslå'' någonting betyder.

Härigenom föreskrivs i fråga om reglementets nya kapitel 12

\begin{dels}
    \item att ett nytt avsnitt införs,
    \item att avsnittets rubrik ska lyda ''Tillkännagivande'',
    \item att avsnittet ska numreras 12.3,
    \item att avsnittet i sin helhet ska ha följande lydelse.
\end{dels}
\begin{lydelse}
    
    \item[] (se \S2.1.1, \S2.2.1, \S3.2.1, \S3.3.2, \S3.3.5, \\, \S4.2.4, \S4.2.10, \S4.3.7, \S 5.1.4.5)
    
    \switchcolumn
  
    \item \emph{Vad som i stadga, reglemente eller annat styrdokument ska anslås görs så via sektionens officiella kommunikationskanaler, vilka utgörs av sektionens hemsida \href{ftek.se} samt sektionens anslagstavlor.}

\end{lydelse}
\subsection{Sekretess}
\label{R:sekretess}
Syftet är att samla punkterna från reglementet om sekretess under ett avsnitt. 
Tillägg är att sektionsstyrelsen kan om synnerliga skäl föreligger med enhälligt beslut lämna ut sekretessbelagd uppgift.

Härigenom föreskrivs i fråga om delar av reglementets nya kapitel 12

\begin{dels}
    \item att ett nytt avsnitt införs,
    \item att avsnittets rubrik ska lyda ''Sekretess'',
    \item att avsnittet ska numreras 12.4,
    \item att avsnittet i sin helhet ska ha följande lydelse.
\end{dels}
\begin{lydelse}
    
    \item[] (se \S 1.2.1)
    
    \vspace{4.4em}
    \item[] (se \S4.3.6)
    
    \vspace{1.4em}
    \item[] (se \S1.2.1, \S4.2.10)
    
    \vspace{2.4em}
    \item[] ---
    
    \switchcolumn
  
    \item \emph{Följande dokument har medlem ej rätt att ta del av:}
    \begin{itemize}
        \item \emph{Incidenthanteringsprotokoll}
        \item \emph{Handling innehållande personuppgifter}
    \end{itemize}
    
    \item \emph{Därutöver har sektionsstyrelsens skyddsombud tystnadsplikt i sitt uppdrag.}
    
    \item \emph{Sektionsaktiv får dock hantera sekretessbelagd uppgift i organets ordinarie verksamhet, men inte föra den vidare.}
    
    \item \emph{Sektionsstyrelsen kan om synnerliga skäl föreligger med enhälligt beslut lämna ut sekretessbelagd uppgift.}

\end{lydelse}
\subsection{Tolkningstvister}
Syftet är redaktionellt. 

Härigenom föreskrivs i fråga om delar av reglementets avsnitt 10.2

\begin{dels}
    \item att avsnittets rubrik ska lyda ''Tolkningstvister'',
    \item att avsnittet ska omnumreras 12.5,
    \item att avsnittet i sin helhet ska ha följande lydelse.
\end{dels}
\begin{lydelse}
    \setcounter{section}{10}
    \setcounter{subsection}{2}
    
    \item Vad gäller ändring och tolkning av enskilda styrdokument ska detta anges i de individuella dokumenten. Om uteblivet sker ändringar och tolkningar av sektionsstyrelsen via enkel majoritet. 

	\item Uppstår tolkningstvist om detta reglementes tolkning skall frågan hänskjutas till sektionens inspektor.

    \item Vid konflikt med Fysikteknologsektionens övriga styrdokument, undantaget stadgan, har reglementet företräde.
    
    \setcounter{section}{12}
    \setcounter{subsection}{5}
    \switchcolumn
  
    \item \emph{Tolkning av övrigt styrdokument görs av sektionsstyrelsen med enkel majoritet om inget annat anges.}
    
    \vspace{2.4em}
    \item \emph{Vid tvist om reglementets tolkning avgörs frågan av inspektor.}
    
    \vspace{1.2em}
    \item[] \emph{(ersätts av stadga \S \ref{S:TolkningStadgaFore})}

\end{lydelse}

\section*{Avsnitt som utgår}
Syftet är att hölja detaljnivån på reglementet; overallregler flyttas till policy.

Härigenom föreskrivs i fråga om reglementets gamla kapitel 12 att kapitlet utgår.

\end{document}

%Notes
% Enl. 12.3.1 måste styretkallelserna anslås officiellt
% Komitteers överblivna tillgångar tillfaller ej sektionen?
% Motivera Finform blir sektionsförening
% Ekonomiskt ansvar för FIFs ordf/kassör är struket ur reglemetet men inte ersatt av stadgan som för kommittéer/snf
% CTRL + F efter R:Broke för att hitta ställen där jag inte vet vad/om det ersätts av något

% Mina reglementesändringar
% Stod Fabiola istället för Foton på ett ställe
% Ändrade FIF till Finform i 6.7.2