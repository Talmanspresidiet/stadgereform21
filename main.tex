\documentclass{article}
\usepackage[T1]{fontenc}
\usepackage[utf8]{inputenc}
\usepackage[swedish]{babel}
\usepackage[a4paper,margin=2cm,headheight=12pt,includehead,includefoot]{geometry}
\usepackage{fancyhdr}
\usepackage{graphicx}
\usepackage{lastpage}
\usepackage{enumitem}
\usepackage{paracol}
\usepackage{parskip}
\usepackage{hyperref}
\usepackage{xfrac}

\newcommand{\motionsnamn}{Totalreform av styrdokument}
\newcommand{\motionar}{Medlemmar insamlade som ämnar revidera allt (MISÄRA)}

\newlist{beslut}{itemize}{1}
\setlist[beslut]{label=\textbf{att}}
\newlist{dels}{itemize}{1}
\setlist[dels]{label=\emph{dels},noitemsep}

\newenvironment{lydelse}
    {\begin{paracol}{2}%
        \emph{Nuvarande lydelse}%
        \switchcolumn%
        \emph{Föreslagen lydelse}%
    \end{paracol}%
    \begin{enumerate}[label=\thesubsection.\arabic*]%
    \begin{paracol}{2}%
    }{\end{paracol}\end{enumerate}}
\newcommand{\itemb}{\item[\textbullet]}


\pagestyle{fancy}
\fancyhead[L]{Motion om \motionsnamn}
\fancyhead[R]{\emph{\motionar}}
\cfoot{\thepage\ (\pageref*{LastPage})}

\begin{document}
\begin{center}
\textsc{\huge\itshape Betänkande}\\[0.5cm]
\textsc{\Huge Motion om}\\[0.5cm]
\textsc{\huge \motionsnamn}\\[0.5cm]
\textsc{\large \motionar\\}
Ruben Seyer (ordf.),
Felix Augustsson,
Erik Broback,
Alexandru Golic,
Linnea Hallin,
Joseph Löfving,
David Winroth
\end{center}

\section*{Bakgrund}
Sektionens styrdokument är i grunden en sammansättning av varierande förlagor från kåren och andra sektioner som genom åren lappats och lagats med diverse mindre ändringar, tills de i nuläget innehåller flertalet redundanser, otydligheter och föråldrade bestämmelser.

Vårt mål är att genom en total genomgång av gällande dokument ge styrdokumenten enhetligt språk, reda ut oklarheter, kodifiera praxis och effektivisera sektionens arbete.

\section*{Sammanfattning}
En sammanfattning av de väsentliga ändringarna jämfört med tidigare dokument är
\begin{itemize}
    \item Före detta särskild ledamot erhåller rösträtt (och ska likställas med ordinarie medlem).
    \item Inspektor omfattas nu av offentlighetsprincipen.
    \item SAMO stryks helt från stadga (och kvarstår som post i styrelsen enligt reglemente).
    \item Praxis kodifieras angående avsägelser och entlediganden.
    \item Bestämmelser om misstroendeförklaring förtydligas, men ska i allt materiellt tolkas lika.
    \item 
    \item TODO
\end{itemize}    

\section*{Yrkanden}
\begin{beslut}
	\item sektionsmötet antar förslag till reviderad stadga.
	\item sektionsmötet antar förslag till reviderat reglemente, som träder i kraft samtidigt som reviderad stadga.
	\item sektionsstyrelsen åläggs göra övriga ändringar i styrdokument tills det att reviderad stadga träder i kraft.
	\item sektionsmötet beslutar om följande övergångsbestämmelser:
	\begin{itemize}
        \item alla särskilda ledamöter utnämns till särskilda medlemmar då reviderad stadga träder i kraft.
	\end{itemize}
\end{beslut}

\tableofcontents
\clearpage

%%%%%%%%%%%%%%%%%%%%%%%%%%%%%%%%%%%%%%%%%%

\part{Stadga}
\section{Inledande bestämmelser}
Härigenom föreskrivs i fråga om stadgans kapitel 1 att kapitlets rubrik ska lyda ''Inledande bestämmelser''.


\subsection{Ändamål}
Syftet är att förnya språket och ta bort redundans.

Härigenom föreskrivs i fråga om stadgans avsnitt 1.1 att \S 1.1.1 ska ha följande lydelse.

\begin{lydelse}
    \item Fysikteknologsektionen, benämns nedan som sektionen, vid Chalmers studentkår är en ideell förening bestående av studerande vid utbildningsprogrammen Teknisk fysik eller Teknisk matematik vid Chalmers tekniska högskola och av studenter vid därtill associerade mastersprogram som betalat sektionsavgift till densamma.
  \switchcolumn
    \item Fysikteknologsektionen \emph{vid Chalmers studentkår, nedan benämnt sektionen}, är en ideell förening \emph{för studenter} vid utbildningsprogrammen Teknisk fysik, Teknisk matematik \emph{och därtill associerade mastersprogram} vid Chalmers tekniska högskola.
\end{lydelse}

\subsection{Verksamhet}
Syftet är att tillfoga styrelsens säte, vilket är allmän föreningsformalia som saknas.

Härigenom föreskrivs i fråga om stadgans avsnitt 1.2
\begin{dels}
    \item att avsnittets rubrik ska lyda ''Verksamhet'',
    \item att \S 1.2.2 ska införas av följande lydelse.
\end{dels}

\begin{lydelse}
    ---
  \switchcolumn
  \setcounter{enumi}{1}
    \item \emph{Sektionsstyrelsen ska ha sitt säte i Göteborg.}
\end{lydelse}

\subsection{Skyddshelgon}
Syftet är att flytta bestämmelserna om skyddshelgon, som inte kan anses motivera ett eget kapitel.

Härigenom föreskrivs i fråga om stadgans kapitel 17
\begin{dels}
    \item att kapitlet ska omformas och flyttas till avsnitt 1.3,
    \item att det nya avsnittet i sin helhet ska ha följande lydelse.
\end{dels}

\begin{lydelse}
  \setcounter{section}{17}
  \setcounter{subsection}{1}
    \item Sektionens skyddshelgon är Dragos.
    \item Sektionens medlemmar vördar Dragos.
    \item[] (se \S 4.2.2)
  \setcounter{section}{1}
  \setcounter{subsection}{3}
  \switchcolumn   
  \setcounter{enumi}{0}
    \item Sektionens skyddshelgon är Dragos.
    \item Sektionens medlemmar vördar Dragos.
    \item Dragos är sektionens högste beskyddare och utövar fanfareriets högsta befäl.
      \label{1.3:beskyddare}
\end{lydelse}

\section{Medlemskap}
Härigenom föreskrivs i fråga om stadgans kapitel 2 att kapitlets rubrik ska lyda ''Medlemskap''.

\subsection{Allmänt}
Syftet är att konsolidera allmänna bestämmelser.
Efterkommande ändringar i kapitlet påverkar detta avsnitt.

Härigenom föreskrivs i fråga om stadgans avsnitt 2.1 och 2.4
\begin{dels}
    \item att rubriken för avsnitt 2.1 ska lyda ''Allmänt'',
    \item att avsnitt 2.4 ska utgå,
    \item att avsnitt 2.1 i sin helhet ska ha följande lydelse.
\end{dels}

\begin{lydelse}
    \item Medlem i sektionen är kårmedlem som studerar vid utbildningsprogrammen för Teknisk fysik eller Teknisk matematik vid Chalmers tekniska högskola, som betalt sektionsavgift. Dessutom är kårmedlemmar som studerar vid programmen associerade mastersprogram, som betalt sektionsavgift, medlemmar. Därutöver kan sektionen ha hedersmedlemmar, seniormedlemmar, phatri\-arker/\-mathri\-arker, och särskild ledamot.
    
  \switchcolumn
  \setcounter{enumi}{0}
    \item \emph{Sektionen har ordinarie medlemmar}, hedersmedlemmar, seniormedlemmar, \emph{emeriti} och särskilda \emph{medlemmar}.
    
  \switchcolumn*
  \switchcolumn
    \item \emph{Medlem som ska erlägga sektionsavgift, eller summa motsvarande denna, gör detta terminsvis.}

  \switchcolumn*
    (\S 2.2.1, \S 2.5.4, \S 2.6.2, \S 2.7.2, \S 2.8.2)
  \switchcolumn
    \item Medlem \emph{äger} närvaro-, och yttranderätt på sektionsmöte.
      \label{2.1:närvaro}
    
  \switchcolumn*
    (\S 2.2.5, \S 2.8.2)
  \switchcolumn
    \item Medlem \emph{äger} rätt att ta del av mötesprotokoll och sektionens övriga handlingar, undantaget de dokument som listas som icke offentliga i reglementet.
      \label{2.1:offentlighet}
    
  \switchcolumn*
    (\S 2.3.1, \S 2.5.5, \S 2.6.3, \S 2.7.3, \S 2.8.3)

  \switchcolumn
    \item Medlem är skyldig att rätta sig efter sektionens stadgar, regle\-mente, beslut samt övriga styrdokument.
      \label{2.1:skyldig}
    
  \switchcolumn*
  \setcounter{subsection}{4}
  \setcounter{enumi}{0}
    \item Sektionens förtroendeposter avser poster listade i reglementet som förtroendeposter.
    
    \item Sektionsaktiv är person vald till post av sektionsmötet eller sektionsstyrelsen.
    
  \switchcolumn
    \emph{(ersätts av \S \ref{4.x:fortroende}, \S \ref{4.x:aktiv})}

  \switchcolumn*
  \setcounter{subsection}{1}
  \setcounter{enumi}{1}
    \item Person sittandes på en post inom sektionen och vars medlemskap avslutas behöver sektionsstyrelsens godkännande för att preliminärt sitta kvar.
    Beslutet fastställs på det kommande sektionsmötet.

  \switchcolumn
    \emph{(ersätts av \S\ref{4.x:kvarstå})} %Person \emph{vald till} post inom sektionen och vars medlemskap avslutas behöver sektionsstyrelsens godkännande för att sitta kvar.
    %Beslutet fastställs på \emph{nästa} sektionsmöte.
\end{lydelse}

\subsection{Ordinarie medlemmar}
Syftet är att ta bort tvetydigheten i begreppet medlem genom att införa begreppet \emph{ordinarie medlem}, så att begrepp medlem kan reserveras för den generella bemärkelsen.
Dessutom görs strukturen enhetlig med övriga typer av medlem.
Konceptet motionsrätt anses redundant och helt täckt av förslagsrätten.

Härigenom föreskrivs i fråga om stadgans avsnitt 2.2 och 2.3
\begin{dels}
    \item att rubriken för avsnitt 2.2 ska lyda ''Ordinarie medlemmar'',
    \item att avsnitt 2.3 ska utgå,
    \item att avsnitt 2.2 i sin helhet ska ha följande lydelse.
\end{dels}

\begin{lydelse}
    (se \S 2.1.1)

  \switchcolumn
    \item \emph{Ordinarie} medlem i sektionen är kårmedlem som studerar vid utbildningsprogrammen för Teknisk fysik, Teknisk matematik \emph{eller därtill associerat mastersprogram} vid Chalmers tekniska högskola \emph{och har} betalt sektionsavgift.

    \subsubsection*{Rättigheter}
  \switchcolumn*
  \setcounter{subsection}{2}
    \item Varje medlem har närvaro-, yttrande-, förslags- och rösträtt på sektionsmöte.
   
    \item Varje medlem har motionsrätt till sektionsmöte
    
  \switchcolumn
    \item \emph{Ordinarie} medlem \emph{äger dessutom} förslags- och rösträtt på sektionsmöte.

  \switchcolumn*

    \item Medlem är valbar till post inom sektionen.

    \item Medlem har rätt till medlemskap i sektionens intresseföreningar.

    \item Medlem har rätt att ta del av mötesprotokoll och sektionens övriga handlingar, undantaget de dokument som listas som icke offentliga i reglementet.
    
    \item Medlem har rätt att utnyttja av sektionen erbjudna tjänster.
    
  \switchcolumn
    
    \item \emph{Ordinarie} medlem är valbar till post inom sektionen.
    
    \item \emph{Ordinarie} medlem \emph{äger} rätt till medlemskap i sektionens intresseföreningar.
    
    \emph{(ersätts av \S \ref{2.1:offentlighet})}
    
    \item \emph{Ordinarie} medlem \emph{äger} rätt att \emph{nyttja} av sektionen erbjudna tjänster.
    
    %\subsubsection*{Skyldigheter}
    
  \switchcolumn*
  \setcounter{subsection}{3}
  \setcounter{enumi}{0}
    \item Medlem är skyldig att rätta sig efter sektionens stadgar, regle\-mente, övriga beslut samt Fysikteknologsektionens övriga styrdokument.
    
  \switchcolumn
    \emph{(ersätts av \S \ref{2.1:skyldig})}
\end{lydelse}

\setcounter{subsection}{2}
\subsection{Hedersmedlemmar}
Syftet är att numrera alla punkter och uppdatera bestämmelser som försummats vid tidigare revisioner.
Dessutom görs bestämmelser om arrangemangsrätt enhetliga.

Härigenom föreskrivs i fråga om stadgans avsnitt 2.5
\begin{dels}
    \item att avsnittet omnumreras till 2.3,
    \item att avsnittet i sin helhet ska ha följande lydelse.
\end{dels}

\begin{lydelse}%
    \subsubsection*{Grundkrav}
    \itemb Till hedersmedlem kan kallas nu levande person som främjat F- eller TM-tekno\-loger, sektionen, ämnesområdet fysik eller matematik eller på annat sätt till\-för\-skansat sig F- eller TM-tekno\-logens vördnad och respekt.
    \item[]
    
\switchcolumn
    \setcounter{subsection}{3}
    \item Sektionens hedersmedlemmar \emph{förtecknas} i reglementet.
    
    \item Till hedersmedlem kan kallas nu levande person som främjat F- eller TM-tekno\-loger, sektionen, ämnesområdet fysik eller matematik\emph{;} eller på annat sätt till\-för\-skansat sig F- eller TM-tekno\-logens vördnad och respekt.
    
\switchcolumn*
    \subsubsection*{Förslag och kallande}%
    \itemb Förslag till hedersmedlem lämnas skriftligt till sektionsstyrelsen med minst 25 namnunderskrifter från sektionsmedlemmar.

    \itemb Ärendet ska tas upp på nästa sektionsmöte. Beslut om kallande skall bifallas med minst 2/3 majoritet.
    
    \itemb Vid bifall kallas personen till nästa sektionsmöte där val förrättas.

    \itemb Personen skall närvara vid valet, eller ha inkommit med skriftligt bifall.

    \itemb Beslut om inval skall bifallas med minst 2/3 majoritet.
    
\switchcolumn
    \subsubsection*{Förslag och kallande}%
    
    \item Förslag till hedersmedlem lämnas \emph{skriftligen} till sektionsstyrelsen \emph{och talmanspresidiet} med minst 25 namnunderskrifter från medlemmar.

    \item Ärendet ska tas upp på nästa sektionsmöte\emph{, dock tidigast 6 läsdagar efter att förslaget inkommit}.
    Beslut om kallande skall \emph{fattas} med minst \sfrac{2}{3} majoritet.
    
    \item Vid bifall kallas personen till nästa sektionsmöte\emph{, adjungerad med närvaro- och yttranderätt, där} val förrättas.

    \item Personen ska närvara vid valet, eller ha inkommit med skriftligt bifall.

    \item Beslut om inval ska \emph{fattas} med minst \sfrac{2}{3} majoritet.
    
\switchcolumn*
    \subsubsection*{Förteckning}%
    \itemb Sektionens hedersmedlemmar är listade i reglementet.
    
\switchcolumn
\switchcolumn*
    \subsubsection*{Hedersmedlemmars rättigheter}%
    \itemb Hedersmedlem har närvaro- och yttranderätt på sektionsmöte.
    
    \itemb Hedersmedlem har närvaro- och intaganderätt på alla sektionens
arrangemang.
    
\switchcolumn
    \subsubsection*{Rättigheter}%
    \emph{(ersätts av \S \ref{2.1:närvaro})}
    
    \item Hedersmedlem \emph{äger} närvaro- och intaganderätt på alla sektionens arrangemang \emph{öppna för samtliga medlemmar}.

%TODO \clearpage
\switchcolumn*
    \subsubsection*{Hedersmedlemmars skyldigheter}%
    \itemb Hedersmedlem är skyldig att rätta sig efter sektionens stadgar, regle\-mente, övriga beslut samt  Fysikteknologsektionens övriga styrdokument.
    
\switchcolumn
    \emph{(ersätts av \S \ref{2.1:skyldig})}
\end{lydelse}

\subsection{Seniormedlemmar}
Syftet är att numrera alla punkter och uppdatera bestämmelser som försummats vid tidigare revisioner.

Härigenom föreskrivs i fråga om stadgans avsnitt 2.6
\begin{dels}
    \item att avsnittet omnumreras till 2.4,
    \item att avsnittet i sin helhet ska ha följande lydelse.
\end{dels}

\begin{lydelse}%
    \subsubsection*{Definition}
    \itemb F- eller TM-teknolog har rätt att efter avslutade eller definitivt avbrutna studier skriftligen ansöka om seniormedlemskap av sektionsstyrelsen som därefter fastställer seniormedlemskap. Seniormedlem kvarstår som senior\-med\-lem så länge en summa motsvarande sektionsavgiften skänks till sektionen.

\switchcolumn
  \setcounter{enumi}{0}
    \item \emph{Seniormedlem är person utnämnd efter ansökan som har skänkt en summa motsvarande sektionsavgiften till sektionen.}
    
    \subsubsection*{Ansökan}
  
    \item F- eller TM-teknolog \emph{äger} rätt att efter avslutade eller definitivt avbrutna studier ansöka om seniormedlemskap.
    \emph{Ansökan lämnas skriftligen till sektionsstyrelsen.}
  
    \item \emph{Sektionsstyrelsen utnämner seniormedlemmar efter ansökan.}
    
\switchcolumn*
    \subsubsection*{Seniormedlemmars rättigheter}%
    \itemb Seniormedlem har närvaro- och yttranderätt på sektionsmöte.

    \itemb Seniormedlem har närvaro- och intaganderätt på alla sektionens arrangemang som är öppna för samtliga sektionens medlemmar.

    \itemb Seniormedlem har rätt att utnyttja av sektionen erbjudna tjänster.
   
    \itemb Seniormedlem är valbar till post inom sektionen.
    
\switchcolumn
    \subsubsection*{Rättigheter}%
    \emph{(ersätts av \S \ref{2.1:närvaro})}
    
    \item Seniormedlem är valbar till post inom sektionen.
    
    \item Seniormedlem \emph{äger} rätt att utnyttja av sektionen erbjudna tjänster.
   
    \item Seniormedlem \emph{äger} närvaro- och intaganderätt på alla sektionens arrangemang öppna för samtliga medlemmar.

  \switchcolumn*
    \subsubsection*{Seniormedlemmars skyldigheter}%
    \itemb Seniormedlem är skyldig att rätta sig efter sektionens stadgar, regle\-mente, övriga beslut samt  Fysikteknologsektionens övriga styrdokument.
    
  \switchcolumn
    \emph{(ersätts av \S \ref{2.1:skyldig})}
\end{lydelse}

\subsection{Emeriti}
Syftet är att numrera alla punkter och uppdatera bestämmelser som försummats vid tidigare revisioner.
Vidare ersätts begreppet phatriark/mathriark av ett nytt könsneutralt begrepp i linje med sektionens policy.
Omdefinitionen av medlemskap innebär automatisk rätt att kvarstå på post.

Härigenom föreskrivs i fråga om stadgans avsnitt 2.7
\begin{dels}
    \item att avsnittets rubrik ska lyda ''Emeriti''
    \item att avsnittet omnumreras till 2.5,
    \item att avsnittet i sin helhet ska ha följande lydelse.
\end{dels}

\begin{lydelse}%
  \subsubsection*{Definition}
    \itemb Med Phatriark/Mathriark avses den som som avlagt masters- eller civil\-ingenjörs\-examen som medlem av Fysik\-teknolog\-sektionen vid Ch\-al\-mers tekniska högskola.

  \switchcolumn
  \setcounter{enumi}{0}
    \item \emph{Emeriti är} den som avlagt masters- eller civilingenjörsexamen som \emph{ordinarie} medlem av sektionen.
    
  \switchcolumn*
    \subsubsection*{Phatriarkers/Mathriarkers rättigheter}%
    \itemb Phatriark/Mathriark har närvaro- och yttranderätt på sektionsmöten.

    \itemb Phatriark/Mathriark må kvarstå på post inom sektionen.
    
  \switchcolumn
    \emph{(ersätts av \S \ref{2.1:närvaro})}

  \switchcolumn*
    \subsubsection*{Phatriarkers/Mathriarkers skyldigheter}%
    \itemb Phatriark/Mathriark är skyldig att rätta sig efter sektionens stadgar, reglemente, övriga beslut samt Fysikteknologsektionens övriga styrdokument.
    
  \switchcolumn
    \emph{(ersätts av \S \ref{2.1:skyldig})}
\end{lydelse}

\subsection{Särskild medlem}
Syftet är att numrera alla punkter och uppdatera bestämmelser som försummats vid tidigare revisioner.
Begreppet ledamot ersätts av medlem, eftersom det redan är i bruk i andra mer vanligt förekommande definitioner.
Enligt den demokratiska principen att den som betalar avgift till föreningen också ska vara fullvärdig medlem tillskrivs nu särskild medlem per automatik alla rättigheter som ordinarie medlem, inklusive rösträtt.
(Det är dock olämpligt att icke-kårmedlemmar har denna rätt, så detsamma gäller ej seniormedlemmar.)
\emph{Tillhörande övergångsbestämmelse finns.}

Härigenom föreskrivs i fråga om stadgans avsnitt 2.8
\begin{dels}
    \item att avsnittets rubrik ska lyda ''Särskild medlem'',
    \item att avsnittet omnumreras till 2.6,
    \item att avsnittet i sin helhet ska ha följande lydelse.
\end{dels}

\begin{lydelse}%
    \subsubsection*{Definition}
    \itemb Särskild ledamot är kårmedlem vid Chalmers tekniska högskola som sektionsmöte med minst 2/3 majoritet beslutar, och som skänkt en summa motsvarande sektionsavgiften till sektionen.
  
    \itemb Kårmedlem vid Chalmers Tekniska Högskola har rätt att söka till särskild ledamot. Kårmedlem som önskar söka särskild ledamot skall meddela detta till sektionsstyrelsen och talmanspresidiet senast 6 läsdagar i förväg.

  \switchcolumn
  \setcounter{enumi}{0}
    \item \emph{Särskild medlem} är kårmedlem vid Chalmers tekniska högskola, \emph{utnämnd av sektionsmötet, som har} skänkt en summa motsvarande sektionsavgiften till sektionen.
    
    \subsubsection*{Ansökan}
    \item Kårmedlem vid Chalmers tekniska högskola har rätt att söka till särskild \emph{medlem}.
    \emph{Ansökan lämnas skriftligen} till sektionsstyrelsen och talmanspresidiet.
    
    \item \emph{Ärendet ska tas upp på nästa sektionsmöte, dock tidigast 6 läsdagar efter att förslaget inkommit.
    Beslut om utnämnande ska fattas med minst \sfrac{2}{3} majoritet.}
    
\switchcolumn*
    \subsubsection*{Särskilda ledamöters rättigheter}%
    \itemb Särskild ledamot har närvaro- och yttranderätt på
   sektionsmöte.
   
   \itemb Särskild ledamot är valbar till post inom sektionen.

   \itemb Särskild ledamot har rätt att ta del av mötesprotokoll och sektionens övriga handlingar, undantaget de dokument som listas som icke offentliga i reglementet.
   
   \itemb Särskild ledamot har rätt att utnyttja av sektionen erbjudna tjänster.
    
\switchcolumn
    \subsubsection*{Rättigheter}%
    %\emph{(ersätts av \S \ref{2.1:närvaro})}

    %\item Särskild \emph{medlem} är valbar till post inom sektionen.
  
    %\emph{(ersätts av \S \ref{2.1:offentlighet})}

    %\item Särskild \emph{medlem äger} rätt att utnyttja av sektionen erbjudna tjänster.

    \item \emph{Särskild medlem har samma rättigheter som ordinarie medlem.}

\switchcolumn*
    \subsubsection*{Särskilda ledamöters skyldigheter}%
    \itemb Särskild ledamot är skyldig att rätta sig efter sektionens stadgar, regle\-mente, övriga beslut samt  Fysikteknologsektionens övriga styrdokument.
    
\switchcolumn
    \emph{(ersätts av \S \ref{2.1:skyldig})}
\end{lydelse}

\section{Inspektor}
Syftet är att numrera alla punkter och använda enhetligt språk.
Eftersom alla sammanträden är tillgängliga för inspektor görs även offentlighetsprincipen gällande.

Härigenom föreskrivs i fråga om stadgans kapitel 3 att kapitlet i sin helhet ska ha följande lydelse.

\begin{lydelse}
    \itemb Sektionens inspektor skall vara fysik- eller matematikprofessor vid institutionen för fysik eller institutionen för matematiska vetenskaper, och tillvarata F- och TM-teknologens intressen, samt fungera som en länk mellan teknologer och anställda.
	
	\itemb Sektionens inspektor väljs på två på varandra följande sektionsmöten med minst 2/3 majoritet, för mandat på tre år. 
	
	\itemb Förslag till inspektor inlämnas till sektionsstyrelsen med minst 25 namnunderskrifter från sektionsmedlemmar eller lyfts av sektionsstyrelsen med enkel majoritet. Innan val skall sektionsstyrelsen tillfråga den nominerade om denne är att betrakta som valbar.
	
    \itemb Inspektors mandat kan förlängas med tre år åt gången av sektionsmötet, om detta sker med enkel majoritet. Om mandatet ej förlängs skall nyval av inspektor ske på nästkommande sektionsmöte. Inspektors mandat förlängs då tills dess att ny inspektor blivit vald.

    \itemb Inspektor har närvaro-, yttrande- och förslagsrätt vid sammanträde i sektionens samtliga organ.
    
  \switchcolumn
  \setcounter{enumi}{0}
    
    \item Sektionens inspektor \emph{ska} vara fysik- eller matematikprofessor vid institutionen för fysik eller institutionen för matematiska vetenskaper och tillvarata F- och TM-teknologens intressen, samt fungera som en länk mellan teknologer och anställda.
	
	\item Sektionens inspektor väljs på två på varandra följande sektionsmöten med minst \sfrac{2}{3} majoritet, för mandat på tre år. 
	
	\item \emph{Nominering} till inspektor inlämnas till sektionsstyrelsen med minst 25 namnunderskrifter från medlemmar, eller lyfts av sektionsstyrelsen med enkel majoritet.
    Innan val skall sektionsstyrelsen \emph{inhämta} den nominerades \emph{samtycke till kandidatur}.
	
	\item Inspektors mandat kan förlängas med tre år åt gången av sektionsmötet.
    Om mandatet ej förlängs \emph{ska} nyval av inspektor ske på \emph{nästa} sektionsmöte.
	\emph{Nuvarande inspektor kvarstår interimistiskt} tills dess att ny inspektor blivit vald.

    \item Inspektor \emph{äger} närvaro-, yttrande- och förslagsrätt vid sammanträde i sektionens samtliga organ.
    
    \item \emph{Inspektor äger rätt att ta del av mötesprotokoll och sektionens övriga handlingar, undantaget de dokument som listas som icke offentliga i reglementet.}
\end{lydelse}

\section{Organisation och ansvar}
\subsection{Verksamhetsutövning}
Syftet är att förnya språket och konsolidera verksamhetsgemensamma bestämmelser.
Studerandearbetsmiljöombud är inte längre ett separat organ inom sektionen.

Härigenom föreskrivs i fråga om stadgans avsnitt 4.1
\begin{dels}
    \item att rubriken för avsnittet ska lyda ''Verksamhetsutövning'',
    \item att avsnittet i sin helhet ska ha följande lydelse.
\end{dels}

\begin{lydelse}
    \item Sektionens verksamhet ut\-övas på det sätt denna stadga med
   till\-hör\-ande regle\-mente föreskriver genom:
      \begin{enumerate}[label=\arabic*]
       \item Sektionsmötet
       \item Studerandearbetsmiljöombud
       \item Sektionens valberedning
       \item Sektionens revisorer
       \item Talmanspresidiet
       \item Sektionsstyrelsen
	   \item Sektionsordföranden
       \item Studienämnden
       \item Sektionskommitéer
       \item Sektionsföreningar
       \item Sektionsfunktionärer
       \item Intresseföreningar
      \end{enumerate}
    
  \switchcolumn
  \setcounter{enumi}{0}
    
    \item Sektionens verksamhet utövas på det sätt denna stadga med
   tillhörande reglemente föreskriver genom \emph{följande organ}:
      \begin{itemize}
       \item Sektionsmötet
       \item Talmanspresidiet
       \item Valberedningen
       \item Revisorer
       \item Sektionsstyrelsen
       \item Sektionsordföranden
       \item Studienämnden
       \item Kommittéer
       \item Sektionsföreningar
       \item Funktionärer
       \item Intresseföreningar
      \end{itemize}
    
  \switchcolumn*
  \setcounter{enumi}{1}
    
	  \item Uppgifter och ansvar får delegeras enligt följande hierarki:
		\begin{enumerate}
			\item[-] Sektionsmötet har rätt att delegera både ansvar och uppgifter till valberedning, revisor, talmanspresidie, sektionsstyrelse, nämnder, sektionskommittéer, sektionsföreningar och sektionsfunktionärer.
			\item[-] Sektionsstyrelsen har rätt att delegera uppgifter till nämnder,  sektionskommittéer, sektionsföreningar och sektionsfunktionärer, förutsatt att uppgiften ligger under deras verksamhetsområde.
		\end{enumerate}
	
	\switchcolumn
	  \emph{(ersätts av \S\S \ref{4.1:högststart}--\ref{4.1:högstend}, \S\ref{4.x:delegation})}
  	%\item Uppgifter och ansvar får delegeras enligt följande hierarki:
  	%	\begin{itemize}
  	%		\item Sektionsmötet \emph{äger} rätt att delegera både ansvar och uppgifter till \emph{alla organ, undantaget intresseföreningar}.
  	%		\item Sektionsstyrelsen \emph{äger} rätt att delegera uppgifter till studienämnden,  kommittéer, sektionsföreningar och funktionärer, förutsatt att uppgiften ligger under deras verksamhetsområde.
  	% \end{itemize}
    
	\switchcolumn*
    \item[] (se \S 8.1.5, \S 9.2.1, \S 10.2.1, \S 11.2.1, \S 12.4)

    \item[] (se \S 8.1.6, \S 9.3.1, \S 10.3.1, \S 11.3.1, \S 12.5)

  \switchcolumn
    \item \emph{Alla organ} äger rätt att i namn och emblem använda sektionens namn och dess symboler.
      \label{4.1:emblem}

    \item \emph{Alla organ} är skyldiga att rätta sig efter sektionens stadgar, reglemente\emph{, beslut} samt övriga styrdokument.
      \label{4.1:rätta}

\end{lydelse}

\subsection{Mandat}
Syftet är att införa ett avsnitt som konsoliderar frågor om mandat och val, eftersom det ansvaret utövas delvis parallellt av sektionsmöte och styrelse.
Dessutom kodifieras praxis om entlediganden.

Härigenom förskrivs i fråga om stadgan
\begin{dels}
    \item att ett nytt avsnitt 4.2 införs,
    \item att avsnittet i sin helhet ska ha följande lydelse.
\end{dels}

\begin{lydelse}
    (se \S 2.4.1, \S 2.4.2)
  
  \switchcolumn
    \item Sektionsaktiv är person vald till post av sektionsmötet eller sektionsstyrelsen.
      \label{4.x:aktiv}
  
    \item Förtroendeposter \emph{förtecknas} i reglementet.
    \emph{Förtroendevald är person vald till förtroendepost.}
      \label{4.x:fortroende}

    \item \emph{Ekonomiskt ansvariga förtroendevalda förtecknas i reglementet.}

  \switchcolumn*
    (se \S 5.13.1)
  \switchcolumn
    \subsubsection*{Tillsättande}
    \item \emph{Post i sektionens organ tillsätts i vanliga fall genom personval på sektionsmöte.}

    \item Vid \emph{vakans kan} sektionsstyrelsen preliminärt tillsätta posten.
    \emph{Beslutet fastställs} på nästkommande sektionsmöte.

  \switchcolumn*
    \item[] (se \S 2.1.2, \S 4.3.6, \S 4.3.8)
    \item[] (se \S 4.3.8)
  
  \switchcolumn
    \subsubsection*{Entledigande}
    \item \emph{Sektionsaktiv kan avsäga sig sin post och på begäran entledigas av sektionsstyrelsen.
    Beslutet fastställs på nästkommande sektionsmöte.}

    \item \emph{Sektionsaktiv som upphör som medlem ska entledigas, såvida inte sektionsstyrelsen beslutar annorlunda.
    Beslutet fastställs på nästkommande sektionsmöte.}
      \label{4.x:kvarstå}

    \item \emph{Vid entledigande av förtroendevald ska extra sektionsmöte utlysas inom 15 läsdagar där fyllnadsval ska äga rum}.

    \item \emph{Vid entledigande av ekonomiskt ansvarig ska bokslut för perioden fram till fyllnadsvalet upprättas inom fyra veckor från fyllnadsvalet.}
      \label{4.2:enteko}
    
\end{lydelse}


\subsection{Ansvarsförhållanden}
Syftet är att göra frågan om ansvar mindre stel och eliminera redundansen i frågan om delegation.

Härigenom föreskrivs i fråga om stadgans avsnitt 4.2
\begin{dels}
    \item att avsnittet omnumreras till 4.3,
    \item att avsnittet i sin helhet ska ha följande lydelse.
\end{dels}

\begin{lydelse}
	\item Sektionsmötet är sektionens högsta beslutande organ.
    \item Dragos är sektionens högste beskyddare och utövar fanfareriets högsta befäl.
    \item Sektionsmötet har till sitt förfogande valberedning,revisorer, talmanspresidiet, sektionsstyrelsen och sektionsordförande.
    \item Övrig verksamhet lyder under sektionsstyrelsen, enligt stadgans kapitel~7.
    
  \switchcolumn
  \setcounter{enumi}{0}  
    \item Sektionsmötet är sektionens högsta beslutande organ.
      \label{4.1:högststart}
    
    \item[] %Dragos är sektionens högste beskyddare och utövar fanfareriets högsta
    %  befäl.
    \emph{(ersätts av \S\ref{1.3:beskyddare})}
     
    
    \item Sektionsmötet har till sitt förfogande \emph{alla sektionens organ, undantaget intresseföreningar}.
    
    \item \emph{Sektionsstyrelsen har till sitt förfogande studienämnden, kommittéer, sektionsföreningar och funktionärer, inom deras respektive verksamhetsområden}.
      \label{4.1:högstend}
	
  \switchcolumn*
    (se \S 4.3.9)
    
  \switchcolumn
    \item \emph{Förtroendevalda svarar inför sektionsmötet. Övriga sektionsaktiva svarar inför sektionsstyrelsen.}
      \label{4.2:ansvar}
    %\item \emph{Hierarkin mellan sektionens organ i fråga om ansvarsförhållanden är som följer:
	%\begin{enumerate}
	%    \item Sektionsmötet är högsta instans.
	%    \item Förtroendevalda lyder direkt under sektionsmötet.
	%    \item Övriga sektionsaktiva lyder under sektionsstyrelsen.
	%   \end{enumerate}}

  \switchcolumn*
    \subsubsection*{}
    \item[] (se \S 4.1.2, \S 8.1.6, \S 9.3.3, \S 10.3.2, \S 11.3.2)

  \switchcolumn
    \subsubsection*{Delegation} 
    \item \emph{Överordnat organ äger rätt att delegera ansvar och uppgifter samt beordra åtgärd enligt \S\S\ref{4.1:högststart}--\ref{4.1:högstend}.} 
      \label{4.x:delegation}
  
\end{lydelse}

\subsection{Misstroendeförklaring}
Syftet är att förnya språket och samla organisatoriska detaljer som varit utspridda eller redundanta.
Förenklingar genomförs i enlighet med föregående avsnitt.

Härigenom föreskrivs i fråga om stadgans avsnitt 4.3
\begin{dels}
    \item att avsnittets rubrik ska lyda ''Misstroendeförklaring'',
    \item att avsnittet omnumreras till 4.4,
    \item att avsnittet i sin helhet ska ha följande lydelse.
\end{dels}

\begin{lydelse}
    \setcounter{subsection}{3}
    \item Misstroendevotum kan kallas av
	  \begin{itemize}
		\item[-] enskild styrelseledamot, 
  		\item[-] 25 medlemmar, eller
  		\item[-] endera av sektionens revisorer
  	  \end{itemize}
	
	\item Misstroendevotum får endast behandlas av instans högre än målet, enligt punkt 4.3.9.

  	\item Misstroendevotum ska tas upp för behandling inom 15 läsdagar.

  	\item Målet för misstroendevotum har rätt att närvarva vid och delta i diskussionen rörande beslutet.

  	\item Misstroendevotum resulterar i avsättande vid 2/3 majoritet i frågan. Omröstning ska ske slutet. 

  	\item Om sektionsstyrelsen avsätts genom misstroendevotum skall interimstyrelse och ny valberedning väljas. Talmanspresidiet utfärdar kallelse till extra sektionsmöte där ny ordinarie styrelse skall väljas. Detta sektionsmöte skall hållas inom 15 läsdagar. Interimstyrelsen övertar ordinarie styrelses befogenheter och skyldigheter tills ny ordinarie styrelse är vald, men får endast handha löpande ärenden. Inom fyra veckor från valet av den nya styrelsen skall ett bokslut för perioden fram till och med datumet för detta val upprättas. Sektionsmötet beslutar i samband med avsättandet om detta bokslut skall upprättas av avgående styrelse eller den nya styrelsen. Den nya styrelsen skall ej hållas ansvarig för brister i denna ekonomiska redovisning.

  	\item Om sektionsordförande avsätts genom misstroendevotum tar vice sektionsordförande över rollen vid avsättande fram tills ny sektionsordförande är vald.

  	\item Om ekonomiskt ansvarig avsätts genom misstroendevotum skall fyllnadsval ske inom 15 läsdagar efter avsättandet. Inom fyra veckor från fyllnadsvalet skall ett bokslut för perioden fram till och med datumet för detta val upprättas. Sektionsmötet beslutar i samband med avsättande av kassören om detta bokslut skall upprättas av den avgående eller den nya kassören. Den nya kassören skall ej hållas  ansvarig för brister i denna ekonomiska redovisning. 

	\item Instanser inom sektionen rangordnas enligt följande, med avseende på misstroendevotum:
  	  \begin{itemize}
  	    \item[-] Sektionsmötet är högsta instans
  		\item[-] Sektionsstyrelsen, förtroendevalda i kommittéer, sektionsföreningar och nämnder, valberedningen, studerandearbetsmiljöombud, talmanspresidiet samt revisorer och förtroendevalda funktionärer lyder direkt under sektionsmötet
  		\item[-] Ledamöter i kommittéer, sektionsföreningar och nämnder och funktionärer som ej nämns ovan lyder under sektionsstyrelsen
  	  \end{itemize}
	
  	\item Ifall misstroendevotum av talmanspresidiet, eller enskild medlem av talmanspresidiet, lyfts tillses kallelse av sektionsmötet av sektionsordförande. Sektionsmötet väljer ett temporärt talmanspresidie till sektionsmötet.

  \switchcolumn
    \setcounter{subsection}{4}  
    \item \emph{En högre instans enligt \S\ref{4.2:ansvar} kan förklara att en sektionsaktiv inte har dess förtroende.}

    \item \emph{Yrkande om misstroendeförklaring kan väckas av antingen} styrelseledamot, 25 medlemmar eller revisor.

    \item \emph{Sådant yrkande ska prövas inom 15 läsdagar.
    Om sektionsmötet ska pröva förtroendet ska kallelsen tydligt ange att en förtroendefråga behandlas.}

    \item \emph{Svarande vars förtroende behandlas ska under alla omständigheter beredas möjlighet att närvara och tala för sin sak.}

    \item \emph{Entledigande till följd av misstroende sker med sluten omröstning med minst \sfrac{2}{3} majoritet.}
      
    \subsubsection*{Sektionsstyrelsen}
    \item Om sektionsstyrelsen \emph{i sin helhet entledigas} genom misstroende\emph{förklaring ska omedelbart} interimstyrelse och ny valberedning väljas.

    \item Interimstyrelsen övertar styrelsens befogenheter och skyldigheter \emph{tills ny styrelse är vald}, men får endast handha löpande ärenden.

    \item Om sektionsordförande \emph{entledigas} genom misstroende\emph{förklaring} tar vice sektionsordförande över \emph{som tillförordnad sådan} tills ny sektionsordförande är vald.

    \subsubsection*{Talmanspresidiet}
    \item \emph{Om yrkande om missförtroendeförklaring} eller \emph{entledigande} avser \emph{talmanspresidiet i del eller helhet sker utlysning av och kallelse till extra} sektionsmöte av sektionsordförande.
    Sektionsmötet väljer ett temporärt presidium för mötet.

    \item \emph{Om talmanspresidiet i sin helhet entledigas genom misstroendeförklaring sker utlysning av och kallelse till extra sektionsmöte där fyllnadsval äger rum av sektionsordförande.
    Sektionsmötet väljer ett temporärt presidium för mötet.}

    \item \emph{Om talmannen entledigas genom misstroendeförklaring tar vice talman över som tillfördordnad sådan tills ny talman är vald.}
      
    \subsubsection*{Ekonomiskt ansvarig}
    \item Om ekonomisk ansvarig \emph{entledigas} genom misstroende\emph{förklaring} beslutar \emph{sektionsmötet} om \emph{bokslutet enligt \S\ref{4.2:enteko}} upprättas av avgående eller nyvald kassör.

    \item Den nyvalda kassören \emph{kan} ej hållas ansvarig för brister i denna ekonomiska redovisning.

\end{lydelse}

\section{Sektionsmötet}
Syftet är redaktionellt, samt att uppdatera bestämmelser i enlighet med övriga förändringar.

Härigenom föreskrivs i fråga om stadgans avsnitt 5.1
\begin{dels}
    \item att avsnittet ej rubriksätts och införlivas under kapitelrubriken,
    \item att avsnittet i sin helhet ska ha följande lydelse.
\end{dels}

\begin{lydelse}
  \setcounter{subsection}{1}
    \item Sektionsmötet är sektionens högsta beslutande organ i vilket samtliga medlemmar äger rätt att delta och har rösträtt.

    \item[] ---
  \switchcolumn
  \setcounter{subsection}{0}
    \item Sektionsmötet är sektionsens högsta beslutande organ, i vilket samtliga medlemmar äger rätt att delta.

    \item \emph{Fråga av större principiell betydelse ska avgöras av sektionsmötet.}

    %\item \emph{Sektionsmötet regleras utöver sektionens stadgar av reglemente och mötesordning.}
\end{lydelse}

\setcounter{subsection}{0}
\subsection{Sammanträden}
Syftet är rent redaktionellt.

Härigenom föreskrivs i fråga om stadgans avsnitt 5.2
\begin{dels}
    \item att avsnittet omnumreras till 5.1,
    \item att avsnittet i sin helhet ska ha följande lydelse.
\end{dels}


   % TODO: rättigheter, hänvisa också till kapitel 2 och 3.

\clearpage
\part{Reglemente}

\end{document}
