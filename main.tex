\documentclass{article}
\usepackage[T1]{fontenc}
\usepackage[utf8]{inputenc}
\usepackage[swedish]{babel}
\usepackage[a4paper,margin=2cm,headheight=12pt,includehead,includefoot]{geometry}
\usepackage{fancyhdr}
\usepackage{graphicx}
\usepackage{lastpage}
\usepackage{enumitem}
\usepackage{paracol}
\usepackage{parskip}
\usepackage{hyperref}
\usepackage{xfrac}

\newcommand{\motionsnamn}{Totalreform av styrdokument}
\newcommand{\motionar}{Medlemmar insamlade som ämnar revidera (MISÄR)}

\newlist{beslut}{itemize}{1}
\setlist[beslut]{label=\textbf{att}}
\newlist{dels}{itemize}{1}
\setlist[dels]{label=\emph{dels},noitemsep}

\newenvironment{lydelse}
    {\begin{paracol}{2}%
        \emph{Nuvarande lydelse}%
        \switchcolumn%
        \emph{Föreslagen lydelse}%
    \end{paracol}%
    \begin{enumerate}[label=\thesubsection.\arabic*]%
    \begin{paracol}{2}%
    }{\end{paracol}\end{enumerate}}
\newcommand{\itemb}{\item[\textbullet]}


\pagestyle{fancy}
\fancyhead[L]{Motion om \motionsnamn}
\fancyhead[R]{\emph{\motionar}}
\cfoot{\thepage\ (\pageref*{LastPage})}

\begin{document}
\begin{center}
%\textsc{\huge\itshape Betänkande}\\[0.5cm]
\textsc{\Huge Motion om}\\[0.5cm]
\textsc{\huge \motionsnamn}\\[0.5cm]
\textsc{\large \motionar\\}
Ruben Seyer (föred.),
Felix Augustsson,
Erik Broback,
Alexandru Golic,
Linnea Hallin,
Joseph Löfving,
David Winroth
\end{center}

\section*{Bakgrund}
Sektionens styrdokument är i grunden en sammansättning av varierande förlagor från kåren och andra sektioner som genom åren lappats och lagats med diverse mindre ändringar, tills de i nuläget innehåller flertalet redundanser, otydligheter och föråldrade bestämmelser.

Vårt mål är att genom en total genomgång av gällande dokument ge styrdokumenten enhetligt språk, reda ut oklarheter, kodifiera praxis och effektivisera sektionens arbete.

\section*{Sammanfattning}
En sammanfattning av de väsentliga ändringarna jämfört med tidigare dokument är
\begin{itemize}
    \item Före detta särskild ledamot erhåller rösträtt (och ska likställas med ordinarie medlem).
    \item Inspektor omfattas nu av offentlighetsprincipen.
    \item Sektionsordförande och SAMO är ej längre separata organ.
    \item Praxis kodifieras angående frågor om mandat, bl.a. valbarhetshinder, avsägelser och entlediganden.
    \item Bestämmelser om misstroendeförklaring förtydligas, men ska i allt materiellt tolkas lika.
    \item TODO sektmöte
    \item En långtgående elimination av redundans i definitionen av organen, särskilt mellan stadga och reglemente.
    \item En konsolidation av bestämmelser om ekonomi för sektionen och dess organ.
    \item Förtydligande görs kring stadgeändringar och reglementesändringar för att lösa frågeställningar kring ändringsyrkanden.
\end{itemize}    

\section*{Yrkanden}
\begin{beslut}
	\item sektionsmötet antar förslag till reviderad stadga.
	\item sektionsmötet antar förslag till reviderat reglemente, som träder i kraft samtidigt som reviderad stadga.
	\item sektionsstyrelsen åläggs göra övriga ändringar i styrdokument tills det att reviderad stadga träder i kraft.
	\item sektionsmötet beslutar om följande övergångsbestämmelser:
	\begin{itemize}
    \item alla särskilda ledamöter utnämns till särskilda medlemmar då reviderad stadga träder i kraft,
    \item alla sektionsaktiva valda till post de ej längre är valbara till enligt avsnitt \ref{4.x:tillsättande} tillåts kvarstå till mandatperiodens slut.
	\end{itemize}
\end{beslut}

\tableofcontents
\clearpage

%%%%%%%%%%%%%%%%%%%%%%%%%%%%%%%%%%%%%%%%%%

\part{Stadga}
\section{Inledande bestämmelser}
Härigenom föreskrivs i fråga om stadgans kapitel 1 att kapitlets rubrik ska lyda ''Inledande bestämmelser''.


\subsection{Ändamål}
Syftet är att förnya språket och ta bort redundans.

Härigenom föreskrivs i fråga om stadgans avsnitt 1.1 att \S 1.1.1 ska ha följande lydelse.

\begin{lydelse}
    \item Fysikteknologsektionen, benämns nedan som sektionen, vid Chalmers studentkår är en ideell förening bestående av studerande vid utbildningsprogrammen Teknisk fysik eller Teknisk matematik vid Chalmers tekniska högskola och av studenter vid därtill associerade mastersprogram som betalat sektionsavgift till densamma.
  \switchcolumn
    \item Fysikteknologsektionen \emph{vid Chalmers studentkår, nedan benämnt sektionen}, är en ideell förening \emph{för studenter} vid utbildningsprogrammen Teknisk fysik, Teknisk matematik \emph{och därtill associerade mastersprogram} vid Chalmers tekniska högskola.
\end{lydelse}

\subsection{Verksamhet}
Syftet är dels att förtydliga hur begreppet verksamhetsår används, dels att tillfoga styrelsens säte vilket är allmän föreningsformalia som saknas.

Härigenom föreskrivs i fråga om stadgans avsnitt 1.2
\begin{dels}
    \item att avsnittets rubrik ska lyda ''Verksamhet'',
    \item att avsnittet ska ha följande lydelse.
\end{dels}

\begin{lydelse}
    \item Sektionens verksamhetsår löper från 1 juli till 30 juni.
    \item[] ---
  \switchcolumn
    \item Sektionens \emph{räkenskapsår} löper från 1 juli till 30 juni. \emph{Ett organ på sektionen får ha annat verksamhetsår enligt reglemente.}
    \item \emph{Sektionsstyrelsen ska ha sitt säte i Göteborg.}
\end{lydelse}

\subsection{Skyddshelgon}
Syftet är att flytta bestämmelserna om skyddshelgon, som inte kan anses motivera ett eget kapitel.

Härigenom föreskrivs i fråga om stadgans kapitel 17
\begin{dels}
    \item att kapitlet ska omformas och flyttas till avsnitt 1.3,
    \item att det nya avsnittet i sin helhet ska ha följande lydelse.
\end{dels}

\begin{lydelse}
  \setcounter{section}{17}
  \setcounter{subsection}{1}
    \item Sektionens skyddshelgon är Dragos.
    \item Sektionens medlemmar vördar Dragos.
    \item[] (se \S 4.2.2)
  \setcounter{section}{1}
  \setcounter{subsection}{3}
  \switchcolumn   
  \setcounter{enumi}{0}
    \item Sektionens skyddshelgon är Dragos.
    \item Sektionens medlemmar vördar Dragos.
    \item Dragos är sektionens högste beskyddare och utövar fanfareriets högsta befäl.
      \label{1.3:beskyddare}
\end{lydelse}

\section{Medlemskap}
Härigenom föreskrivs i fråga om stadgans kapitel 2 att kapitlets rubrik ska lyda ''Medlemskap''.

\subsection{Allmänt}
Syftet är att konsolidera allmänna bestämmelser.
Efterkommande ändringar i kapitlet påverkar detta avsnitt.

Härigenom föreskrivs i fråga om stadgans avsnitt 2.1 och 2.4
\begin{dels}
    \item att rubriken för avsnitt 2.1 ska lyda ''Allmänt'',
    \item att avsnitt 2.4 ska utgå,
    \item att avsnitt 2.1 i sin helhet ska ha följande lydelse.
\end{dels}

\begin{lydelse}
    \item Medlem i sektionen är kårmedlem som studerar vid utbildningsprogrammen för Teknisk fysik eller Teknisk matematik vid Chalmers tekniska högskola, som betalt sektionsavgift. Dessutom är kårmedlemmar som studerar vid programmen associerade mastersprogram, som betalt sektionsavgift, medlemmar. Därutöver kan sektionen ha hedersmedlemmar, seniormedlemmar, phatri\-arker/\-mathri\-arker, och särskild ledamot.
    
  \switchcolumn
  \setcounter{enumi}{0}
    \item \emph{Sektionen har ordinarie medlemmar}, hedersmedlemmar, seniormedlemmar, \emph{emeriti} och särskilda \emph{medlemmar}.
    
  \switchcolumn*
  \switchcolumn
    \item \emph{Medlem som ska erlägga sektionsavgift, eller summa motsvarande denna, gör detta terminsvis.}

  \switchcolumn*
    (\S 2.2.1, \S 2.5.4, \S 2.6.2, \S 2.7.2, \S 2.8.2)
  \switchcolumn
    \item Medlem \emph{äger} närvaro-, och yttranderätt på sektionsmöte.
      \label{2.1:närvaro}
    
  \switchcolumn*
    (\S 2.2.5, \S 2.8.2)
  \switchcolumn
    \item Medlem \emph{äger} rätt att ta del av mötesprotokoll och sektionens övriga handlingar, undantaget de dokument som listas som icke offentliga i reglemente.
      \label{2.1:offentlighet}
    
  \switchcolumn*
    (\S 2.3.1, \S 2.5.5, \S 2.6.3, \S 2.7.3, \S 2.8.3)

  \switchcolumn
    \item Medlem är skyldig att rätta sig efter sektionens stadgar, regle\-mente, beslut samt övriga styrdokument.
      \label{2.1:skyldig}
    
  \switchcolumn*
  \setcounter{subsection}{4}
  \setcounter{enumi}{0}
    \item Sektionens förtroendeposter avser poster listade i reglementet som förtroendeposter.
    
    \item Sektionsaktiv är person vald till post av sektionsmötet eller sektionsstyrelsen.
    
  \switchcolumn
    \emph{(ersätts av \S \ref{4.x:fortroende}, \S \ref{4.x:aktiv})}

  \switchcolumn*
  \setcounter{subsection}{1}
  \setcounter{enumi}{1}
    \item Person sittandes på en post inom sektionen och vars medlemskap avslutas behöver sektionsstyrelsens godkännande för att preliminärt sitta kvar.
    Beslutet fastställs på det kommande sektionsmötet.

  \switchcolumn
    \emph{(ersätts av \S\ref{4.x:kvarstå})} %Person \emph{vald till} post inom sektionen och vars medlemskap avslutas behöver sektionsstyrelsens godkännande för att sitta kvar.
    %Beslutet fastställs på \emph{nästa} sektionsmöte.
\end{lydelse}

\subsection{Ordinarie medlemmar}
Syftet är att ta bort tvetydigheten i begreppet medlem genom att införa begreppet \emph{ordinarie medlem}, så att begrepp medlem kan reserveras för den generella bemärkelsen.
Dessutom görs strukturen enhetlig med övriga typer av medlem.
Konceptet motionsrätt anses redundant och helt täckt av förslagsrätten.

Härigenom föreskrivs i fråga om stadgans avsnitt 2.2 och 2.3
\begin{dels}
    \item att rubriken för avsnitt 2.2 ska lyda ''Ordinarie medlemmar'',
    \item att avsnitt 2.3 ska utgå,
    \item att avsnitt 2.2 i sin helhet ska ha följande lydelse.
\end{dels}

\begin{lydelse}
    (se \S 2.1.1)

  \switchcolumn
    \item \emph{Ordinarie} medlem i sektionen är kårmedlem som studerar vid utbildningsprogrammen för Teknisk fysik, Teknisk matematik \emph{eller därtill associerat mastersprogram} vid Chalmers tekniska högskola \emph{och har} betalt sektionsavgift.

    \subsubsection*{Rättigheter}
  \switchcolumn*
  \setcounter{subsection}{2}
    \item Varje medlem har närvaro-, yttrande-, förslags- och rösträtt på sektionsmöte.
   
    \item Varje medlem har motionsrätt till sektionsmöte
    
  \switchcolumn
    \item \emph{Ordinarie} medlem \emph{äger dessutom} förslags- och rösträtt på sektionsmöte.

  \switchcolumn*

    \item Medlem är valbar till post inom sektionen.

    \item Medlem har rätt till medlemskap i sektionens intresseföreningar.

    \item Medlem har rätt att ta del av mötesprotokoll och sektionens övriga handlingar, undantaget de dokument som listas som icke offentliga i reglementet.
    
    \item Medlem har rätt att utnyttja av sektionen erbjudna tjänster.
    
  \switchcolumn
    
    \item \emph{Ordinarie} medlem är valbar till post inom sektionen.
    
    \item \emph{Ordinarie} medlem \emph{äger} rätt till medlemskap i sektionens intresseföreningar.
    
    \emph{(ersätts av \S \ref{2.1:offentlighet})}
    
    \item \emph{Ordinarie} medlem \emph{äger} rätt att \emph{nyttja} av sektionen erbjudna tjänster.
    
    %\subsubsection*{Skyldigheter}
    
  \switchcolumn*
  \setcounter{subsection}{3}
  \setcounter{enumi}{0}
    \item Medlem är skyldig att rätta sig efter sektionens stadgar, regle\-mente, övriga beslut samt Fysikteknologsektionens övriga styrdokument.
    
  \switchcolumn
    \emph{(ersätts av \S \ref{2.1:skyldig})}
\end{lydelse}

\setcounter{subsection}{2}
\subsection{Hedersmedlemmar}
Syftet är att numrera alla punkter och uppdatera bestämmelser som försummats vid tidigare revisioner.
Dessutom görs bestämmelser om arrangemangsrätt enhetliga.

Härigenom föreskrivs i fråga om stadgans avsnitt 2.5
\begin{dels}
    \item att avsnittet omnumreras till 2.3,
    \item att avsnittet i sin helhet ska ha följande lydelse.
\end{dels}

\begin{lydelse}%
    \subsubsection*{Grundkrav}
    \itemb Till hedersmedlem kan kallas nu levande person som främjat F- eller TM-tekno\-loger, sektionen, ämnesområdet fysik eller matematik eller på annat sätt till\-för\-skansat sig F- eller TM-tekno\-logens vördnad och respekt.
    \item[]
    
\switchcolumn
    \setcounter{subsection}{3}
    \item Sektionens hedersmedlemmar \emph{förtecknas} i reglemente.
    
    \item Till hedersmedlem kan kallas nu levande person som främjat F- eller TM-tekno\-loger, sektionen, ämnesområdet fysik eller matematik\emph{;} eller på annat sätt till\-för\-skansat sig F- eller TM-tekno\-logens vördnad och respekt.
    
\switchcolumn*
    \subsubsection*{Förslag och kallande}%
    \itemb Förslag till hedersmedlem lämnas skriftligt till sektionsstyrelsen med minst 25 namnunderskrifter från sektionsmedlemmar.

    \itemb Ärendet ska tas upp på nästa sektionsmöte. Beslut om kallande skall bifallas med minst 2/3 majoritet.
    
    \itemb Vid bifall kallas personen till nästa sektionsmöte där val förrättas.

    \itemb Personen skall närvara vid valet, eller ha inkommit med skriftligt bifall.

    \itemb Beslut om inval skall bifallas med minst 2/3 majoritet.
    
\switchcolumn
    \subsubsection*{Förslag och kallande}%
    
    \item Förslag till hedersmedlem lämnas \emph{skriftligen} till sektionsstyrelsen \emph{och talmanspresidiet} med minst 25 namnunderskrifter från medlemmar.

    \item Ärendet ska tas upp på nästa sektionsmöte\emph{, dock tidigast 6 läsdagar efter att förslaget inkommit}.
    Beslut om kallande skall \emph{fattas} med minst \sfrac{2}{3} majoritet.
    
    \item Vid bifall kallas personen till nästa sektionsmöte\emph{, adjungerad med närvaro- och yttranderätt, där} val förrättas.

    \item Personen ska närvara vid valet, eller ha inkommit med skriftligt bifall.

    \item Beslut om inval ska \emph{fattas} med minst \sfrac{2}{3} majoritet.
    
\switchcolumn*
    \subsubsection*{Förteckning}%
    \itemb Sektionens hedersmedlemmar är listade i reglementet.
    
\switchcolumn
\switchcolumn*
    \subsubsection*{Hedersmedlemmars rättigheter}%
    \itemb Hedersmedlem har närvaro- och yttranderätt på sektionsmöte.
    
    \itemb Hedersmedlem har närvaro- och intaganderätt på alla sektionens
arrangemang.
    
\switchcolumn
    \subsubsection*{Rättigheter}%
    \emph{(ersätts av \S \ref{2.1:närvaro})}
    
    \item Hedersmedlem \emph{äger} närvaro- och intaganderätt på alla sektionens arrangemang \emph{öppna för samtliga medlemmar}.

%TODO \clearpage
\switchcolumn*
    \subsubsection*{Hedersmedlemmars skyldigheter}%
    \itemb Hedersmedlem är skyldig att rätta sig efter sektionens stadgar, regle\-mente, övriga beslut samt  Fysikteknologsektionens övriga styrdokument.
    
\switchcolumn
    \emph{(ersätts av \S \ref{2.1:skyldig})}
\end{lydelse}

\subsection{Seniormedlemmar}
Syftet är att numrera alla punkter och uppdatera bestämmelser som försummats vid tidigare revisioner.

Härigenom föreskrivs i fråga om stadgans avsnitt 2.6
\begin{dels}
    \item att avsnittet omnumreras till 2.4,
    \item att avsnittet i sin helhet ska ha följande lydelse.
\end{dels}

\begin{lydelse}%
    \subsubsection*{Definition}
    \itemb F- eller TM-teknolog har rätt att efter avslutade eller definitivt avbrutna studier skriftligen ansöka om seniormedlemskap av sektionsstyrelsen som därefter fastställer seniormedlemskap. Seniormedlem kvarstår som senior\-med\-lem så länge en summa motsvarande sektionsavgiften skänks till sektionen.

\switchcolumn
  \setcounter{enumi}{0}
    \item \emph{Seniormedlem är person utnämnd efter ansökan som har skänkt en summa motsvarande sektionsavgiften till sektionen.}
    
    \subsubsection*{Ansökan}
  
    \item F- eller TM-teknolog \emph{äger} rätt att efter avslutade eller definitivt avbrutna studier ansöka om seniormedlemskap.
    \emph{Ansökan lämnas skriftligen till sektionsstyrelsen.}
  
    \item \emph{Sektionsstyrelsen utnämner seniormedlemmar efter ansökan.}
    
\switchcolumn*
    \subsubsection*{Seniormedlemmars rättigheter}%
    \itemb Seniormedlem har närvaro- och yttranderätt på sektionsmöte.

    \itemb Seniormedlem har närvaro- och intaganderätt på alla sektionens arrangemang som är öppna för samtliga sektionens medlemmar.

    \itemb Seniormedlem har rätt att utnyttja av sektionen erbjudna tjänster.
   
    \itemb Seniormedlem är valbar till post inom sektionen.
    
\switchcolumn
    \subsubsection*{Rättigheter}%
    \emph{(ersätts av \S \ref{2.1:närvaro})}
    
    \item Seniormedlem är valbar till post inom sektionen.
    
    \item Seniormedlem \emph{äger} rätt att utnyttja av sektionen erbjudna tjänster.
   
    \item Seniormedlem \emph{äger} närvaro- och intaganderätt på alla sektionens arrangemang öppna för samtliga medlemmar.

  \switchcolumn*
    \subsubsection*{Seniormedlemmars skyldigheter}%
    \itemb Seniormedlem är skyldig att rätta sig efter sektionens stadgar, regle\-mente, övriga beslut samt  Fysikteknologsektionens övriga styrdokument.
    
  \switchcolumn
    \emph{(ersätts av \S \ref{2.1:skyldig})}
\end{lydelse}

\subsection{Emeriti}
Syftet är att numrera alla punkter och uppdatera bestämmelser som försummats vid tidigare revisioner.
Vidare ersätts begreppet phatriark/mathriark av ett nytt könsneutralt begrepp i linje med sektionens policy.
Omdefinitionen av medlemskap innebär automatisk rätt att kvarstå på post.

Härigenom föreskrivs i fråga om stadgans avsnitt 2.7
\begin{dels}
    \item att avsnittets rubrik ska lyda ''Emeriti''
    \item att avsnittet omnumreras till 2.5,
    \item att avsnittet i sin helhet ska ha följande lydelse.
\end{dels}

\begin{lydelse}%
  \subsubsection*{Definition}
    \itemb Med Phatriark/Mathriark avses den som som avlagt masters- eller civil\-ingenjörs\-examen som medlem av Fysik\-teknolog\-sektionen vid Ch\-al\-mers tekniska högskola.

  \switchcolumn
  \setcounter{enumi}{0}
    \item \emph{Emeriti är} den som avlagt masters- eller civilingenjörsexamen som \emph{ordinarie} medlem av sektionen.
    
  \switchcolumn*
    \subsubsection*{Phatriarkers/Mathriarkers rättigheter}%
    \itemb Phatriark/Mathriark har närvaro- och yttranderätt på sektionsmöten.

    \itemb Phatriark/Mathriark må kvarstå på post inom sektionen.
    
  \switchcolumn
    \emph{(ersätts av \S \ref{2.1:närvaro})}

  \switchcolumn*
    \subsubsection*{Phatriarkers/Mathriarkers skyldigheter}%
    \itemb Phatriark/Mathriark är skyldig att rätta sig efter sektionens stadgar, reglemente, övriga beslut samt Fysikteknologsektionens övriga styrdokument.
    
  \switchcolumn
    \emph{(ersätts av \S \ref{2.1:skyldig})}
\end{lydelse}

\subsection{Särskild medlem}
Syftet är att numrera alla punkter och uppdatera bestämmelser som försummats vid tidigare revisioner.
Begreppet ledamot ersätts av medlem, eftersom det redan är i bruk i andra mer vanligt förekommande definitioner.
Enligt den demokratiska principen att den som betalar avgift till föreningen också ska vara fullvärdig medlem tillskrivs nu särskild medlem per automatik alla rättigheter som ordinarie medlem, inklusive rösträtt.
(Det är dock olämpligt att icke-kårmedlemmar har denna rätt, så detsamma gäller ej seniormedlemmar.)
\emph{Tillhörande övergångsbestämmelse finns.}

Härigenom föreskrivs i fråga om stadgans avsnitt 2.8
\begin{dels}
    \item att avsnittets rubrik ska lyda ''Särskild medlem'',
    \item att avsnittet omnumreras till 2.6,
    \item att avsnittet i sin helhet ska ha följande lydelse.
\end{dels}

\begin{lydelse}%
    \subsubsection*{Definition}
    \itemb Särskild ledamot är kårmedlem vid Chalmers tekniska högskola som sektionsmöte med minst 2/3 majoritet beslutar, och som skänkt en summa motsvarande sektionsavgiften till sektionen.
  
    \itemb Kårmedlem vid Chalmers Tekniska Högskola har rätt att söka till särskild ledamot. Kårmedlem som önskar söka särskild ledamot skall meddela detta till sektionsstyrelsen och talmanspresidiet senast 6 läsdagar i förväg.

  \switchcolumn
  \setcounter{enumi}{0}
    \item \emph{Särskild medlem} är kårmedlem vid Chalmers tekniska högskola, \emph{utnämnd av sektionsmötet, som har} skänkt en summa motsvarande sektionsavgiften till sektionen.
    
    \subsubsection*{Ansökan}
    \item Kårmedlem vid Chalmers tekniska högskola har rätt att söka till särskild \emph{medlem}.
    \emph{Ansökan lämnas skriftligen} till sektionsstyrelsen och talmanspresidiet.
    
    \item \emph{Ärendet ska tas upp på nästa sektionsmöte, dock tidigast 6 läsdagar efter att förslaget inkommit.
    Beslut om utnämnande ska fattas med minst \sfrac{2}{3} majoritet.}
    
\switchcolumn*
    \subsubsection*{Särskilda ledamöters rättigheter}%
    \itemb Särskild ledamot har närvaro- och yttranderätt på
   sektionsmöte.
   
   \itemb Särskild ledamot är valbar till post inom sektionen.

   \itemb Särskild ledamot har rätt att ta del av mötesprotokoll och sektionens övriga handlingar, undantaget de dokument som listas som icke offentliga i reglemente.
   
   \itemb Särskild ledamot har rätt att utnyttja av sektionen erbjudna tjänster.
    
\switchcolumn
    \subsubsection*{Rättigheter}%
    %\emph{(ersätts av \S \ref{2.1:närvaro})}

    %\item Särskild \emph{medlem} är valbar till post inom sektionen.
  
    %\emph{(ersätts av \S \ref{2.1:offentlighet})}

    %\item Särskild \emph{medlem äger} rätt att utnyttja av sektionen erbjudna tjänster.

    \item \emph{Särskild medlem har samma rättigheter som ordinarie medlem.}

\switchcolumn*
    \subsubsection*{Särskilda ledamöters skyldigheter}%
    \itemb Särskild ledamot är skyldig att rätta sig efter sektionens stadgar, regle\-mente, övriga beslut samt  Fysikteknologsektionens övriga styrdokument.
    
\switchcolumn
    \emph{(ersätts av \S \ref{2.1:skyldig})}
\end{lydelse}

\section{Inspektor}
Syftet är att numrera alla punkter och använda enhetligt språk.
Eftersom alla sammanträden är tillgängliga för inspektor görs även offentlighetsprincipen gällande.

Härigenom föreskrivs i fråga om stadgans kapitel 3 att kapitlet i sin helhet ska ha följande lydelse.

\begin{lydelse}
    \itemb Sektionens inspektor skall vara fysik- eller matematikprofessor vid institutionen för fysik eller institutionen för matematiska vetenskaper, och tillvarata F- och TM-teknologens intressen, samt fungera som en länk mellan teknologer och anställda.
	
	\itemb Sektionens inspektor väljs på två på varandra följande sektionsmöten med minst 2/3 majoritet, för mandat på tre år. 
	
	\itemb Förslag till inspektor inlämnas till sektionsstyrelsen med minst 25 namnunderskrifter från sektionsmedlemmar eller lyfts av sektionsstyrelsen med enkel majoritet. Innan val skall sektionsstyrelsen tillfråga den nominerade om denne är att betrakta som valbar.
	
    \itemb Inspektors mandat kan förlängas med tre år åt gången av sektionsmötet, om detta sker med enkel majoritet. Om mandatet ej förlängs skall nyval av inspektor ske på nästkommande sektionsmöte. Inspektors mandat förlängs då tills dess att ny inspektor blivit vald.

    \itemb Inspektor har närvaro-, yttrande- och förslagsrätt vid sammanträde i sektionens samtliga organ.
    
  \switchcolumn
  \setcounter{enumi}{0}
    
    \item Sektionens inspektor \emph{ska} vara professor vid institutionen för fysik eller institutionen för matematiska vetenskaper och tillvarata F- och TM-teknologens intressen, samt fungera som en länk mellan teknologer och anställda.
	
	\item Sektionens inspektor väljs på två på varandra följande sektionsmöten med minst \sfrac{2}{3} majoritet, för mandat på tre år. 
	
	\item \emph{Nominering} till inspektor inlämnas till sektionsstyrelsen med minst 25 namnunderskrifter från medlemmar, eller lyfts av sektionsstyrelsen med enkel majoritet.
    Innan val skall sektionsstyrelsen \emph{inhämta} den nominerades \emph{samtycke till kandidatur}.
	
	\item Inspektors mandat kan förlängas med tre år åt gången av sektionsmötet.
    Om mandatet ej förlängs \emph{ska} nyval av inspektor ske på \emph{nästa} sektionsmöte.
	\emph{Nuvarande inspektor kvarstår interimistiskt} tills dess att ny inspektor blivit vald.

    \item Inspektor \emph{äger} närvaro-, yttrande- och förslagsrätt vid sammanträde i sektionens samtliga organ.
    
    \item \emph{Inspektor äger rätt att ta del av mötesprotokoll och sektionens övriga handlingar, undantaget de dokument som listas som icke offentliga i reglemente.}
\end{lydelse}

\section{Organisation och ansvar}
\subsection{Verksamhetsutövning}
Syftet är att förnya språket och konsolidera verksamhetsgemensamma bestämmelser.
Sektionsordförande och studerandearbetsmiljöombud är ej längre separata organ inom sektionen, utan omfattas båda av sektionsstyrelsen per reglemente.

Härigenom föreskrivs i fråga om stadgans avsnitt 4.1
\begin{dels}
    \item att rubriken för avsnittet ska lyda ''Verksamhetsutövning'',
    \item att avsnittet i sin helhet ska ha följande lydelse.
\end{dels}

\begin{lydelse}
    \item Sektionens verksamhet ut\-övas på det sätt denna stadga med
   till\-hör\-ande regle\-mente föreskriver genom:
      \begin{enumerate}[label=\arabic*]
       \item Sektionsmötet
       \item Studerandearbetsmiljöombud
       \item Sektionens valberedning
       \item Sektionens revisorer
       \item Talmanspresidiet
       \item Sektionsstyrelsen
	   \item Sektionsordföranden
       \item Studienämnden
       \item Sektionskommitéer
       \item Sektionsföreningar
       \item Sektionsfunktionärer
       \item Intresseföreningar
      \end{enumerate}
    
  \switchcolumn
  \setcounter{enumi}{0}
    
    \item Sektionens verksamhet utövas på det sätt denna stadga med
   tillhörande reglemente föreskriver genom \emph{följande organ}:
      \begin{itemize}
       \item Sektionsmötet
       \item Sektionsstyrelsen
       %\item Sektionsordföranden
       \item Studienämnden
       \item Kommittéer
       \item Sektionsföreningar
       \item Funktionärer
       \item Intresseföreningar
       \item Talmanspresidiet
       \item Valberedningen
       \item Revisorer
      \end{itemize}
    
  \switchcolumn*
  \setcounter{enumi}{1}
    
	  \item Uppgifter och ansvar får delegeras enligt följande hierarki:
		\begin{enumerate}
			\item[-] Sektionsmötet har rätt att delegera både ansvar och uppgifter till valberedning, revisor, talmanspresidie, sektionsstyrelse, nämnder, sektionskommittéer, sektionsföreningar och sektionsfunktionärer.
			\item[-] Sektionsstyrelsen har rätt att delegera uppgifter till nämnder,  sektionskommittéer, sektionsföreningar och sektionsfunktionärer, förutsatt att uppgiften ligger under deras verksamhetsområde.
		\end{enumerate}
	
	\switchcolumn
	  \emph{(ersätts av \S\S \ref{4.1:högststart}--\ref{4.1:högstend}, \S\ref{4.x:delegation})}
  	%\item Uppgifter och ansvar får delegeras enligt följande hierarki:
  	%	\begin{itemize}
  	%		\item Sektionsmötet \emph{äger} rätt att delegera både ansvar och uppgifter till \emph{alla organ, undantaget intresseföreningar}.
  	%		\item Sektionsstyrelsen \emph{äger} rätt att delegera uppgifter till studienämnden,  kommittéer, sektionsföreningar och funktionärer, förutsatt att uppgiften ligger under deras verksamhetsområde.
  	% \end{itemize}
    
	\switchcolumn*
    \item[] (se \S 8.1.5, \S 9.2.1, \S 10.2.1, \S 11.2.1, \S 12.4)

    \item[] (se \S 8.1.6, \S 9.3.1, \S 10.3.1, \S 11.3.1, \S 12.5)

    \item[] (se \S 8.4.6, \S 9.3.2, \S 12.5.2)

  \switchcolumn
    \item \emph{Alla organ} äger rätt att i namn och emblem använda sektionens namn och dess symboler.
      \label{4.1:emblem}

    \item \emph{Alla organ} är skyldiga att rätta sig efter sektionens stadgar, reglemente\emph{, beslut} samt övriga styrdokument.
      \label{4.1:rätta}
      
    \item \emph{Alla organ, med undantag för funktionärer, ska ha representation vid sektionsmötet.} \label{4.1:rep}

  \switchcolumn*
      (se \S 2.4.1, \S 2.4.2)
      
  \switchcolumn
    \item Sektionsaktiv är person vald till post av sektionsmötet eller sektionsstyrelsen.
      \label{4.x:aktiv}
      
    \item Förtroendeposter \emph{förtecknas} i \emph{stadga eller} reglemente.
      \emph{Förtroendevald är person vald till förtroendepost.}
      \label{4.x:fortroende}

    \item \emph{Ekonomiskt ansvariga förtecknas i stadga eller reglemente.}

\end{lydelse}

\subsection{Ansvarsförhållanden} \label{4.2:ansvar}
Syftet är att göra frågan om ansvar mindre stel och eliminera redundansen i frågan om delegation.

Härigenom föreskrivs i fråga om stadgans avsnitt 4.2 att avsnittet i sin helhet ska ha följande lydelse.

\begin{lydelse}
	\item Sektionsmötet är sektionens högsta beslutande organ.
    \item Dragos är sektionens högste beskyddare och utövar fanfareriets högsta befäl.
    \item Sektionsmötet har till sitt förfogande valberedning,revisorer, talmanspresidiet, sektionsstyrelsen och sektionsordförande.
    \item Övrig verksamhet lyder under sektionsstyrelsen, enligt stadgans kapitel~7.
    
  \switchcolumn
  \setcounter{enumi}{0}  
    \item Sektionsmötet är sektionens högsta beslutande organ.
      \label{4.1:högststart}
    
    \item[] %Dragos är sektionens högste beskyddare och utövar fanfareriets högsta
    %  befäl.
    \emph{(ersätts av \S\ref{1.3:beskyddare})}

    \item Sektionsmötet har till sitt förfogande \emph{alla sektionens organ, undantaget intresseföreningar}.
    
    \item \emph{Sektionsstyrelsen har till sitt förfogande studienämnden, kommittéer, sektionsföreningar och funktionärer, inom deras respektive verksamhetsområden}.
      \label{4.1:högstend}
	
  \switchcolumn*
    (se \S 4.3.9)
    
  \switchcolumn
   \item \emph{Sektionsstyrelsen, talmanspresidiet, valberedningen, revisorer och förtroendevalda svarar inför sektionsmötet.} \label{4.2:ansvar1}
   \item \emph{Övriga sektionsaktiva svarar inför sektionsstyrelsen.}
      \label{4.2:ansvar2}
    %\item \emph{Hierarkin mellan sektionens organ i fråga om ansvarsförhållanden är som följer:
	%\begin{enumerate}
	%    \item Sektionsmötet är högsta instans.
	%    \item Förtroendevalda lyder direkt under sektionsmötet.
	%    \item Övriga sektionsaktiva lyder under sektionsstyrelsen.
	%   \end{enumerate}}

  \switchcolumn*
    \subsubsection*{}
    \item[] (se \S 4.1.2, \S 8.1.6, \S 9.3.3, \S 10.3.2, \S 11.3.2)

  \switchcolumn
    \subsubsection*{Delegation} 
    \item \emph{Överordnat organ enligt detta avsnitt äger rätt att delegera ansvar och uppgifter samt beordra åtgärd.} 
      \label{4.x:delegation}
  
\end{lydelse}

\subsection{Tillsättande} \label{4.x:tillsättande}
Syftet är att införa två avsnitt som konsoliderar frågor om mandat och val, eftersom det ansvaret utövas delvis parallellt av sektionsmöte och styrelse.
I detta avsnitt tydliggörs även valbarhetsregler.
\emph{Tillhörande övergångsbestämmelse finns.}

Härigenom förskrivs i fråga om stadgan
\begin{dels}
    \item att ett nytt avsnitt 4.3 införs,
    \item att rubriken för avsnittet ska lyda ''Tillsättande'',
    \item att avsnittet i sin helhet ska ha följande lydelse.
\end{dels}

\begin{lydelse}
    (se \S 5.13.1, \S 9.1.3)
  \switchcolumn
    \item \emph{Post i sektionens organ tillsätts i vanliga fall genom personval på sektionsmöte.} \label{4.x:tillsätt}

    \item Vid \emph{vakans kan} sektionsstyrelsen preliminärt tillsätta posten.
    \emph{Beslutet fastställs} på nästkommande sektionsmöte.

  \switchcolumn*
    \item[] (se \S 5.9.1, \S 6.2.2, \S\S 13.1.4--5, samt reglemente)
    \item[] (se \S 5.9.3, \S 13.1.8)
    \item[] (se \S 13.1.2, samt reglemente)
    \switchcolumn  
    \subsubsection*{Valbarhet}
    \item \emph{Ingen kan samtidigt inneha två förtroendeposter, eller två poster i samma organ.}

    \item \emph{Ingen kan samtidigt inneha post i två eller fler av sektionsstyrelsen, talmanspresidiet, revisorer och valberedning.} \label{4.x:valbar.dubbel}

    \item \emph{Talman, vice talman och ledamot i valberedningen får dessutom ej inneha post i studienämnden eller kommitté.} \label{4.x:valbar.ober}

    \item \emph{Revisor får dessutom ej inneha ekonomiskt ansvar inom sektionen.}
      
    \item \emph{Talmanspresidiet och revisorer behöver ej vara medlemmar av sektionen.}

    \item \emph{Revisor ska vara myndig.} \label{4.x:valbar.revisormyndig}
      
    \item \emph{Ekonomiskt ansvarig ska vara förtroendevald och myndig.}
\end{lydelse}

\subsection{Entledigande}
Syftet är stadga generella bestämmelser om entledigande i stället för onödigt detaljerade regler om misstroende.
Dessutom förnyas språket och organisatoriska detaljer som varit utspridda eller redundanta samlas.
Förenklingar genomförs i enlighet med föregående avsnitt.

Härigenom föreskrivs i fråga om stadgans avsnitt 4.3
\begin{dels}
    \item att avsnittets rubrik ska lyda ''Entledigande'',
    \item att avsnittet omnumreras till 4.4,
    \item att avsnittet i sin helhet ska ha följande lydelse.
\end{dels}

\begin{lydelse}
  \item[] (se \S 2.1.2, \S 4.3.6, \S 4.3.8)
  \item[] (se \S 4.3.8)
  
\switchcolumn
  \setcounter{enumi}{0}
  \item \emph{Sektionsaktiv kan avsäga sig sin post och på begäran entledigas av sektionsstyrelsen.
      Beslutet fastställs på nästkommande sektionsmöte.}

  \item \emph{Sektionsaktiv som upphör som medlem ska entledigas, såvida inte sektionsstyrelsen beslutar annorlunda.
      Beslutet fastställs på nästkommande sektionsmöte.}
  \label{4.x:kvarstå}

  \item \emph{Vid entledigande av förtroendevald ska extra sektionsmöte utlysas inom 15 läsdagar där fyllnadsval ska äga rum}.

  \item \emph{Vid entledigande av ekonomiskt ansvarig ska bokslut för perioden fram till fyllnadsvalet upprättas inom fyra veckor från fyllnadsvalet, och revision ska ske.}
    \label{4.2:enteko}

    \switchcolumn*
    \item[] (se \S\S 4.3.6--7, \S 4.3.10)
    \switchcolumn
    \subsubsection*{Sektionsstyrelsen}
    \item Om sektionsstyrelsen \emph{i sin helhet entledigas ska omedelbart} interimstyrelse och ny valberedning väljas.

    \item Interimstyrelsen övertar styrelsens befogenheter och skyldigheter \emph{tills ny styrelse är vald}, men får endast handha löpande ärenden.

    \item Om sektionsordförande \emph{entledigas} tar vice sektionsordförande över \emph{som tillförordnad sådan} tills ny sektionsordförande är vald.

    \subsubsection*{Talmanspresidiet}
    \item \emph{Om talmanspresidiet i sin helhet entledigas sker utlysning av och kallelse till extra sektionsmöte där fyllnadsval äger rum av sektionsordförande.
    Sektionsmötet väljer ett temporärt presidium för mötet.}

    \item \emph{Om talmannen entledigas tar vice talman över som tillförordnad sådan tills ny talman är vald.}
     

  \switchcolumn*
    \setcounter{subsection}{3}
    \item Misstroendevotum kan kallas av
	  \begin{itemize}
		\item[-] enskild styrelseledamot, 
  		\item[-] 25 medlemmar, eller
  		\item[-] endera av sektionens revisorer
  	  \end{itemize}
	
	\item Misstroendevotum får endast behandlas av instans högre än målet, enligt punkt 4.3.9.

  	\item Misstroendevotum ska tas upp för behandling inom 15 läsdagar.

  	\item Målet för misstroendevotum har rätt att närvarva vid och delta i diskussionen rörande beslutet.

  	\item Misstroendevotum resulterar i avsättande vid 2/3 majoritet i frågan. Omröstning ska ske slutet. 

  	\item Om sektionsstyrelsen avsätts genom misstroendevotum skall interimstyrelse och ny valberedning väljas. Talmanspresidiet utfärdar kallelse till extra sektionsmöte där ny ordinarie styrelse skall väljas. Detta sektionsmöte skall hållas inom 15 läsdagar. Interimstyrelsen övertar ordinarie styrelses befogenheter och skyldigheter tills ny ordinarie styrelse är vald, men får endast handha löpande ärenden. Inom fyra veckor från valet av den nya styrelsen skall ett bokslut för perioden fram till och med datumet för detta val upprättas. Sektionsmötet beslutar i samband med avsättandet om detta bokslut skall upprättas av avgående styrelse eller den nya styrelsen. Den nya styrelsen skall ej hållas ansvarig för brister i denna ekonomiska redovisning.

  	\item Om sektionsordförande avsätts genom misstroendevotum tar vice sektionsordförande över rollen vid avsättande fram tills ny sektionsordförande är vald.

  	\item Om ekonomiskt ansvarig avsätts genom misstroendevotum skall fyllnadsval ske inom 15 läsdagar efter avsättandet. Inom fyra veckor från fyllnadsvalet skall ett bokslut för perioden fram till och med datumet för detta val upprättas. Sektionsmötet beslutar i samband med avsättande av kassören om detta bokslut skall upprättas av den avgående eller den nya kassören. Den nya kassören skall ej hållas  ansvarig för brister i denna ekonomiska redovisning. 

	\item Instanser inom sektionen rangordnas enligt följande, med avseende på misstroendevotum:
  	  \begin{itemize}
  	    \item[-] Sektionsmötet är högsta instans
  		\item[-] Sektionsstyrelsen, förtroendevalda i kommittéer, sektionsföreningar och nämnder, valberedningen, studerandearbetsmiljöombud, talmanspresidiet samt revisorer och förtroendevalda funktionärer lyder direkt under sektionsmötet
  		\item[-] Ledamöter i kommittéer, sektionsföreningar och nämnder och funktionärer som ej nämns ovan lyder under sektionsstyrelsen
  	  \end{itemize}
	
  	\item Ifall misstroendevotum av talmanspresidiet, eller enskild medlem av talmanspresidiet, lyfts tillses kallelse av sektionsmötet av sektionsordförande. Sektionsmötet väljer ett temporärt talmanspresidie till sektionsmötet.

  \switchcolumn
    \subsubsection*{Misstroendeförklaring}
    \item \emph{En högre instans enligt avsnitt~\ref{4.2:ansvar} kan förklara att en sektionsaktiv inte har dess förtroende.}

    \item \emph{Yrkande om misstroendeförklaring kan väckas av antingen} styrelseledamot, 25 medlemmar eller revisor.

    \item \emph{Sådant yrkande ska prövas inom 15 läsdagar.
    Om sektionsmötet ska pröva förtroendet ska kallelsen tydligt ange att en förtroendefråga behandlas.}

    \item \emph{Svarande vars förtroende behandlas ska under alla omständigheter beredas möjlighet att närvara och tala för sin sak.}

    \item \emph{Entledigande till följd av misstroende sker med sluten omröstning med minst \sfrac{2}{3} majoritet.}  
     
    \item \emph{Om yrkande om missförtroendeförklaring} avser \emph{talmanspresidiet i del eller helhet sker utlysning av och kallelse till extra} sektionsmöte av sektionsordförande.
      Sektionsmötet väljer ett temporärt presidium för mötet.
    
    \item Om ekonomiskt ansvarig \emph{entledigas} genom misstroende\emph{förklaring} beslutar \emph{sektionsmötet} om \emph{bokslutet enligt \S\ref{4.2:enteko}} upprättas av avgående eller nyvald kassör. De nyvalda \emph{ekonomiskt ansvariga anses fria från ansvar för föregående redovisning.}
\end{lydelse}
\setcounter{subsection}{4}    

\subsection{Överklagan}
Syftet är att konsolidera bestämmelser om överklagan från övriga kapitel till ett enda ställe.

Härigenom föreskrivs i fråga om stadgan
\begin{dels}
  \item att ett nytt avsnitt 4.5 införs,
  \item att avsnittets rubrik ska lyda ''Överklagan'',
  \item att avsnittet i sin helhet ska ha följande lydelse.
\end{dels}

\begin{lydelse}
  \item[] (se \S 5.10.1, \S 7.7.1)
  
  \switchcolumn
  \item \emph{Beslut av sektionsmötet eller sektionsstyrelsen som strider mot kårens eller sektionens stadga, reglemente, beslut eller övriga styrdokument får undanröjas efter överklagan.}
  \item \emph{Överklagan gällande kårens styrdokument framställs av kårmedlem till kårfullmäktige.}
  \item \emph{Överklagan gällande sektionens styrdokument framställs av sektionsmedlem till kårfullmäktige, och om det gäller sektionsstyrelsebeslut även sektionsmötet.}
\end{lydelse}

\section{Sektionsmötet}
Syftet är redaktionellt, samt att uppdatera bestämmelser i enlighet med övriga förändringar.

Härigenom föreskrivs i fråga om stadgans avsnitt 5.1
\begin{dels}
    \item att avsnittet ej rubriksätts och införlivas under kapitelrubriken,
    \item att avsnittet i sin helhet ska ha följande lydelse.
\end{dels}

\begin{lydelse}
  \setcounter{subsection}{1}
    \item Sektionsmötet är sektionens högsta beslutande organ i vilket samtliga medlemmar äger rätt att delta och har rösträtt.

    \item[] ---
  \switchcolumn
  \setcounter{subsection}{0}
  \setcounter{enumi}{0}
    \item Sektionsmötet är sektionens högsta beslutande organ, i vilket samtliga medlemmar äger rätt att delta.

    \item \emph{Fråga av större principiell betydelse ska avgöras av sektionsmötet.}

    %\item \emph{Sektionsmötet regleras utöver sektionens stadgar av reglemente och mötesordning.}
\end{lydelse}

\setcounter{subsection}{0}
\subsection{Åligganden}
Syftet är rent redaktionellt.

Härigenom föreskrivs i fråga om stadgans avsnitt 5.4
\begin{dels}
    \item att avsnittet omnumreras till 5.1,
    \item att avsnittet i sin helhet ska ha följande lydelse.
\end{dels}


% TODO: rättigheter, hänvisa också till kapitel 2 och 3.
% GLÖM inte välja valberedning, vice sektionsordförande. kan kollapsa läsperioder

\section{Sektionsstyrelsen}
Syftet är redaktionellt.
Många tidigare bestämmelser har införlivats i övriga kapitel.
Definitionen av förtroendeposter lyfts från reglemente.

Härigenom föreskrivs i fråga om stadgans kapitel 7
\begin{dels}
  \item att kapitlet omnumreras till 6,
  \item att kapitlet i sin helhet, inklusive avsnitt, ska ha följande lydelse.
\end{dels}

\begin{lydelse}
  \setcounter{section}{7}
  \setcounter{subsection}{1}
  \item Sektionsstyrelsen handhar i överensstämmelse med denna stadga, befintligt reglemente samt beslut tagna av sektionsmötet den verkställande ledningen av sektionens verksamhet. Sektionsstyrelsen är sektionsmötets ställföreträdare.

  \setcounter{subsection}{3}
  \setcounter{enumi}{0}
  \item Sektionsstyrelsen ansvarar inför sektionsmötet för sektionens verksamhet och sektionens ekonomi.

  \setcounter{section}{7}
  \setcounter{subsection}{2}
  \item Sektionsstyrelsen består av sektionsordförande, vice sektionsordförande, sektionskassör samt övriga ledamöter enligt reglemente.

  \setcounter{subsection}{9}
  \item Sektionsordförande utövar i brådskande fall sektionsstyrelsens befogenheter. Ordförandebeslut skall prövas på följande styrelsemöte.

  \item I ordförandes frånvaro utövar vice sektionsordförande dennes befogenheter och fullgör dennes plikter.
\switchcolumn
  \item \emph{Sektionsstyrelsen handhar den verkställande ledningen av sektionens verksamhet och ansvar för sektionens ekonomi}.

  \item Sektionsstyrelsen består av sektionsordförande, vice sektionsordförande, sektionskassör samt övriga ledamöter enligt reglemente.
    \emph{Samtliga är förtroendeposter.}

  \item Sektionsordförande utövar i \emph{nödfall} sektionsstyrelsens befogenheter.
    Ordförandebeslut \emph{ska fastställas snarast av sektionsstyrelsen}.

  \item \emph{Vice sektionsordförande övertar i sektionsordförandes frånvaro} dennes befogenheter och plikter.

  \item \emph{Sektionsordförande och sektionskassör är ekonomiskt ansvariga.}
\end{lydelse}
\setcounter{section}{6}
\setcounter{subsection}{0}

%\subsection{Sammansättning}
%\begin{lydelse}
%\switchcolumn
%\end{lydelse}
%\setcounter{section}{6}
%\setcounter{subsection}{1}

\subsection{Styrelsemöte}
\begin{lydelse}
  \setcounter{section}{7}
  \setcounter{subsection}{5}
  \item Sektionsstyrelsen sammanträder minst tre gånger per läsperiod.
  \item Sektionsstyrelsen sammanträder på kallelse av sektionsordförande.
  \setcounter{subsection}{6}
  \setcounter{enumi}{0}
  \item Sektionsstyrelsen är beslutsmässigt om sektionsordförande eller vice sektionsordförande och sammanlagt mer än hälften av sektionsstyrelsens ledamöter är närvarande.
  \setcounter{subsection}{8}
  \setcounter{enumi}{0}
  \item Protokoll skall föras vid styrelsemöte, justeras av två
    styrelseledamöter och anslås senast två läsveckor efter mötet.
  \switchcolumn  

  \item Sektionsstyrelsen sammanträder minst tre gånger per läsperiod på kallelse av sektionsordförande.
  \item Sektionsstyrelsen är \emph{beslutsför} om \emph{mer än hälften av sektionsstyrelsens ledamöter är närvarande inklusive sektionsordförande eller vice sektionsordförande}.
  \item Protokoll förs vid styrelsemöte, justeras av två styrelseledamöter och anslås senast två läsveckor efter mötet.
\end{lydelse}
\setcounter{section}{6}
\setcounter{subsection}{2}

\section{Studienämnden}
Syftet är detsamma som för föregående kapitel.
Dessutom förtydligas rättigheterna angående studienämndsmöte.

Härigenom föreskrivs i fråga om stadgans kapitel 8
\begin{dels}
\item att kapitlet omnumreras till 7,
\item att kapitlet i sin helhet, inklusive avsnitt, ska ha följande lydelse.
\end{dels}

\begin{lydelse}
  \setcounter{section}{8}
  \setcounter{subsection}{1}
  \item Studienämnden vid Fysikteknologsektionen, även kallad SNF, har till uppgift att inom sektionen övervaka tillståndet och utvecklingen beträffande studiefrågor, aktivt verka för god kurslitteratur, främja kontakten med lärarna samt hålla god kontakt med sektionens medlemmar och styrelse.

  \setcounter{section}{8}
  \setcounter{subsection}{2}
  \item Studienämnden består av studienämndsordförande, studienämndskassör och medlemmar enligt reglemente angående studienämnden.
\switchcolumn
  \setcounter{section}{7}
  \item Studienämnden \emph{(SNF)} har till uppgift att inom sektionen övervaka tillståndet och utvecklingen beträffande studiefrågor, aktivt verka för god kurslitteratur samt främja kontakten med lärarna.
  
  \item Studienämnden består av studienämndsordförande, studienämndskassör och övriga \emph{ledamöter} enligt reglemente.

  \item \emph{Studienämndsordförande och studienämndskassör är förtroendeposter och ekonomiskt ansvariga.}
\end{lydelse}
\setcounter{section}{7}
\setcounter{subsection}{0}

%\subsection{Sammansättning}

\subsection{Studienämndsmöte}
\begin{lydelse}
  \setcounter{section}{8}
  \setcounter{subsection}{4}
  \item[] ---
  \item Medlem i studienämnden har närvaro-, yttrande-, förslags- och rösträtt på studienämndsmöte.
  \item Sektionsmedlem har närvaro-, yttrande- och förslagsrätt på studienämndsmöte.
  \setcounter{subsection}{3}
  \setcounter{enumi}{0}   
  \item Protokoll skall föras vid studienämndsmöte, justeras av en studienämndsledamot och anslås senast två läsveckor efter mötet.
  \switchcolumn
  \setcounter{section}{7}
  \setcounter{subsection}{1}
  \item \emph{Studienämnden sammanträder minst tre gånger per läsperiod på kallelse av studienämndsordförande.}
  \item \emph{Alla medlemmar äger} närvaro-, yttrande- och förslagsrätt på studienämndsmöte.
  \item \emph{Endast studienämndens ledamöter äger dessutom rösträtt på studienämndsmöte.}
  \item Protokoll \emph{förs} vid studienämndsmöte, justeras av en studienämndsledamot och anslås senast två läsveckor efter mötet.
\end{lydelse}
\setcounter{section}{7}
\setcounter{subsection}{2}

\section{Kommittéer}
Syftet är redaktionellt.
Alla avsnitt utöver definitionen har införlivats i övriga kapitel.
Definitionen av förtroendeposter lyfts från reglemente.

Härigenom föreskrivs i fråga om stadgans kapitel 9
\begin{dels}
\item att kapitlet omnumreras till 8,
\item att kapitlet i sin helhet ska ha följande lydelse.
\end{dels}

\begin{lydelse}
\setcounter{section}{9}
\setcounter{subsection}{1}
  \item Sektionskommitté på sektionen skall ha ett i reglemente fastställt antal förtroendeposter.
  \item Sektionskommitté på sektionen kan ha ett i reglementet fastställt antal övriga medlemmar.
  \item Förtroendeposter tillsätts av sektionsmötet på förslag av valberedningen.
  \item Övriga medlemmar fastslås i enlighet med reglementet.
  \item Sektionskommitté skall verka för sektionens bästa och ha en i reglementet fastslagen uppgift.
\switchcolumn
  \setcounter{section}{8}
  \item \emph{Kommittéer verkar för att uppfylla sektionens ändamål och ska vara dess representanter.}
  \item \emph{Kommittéer fastställs och förtecknas i reglemente, med syfte, sammansättning och åligganden.}
  \item \emph{Sammansättningen ska åtminstone innehålla ordförande och kassör, vilka är förtroendeposter och ekonomiskt ansvariga.}
  \item[] \emph{(ersätts av \S\ref{4.x:tillsätt} och \S\ref{13.x:valBfp})}
\end{lydelse}
\setcounter{section}{8}

\section{Sektionsföreningar}
Syftet är redaktionellt.
Alla avsnitt utöver definitionen har införlivats i övriga kapitel.

Härigenom föreskrivs i fråga om stadgans kapitel 10
\begin{dels}
\item att kapitlet omnumreras till 9,
\item att kapitlet i sin helhet ska ha följande lydelse.
\end{dels}

\begin{lydelse}
\setcounter{section}{10}
\setcounter{subsection}{1}
  \item Sektionsförening på sektionen skall ha ett i reglemente fastställt antal förtroendeposter.
  \item Övriga medlemmar tecknas i reglementet.
  \item Medlemmar tillsätts enligt reglementet.
  \item Sektionsförening skall verka för sektionens bästa och ha en i reglementet fastslagen uppgift.
\switchcolumn
  \setcounter{section}{9}
  \item \emph{Sektionsföreningar verkar för att uppfylla sektionens ändamål.}
  \item \emph{Sektionsföreningar fastställs och förtecknas i reglemente, med sammansättning och åligganden.}
  \item[] \emph{(ersätts av \S\ref{4.x:tillsätt})}
\end{lydelse}
\setcounter{section}{9}

\section{Funktionärer}
Syftet är mestadels redaktionellt.
Alla avsnitt utöver definitionen har införlivats i övriga kapitel.
Stadgan öppnar för direkt tillsättning av funktionärer om reglemente så föreskriver (och gör dumvästen stadgeenlig!).

Härigenom föreskrivs i fråga om stadgans kapitel 11
\begin{dels}
\item att kapitlet omnumreras till 10,
\item att kapitlet i sin helhet ska ha följande lydelse.
\end{dels}

\begin{lydelse}
  \setcounter{section}{11}
  \setcounter{subsection}{1}
\item Sektionsfunktionär på sektionen ska finnas enligt reglemente.
\item Sektionsfunktionär tillsätts av sektionsmötet.
\item Sektionsförening skall verka för sektionens bästa och ha en i reglementet fastslagen uppgift.
\item Sektionsfunktionärer betraktas ej som förtroendevalda om ej annorlunda specificerats i regle-
mentet.
\switchcolumn
\setcounter{section}{10}
\item \emph{Funktionärer verkar för att uppfylla sektionens ändamål genom enskild uppgift.}
\item \emph{Funktionärer fastställs och förtecknas i reglemente, med sammansättning och åligganden.}
\item \emph{Om reglemente så föreskriver kan funktionär tillsättas på annat sätt än enligt \S\ref{4.x:tillsätt}.}
\item[] \emph{(redundant)}
\end{lydelse}
\setcounter{section}{10}

\section{Intresseföreningar}
Syftet är rent redaktionellt.
Många bestämmelser har införlivats i övriga kapitel.

Härigenom föreskrivs i fråga om stadgans kapitel 12
\begin{dels}
\item att kapitlet omnumreras till 11,
\item att kapitlet i sin helhet, inklusive avsnitt, ska ha följande lydelse.
\end{dels}

\begin{lydelse}
  \setcounter{section}{12}
  \setcounter{subsection}{1}
  \item En intresseförening på sektionen är en sammanslutning av sektionsmedlemmar med ett gemensamt intresse.
  \item Intresseföreningen skall verka för sektionens bästa och ha en i reglementet fastslagen uppgift.
\switchcolumn
  \setcounter{section}{11}
  \item Intresseförening är en sammanslutning av sektionsmedlemmar med gemensamt intresse \emph{som verkar enligt sektionens ändamål}.
\switchcolumn*
  \setcounter{subsection}{3}
  \setcounter{enumi}{0}
  \item Intresseförenings status beviljas och fråntas av sektionsmötet.
  
  \setcounter{subsection}{8}
  \setcounter{enumi}{0}
  \item Sektionens intresseföreningar är listade i reglementet.
\switchcolumn
  \item Intresseföreningsstatus beviljas och fråntas av sektionsmötet.
  \item Intresseföreningar \emph{förtecknas} i reglemente.
\end{lydelse}
\setcounter{section}{11}
\setcounter{subsection}{0}

\subsection{Krav}
\begin{lydelse}
  \setcounter{section}{12}
  \setcounter{subsection}{2}
  \item Intresseförening skall ha en av sektionsstyrelsen godkänd stadga.
  \item Intresseförening skall ha en styrelse. Minst två tredjedelar av styrelsledamöterna skall vara sektionsmedlemmar.
  \setcounter{subsection}{7}
  \item Varje sektionsmedlem skall ha rätt till medlemskap. Dock kan föreningsmedlem som motverkar föreningens syfte uteslutas.
  \item Styrelsen kan besluta om medlemskap för någon som ej är medlem i Fysikteknologsektionen så länge dylika medlemmar ej utgör mer än hälften av föreningens medlemmar.
  \switchcolumn
  \setcounter{section}{11}
  \item Intresseförening \emph{ska} ha en av sektionsstyrelsen godkänd stadga.
  \item Intresseförening \emph{ska} ha en styrelse \emph{bestående till minst \sfrac{2}{3} av} sektionsmedlemmar.
  \item Varje sektionsmedlem \emph{ska äga} rätt till medlemskap i \emph{intresseförening}.
    Dock \emph{får} medlem som motverkar \emph{intresse}föreningens syfte uteslutas.
  \item \emph{Sektionsmedlemmar ska utgöra minst hälften av intresseförenings medlemmar.}
\end{lydelse}
\setcounter{section}{11}
\setcounter{subsection}{1}

\subsection{Ekonomi och revision}
\begin{lydelse}
  \setcounter{section}{12}
  \setcounter{subsection}{6}
  \item Intresseförening skall ha en fristående ekonomi.
  \item Intresseförenings verksamhet och ekonomi granskas, utöver deras egna revisorer, av sektionens revisorer.
  \item Senast tre veckor efter att årsmötet godkänt verksamhetsberättelsen för föregående år skall denna lämnas till sektionsstyrelsen och frågan om ansvarsfrihet för intresseföreningens aktuella styrelse skall behandlas.
  \item Senast tre veckor efter att intresseföreningens revisors revisionsberättelse godkänts av årsmötet skall av sektionsrevisorerna begärt material lämnas till dessa.
\switchcolumn
  \setcounter{section}{11}
  \item Intresseförening \emph{ska} ha \emph{ekonomi fristående från sektionen}.
  \item Intresseförening\emph{ar}s verksamhet och ekonomi granskas, utöver \emph{egen revision}, av sektionens revisorer.
  \item \emph{Efter att intresseförenings årsmöte behandlat förgående verksamhetsår ska alla handlingar inom tre veckor tillsändas sektionsstyrelsen och sektionens revisorer.}
  \item \emph{Sektionsstyrelsen prövar frågan om ansvarsfrihet för intresseföreningens styrelse efter fullgjort verksamhetsår.}
\end{lydelse}
\setcounter{section}{11}
\setcounter{subsection}{2}

\section{Talmanspresidiet}
Syftet är att ge Talmanspresidiet som det organ det är ett eget kapitel.
Det tydliggör hur stadga och reglemente speglar sektionens struktur.
Presidiets sammansättning lyfts till stadgan, eftersom reglerna om organisation och ansvar ändå nämner vice talman.
Notera att valbarhetshindren flyttats upp till tidigare kapitel.

Härigenom föreskrivs i fråga om stadgans avsnitt 5.9
\begin{dels}
  \item att avsnittet ska omformas och flyttas till kapitel 12,
  \item att det nya kapitlet i sin helhet ska ha följande lydelse.
\end{dels}

\begin{lydelse}
  \setcounter{section}{5}
  \setcounter{subsection}{9}
  \item Talmannen väljs av sektionsmötet. Talmannen får ej vara medlem av sektionsstyrelsen.
  \item Talmannen ska leda sektionsmötet i överensstämmelse med stadgan, reglementet samt av sektionsmötet fastslagen mötesordning.
  \item Talmannen behöver ej vara medlem av sektionen.
  \item Talmanspresidiet består av talman och övriga ledamöter i enlighet med reglementet.
\switchcolumn
  \item \emph{Talmanspresidiet leder sektionsmötet och förvaltar sektionens demokratiska grund.}
  \item Talmanspresidiet består av talman\emph{, vice talman och sekreterare.}
\end{lydelse}
\setcounter{section}{12}

\section{Valberedningen}
Syftet är redaktionellt.
En del redundans (t.ex. valprocess, ansvarsbeskrivning) tas bort.
Jävsfrågor sänks ned till reglemente.
Proceduren för val av ordförande och vice ordförande sänks ned till reglemente.

Härigenom föreskrivs i fråga om stadgans kapitel 6
\begin{dels}
  \item att kapitlet omnumreras till 13,
  \item att kapitlet i sin helhet, inklusive avsnitt, ska ha följande lydelse.
\end{dels}

\begin{lydelse}
  \setcounter{section}{6}
  \setcounter{subsection}{1}
  \item Sektionens valberedning skall väljas av sektionsmötet. Valberedningen skall agera oberoende.
  \item Ledamot i valberedningen skall ej vara jävig gentemot dem den valbereder.
  \setcounter{subsection}{2}
  \setcounter{enumi}{0}
  \item Valberedningen består av 3---7 ledamöter varav två internt väljs till ordförande respektive vice ordförande.
    % dagens trigger att det är ett långt tankstreck >:(
  \item Ledamot i valberedningen får ej vara medlem i sektionsstyrelsen eller någon kommitté eller nämnd som har representant i sektionsstyrelsen.
  \item I det fall då valberedningen ej är fulltalig och sektionsstyrelsen så anser lämpligt har sektionsstyrelsen möjlighet att temporärt välja kvarstående valberedningsposter inför varje valberedningsprocess. Dessa temporära medlemmar får inneha post i sektionsstyrelsen.
  \setcounter{subsection}{4}
  \setcounter{enumi}{0}  
  \item Valberedningen ansvarar för nomineringar till samtliga poster i sektionsstyrelsen. Därutöver ansvarar valberedningen för nomineringar till förtroendeposter och övriga poster på sektionen enligt reglementet.

  \switchcolumn
  \item Valberedningen ansvarar för \emph{oberoende} nomineringar till \emph{samtliga förtroendeposter, inklusive sektionsstyrelsen,} och övriga poster enligt reglemente. \label{13.x:valBfp}
  \item Valberedningen består av 3--7 ledamöter varav två internt väljs till ordförande respektive vice ordförande.
  \item \emph{Vid vakans kan sektionsstyrelsen tillförordna valberedningsledamöter till nästa sektionsmöte. }

  \item[] \emph{(ersätts av \S\S\ref{4.x:valbar.dubbel}--\ref{4.x:valbar.ober})}
\end{lydelse}
\setcounter{section}{13}
\setcounter{subsection}{0}

\subsection{Nomineringsbeslut}
\begin{lydelse}
  \setcounter{section}{6}
  \setcounter{subsection}{3}
  \item Vid valberedningens första sammanträde skall ordförande och vice ordförande väljas. Minst 3 av valberedningens ledamöter måste då vara närvarande.
  \item När valberedningen sammanträder har max två medlemmar ur berörd styrelse, nämnd, kommitté, sektions\-förening eller funktionär närvaro-, för\-slags-, yttrande- och rösträtt. Valberedningens ordförande är ordförande samt sammankallande för valberedningen.
  \item Valberedningen är beslutsförig om dess ordförande, eller vice ordförande, minst två ytterligare
ledamöter samt minst en representant för berörd styrelse, nämnd, kommitté eller funktionär
är närvarande.
  \setcounter{subsection}{5}
  \setcounter{enumi}{0}
  \item Valberedningens nomineringar skall anslås i enlighet med reglementet.
  \switchcolumn
  \item[] \emph{(ersätts av reglemente)}
  \item \emph{Valberedningen sammanträder för nominering på kallelse av ordförande. Upp till två representanter för organ till vilket ska nomineras får adjungeras.}
  \item Valberedningen är beslutsför \emph{om minst tre ledamöter är närvarande inklusive ordförande eller vice ordförande}, samt minst \emph{en adjungerad representant för organ till vilket ska nomineras.}
  \item Valberedningens nomineringar anslås i enlighet med reglemente.
\end{lydelse}
\setcounter{section}{13}
\setcounter{subsection}{1}

\section{Ekonomi och revision}
Syftet är att konsolidera bestämmelser om ekonomi och revision till ett enda lättillgängligt ställe.
Detta görs genom att omstrukturera kapitlet för revision.
Många bestämmelser överförs från andra kapitel.

Härigenom föreskrivs i fråga om stadgans kapitel 13
\begin{dels}
  \item att kapitlet omnumreras till 14,
  \item att kapitlets rubrik ska lyda ''Ekonomi och revision'',
  \item att kapitlet i sin helhet, inklusive avsnitt, ska ha följande lydelse.
\end{dels}

\begin{lydelse}
  \item[] ---

  \item[] (se \S 7.4.1)
  
  \switchcolumn

  \item \emph{Sektionens organisationsnummer är 857208-8477.}

  \item \emph{Sektionens firma tecknas av sektionsordförande och sektionskassör var för sig.}
\end{lydelse}

\subsection{Redovisning och ansvarsfrihet}
\begin{lydelse}
  \item[] (se \S 8.7.2, \S 9.3.4, \S 9.4.2, \S 10.3.3, \S 10.4.2, \S 11.4.1)
  \item[] (se \S 13.1.9)
  \item[] (se \S 13.3.1)
  \switchcolumn
  \setcounter{enumi}{0}  
  \item \emph{Sektionsstyrelsen upprättar gemensamt bokslut för sektionen, innefattande alla organ utom intresseföreningar.}
  \item \emph{Studienämnden, kommittéer och övriga organ enligt reglemente upprättar dessutom egen löpande redovising och delbokslut.}
  \item \emph{Sektionsstyrelsen handhar annars den löpande redovisningen.}
  \item \emph{Bokslut och årsredovisning enligt detta avsnitt presenteras vid första ordinarie sektionsmöte efter verksamhetsårets slut.
      Handlingarna tillställs revisorer, sektionsstyrelsen och talmanspresidiet senast 11 läsdagar innan.} \label{14.x:tillställa}
  \item \emph{Frågan om ansvarsfrihet för förtroendeposter behandlas då av sektionsmötet.}
\end{lydelse}
% budget
% bokslut

\subsection{Revisor}
\begin{lydelse}
  \setcounter{section}{13}
  \setcounter{subsection}{1}
  \item Sektionsmötet utser två lekmannarevisorer med uppgift att granska sektionens verksamhet och ekonomi.
  \item Revisor skall vara myndig.
  \item Revisor skall ej vara jävig gentemot de de granskar.
  \item Revisor kan ej vara medlem av sektionsstyrelsen under sitt verksamhetsår.
  \item Revisor kan ej inneha ekonomiskt ansvar inom sektionen.
  \item En revisor kan ej granska ekonomin för en kommitté, sektionsförening, nämnd eller sektionsstyrelse för ett år då denne var medlem av denna eller direkt påföljande år.
  \item I de fall då samtliga verksamheter inte kan granskas av ordinarie revisorer kan extra revisor väljas för att granska dessa verksamheter.
  \item Revisor behöver ej vara medlem av sektionen.
 
  \switchcolumn
  \setcounter{enumi}{0}  
  \item \emph{Lekmannarevisorerna granskar oberoende och kontinuerligt sektionens verksamhet och ekonomi.}

  \item \emph{Revisor får ej granska verksamhetsår för organ, då denna var ledamot det året eller föregående, eller annat jäv föreligger.} \label{14.x:revisorjäv}

  \item \emph{Två ordinare revisorer väljs. Om revisorer enligt \S\ref{14.x:revisorjäv} inte kan granska samtliga organ väljs extra revisor för dessa.}

  \item[] \emph{(ersätts av \S\S\ref{4.x:valbar.dubbel}--\ref{4.x:valbar.revisormyndig})}

  \switchcolumn*
  \item Räkenskaper och övriga handlingar skall tillställas revisorerna senast 10 läsdagar före ordinarie sektionsmöte.
    \setcounter{subsection}{1}
    \setcounter{enumi}{0}  
  \item Det åligger revisorerna att anslå revisionsberättelser senast tre läsdagar före ordinarie sektionsmöte.
  \item Revisionsberättelsen skall innehålla yttrande i fråga om ansvarsfrihet för berörda personer.
  \item Det åligger revisorerna att under året kontinuerligt granska räkenskaper och förvaltning.
 
  \switchcolumn
  \item[] \emph{(ersätts av \S\ref{14.x:tillställa})}
 
  \item \emph{Revisor åligger att avge och anslå revisionberättelse, inklusive yttrande angående ansvarsfrihet, senast 4 läsdagar innan sektionsmöte.}

\end{lydelse}
\setcounter{section}{14}
\setcounter{subsection}{2}
\setcounter{enumi}{0}

\section{Styrdokument}
Syftet är dels redaktionellt, dels att förtydliga ändringsförfarandet.
En gammal tvist om huruvida ändringsyrkanden kan göras och påverkar stageändringen löses.
Eftersom talmannen har rätt att förordna vilka sektionsmöten är ordinarie (det är de vilka inte kallas på särskild anmodan) läggs tidskrav till för stadgeändring.

Härigenom föreskrivs i fråga om stadgans kapitel 14
\begin{dels}
  \item att kapitlet omnumreras till 15,
  \item att kapitlet i sin helhet, inklusive avsnitt, ska ha följande lydelse.
\end{dels}

\begin{lydelse}
  \setcounter{section}{14}
  \setcounter{subsection}{1}
  \item Förutom denna stadga finns reglemente och övriga styrdokument i enlighet med reglementet.
  \setcounter{subsection}{6}
  \setcounter{enumi}{0}  
  \item Originalstadgar tillhandahas av sektionsstyrelsen.
  \switchcolumn
  \item \emph{Utöver denna stadga ska finnas reglemente och övriga styrdokument förtecknade däri.}
  \item Originalstadgar tillhandahas av sektionsstyrelsen.
\end{lydelse}
\setcounter{section}{15}
\setcounter{subsection}{0}

\subsection{Ändring}
\begin{lydelse}
  \setcounter{subsection}{2}
  \item Ändring av eller tillägg till denna stadga inklusive dess bilagor kan endast göras av sektionsmötet och om minst 2/3 av de närvarande är om beslutet ense under två på varandra följande ordinarie sektionsmöten och den föreslagna lydelsen varit anslagen tillsammans med kallelsen.
  \item Ändring av eller tillägg till denna stadga skall godkännas av Chalmers Studentkårs styrelse.
  \setcounter{subsection}{3}
  \setcounter{enumi}{0}  
  \item Ändring av eller tillägg till sektionens reglemente inklusive dess bilagor kan endast göras av sektionsmötet och om minst 2/3 av de närvarande är om beslutet ense.
  \switchcolumn
  \item \emph{Ändring av stadga eller reglemente görs av sektionsmötet. Kallelsen ska tydligt ange att fråga om ändring behandlas tillsammans med fullständigt förslag.}
  \item \emph{Beslut om stadgeändring fattas med \sfrac{2}{3} majoritet.
      Fastställande av beslutet görs med \sfrac{2}{3} majoritet på nästkommande sektionsmöte, dock tidigast efter fyra läsveckor förflutit.
      Lydelsen får ej ändras vid fastställandet.}
  \item \emph{Stadgeändring ska dessutom fastställas av Chalmers studentkårs styrelse.}
  \item \emph{Beslut om reglementesändring fattas med \sfrac{2}{3} majoritet.}
\end{lydelse}
\setcounter{section}{15}
\setcounter{subsection}{1}

\subsection{Protokoll och tillkännagivande}
\begin{lydelse}
  \setcounter{section}{14}
  \setcounter{subsection}{5}
  \item Protokoll som förs i sektionens olika organ skall innehålla anteckningar om ärendenas art, samtliga ställda och ej återtagna yrkanden, beslut samt särskilda yttranden och reservationer.
  \item Beslut som fattas inom sektionen och berör Chalmers Studentkår i dess helhet skall meddelas Chalmers Studentkårs styrelse.
  \item Sektionens officiella kommunikationskanaler utgörs av sektionens hemsida samt sektionens officiella anslagstavla.
  \item Meddelanden och beslut är behörigt anslagna då de anslås via någon av sektionens officiella kommunikationskanaler.
  \switchcolumn
  \item Protokoll som förs i sektionens organ ska innehålla anteckningar om ärendenas art, samtliga ställda och ej återtagna yrkanden, beslut samt särskilda yttranden och reservationer.
  \item Protokoll och beslut \emph{tillkännages genom behörigt anslag enligt reglemente.}
  \item Beslut \emph{fattade} inom sektionen som berör Chalmers studentkår i dess helhet \emph{meddelas} Chalmers studentkårs styrelse.
\end{lydelse}
\setcounter{section}{15}
\setcounter{subsection}{2}

\subsection{Tolkningstvister}
\begin{lydelse}
  \setcounter{section}{14}
  \setcounter{subsection}{4}
  \item Uppstår tolkningstvist om dessa stadgars tolkning skall frågan hänskjutas till sektionens inspektor.
  \item Vid konflikt med reglemente eller Fysikteknologsektionens övriga styrdokument har stadgan företräde.
  \switchcolumn
  \item \emph{Vid konflikt med annat styrdokument har stadga och reglemente företräde. Vid konflikt med reglemente har stadga företräde.}
  \item \emph{Vid tvist om stadgans tolkning avgörs frågan av inspektor.}
\end{lydelse}

\section{Upplösning}
Syftet är redaktionellt.
Kraven för upplösning justeras enligt ovanstående observationer.
En ideell förening är en juridisk person med begränsat personligt ansvar; det finns ingen anledning att påtvinga ChS att åta sig sektionens skulder om den t.ex. skulle upplösas till följd av förestående konkurs.
Det är också i linje med Skatteverkets instruktioner om avveckling av ideell förening.
Med det sagt hindrar inget oss från att stadga disposition av eventuellt överskott.
Dessutom bör vi bevara vår relativt långa historia sett till övriga sektioner och se till att arkivera handlingar om det värsta händer.

Härigenom föreskrivs i fråga om stadgans kapitel 15
\begin{dels}
  \item att kapitlet omnumreras till 16,
  \item att kapitlets rubrik ska lyda ''Upplösning'',
  \item att kapitlet i sin helhet ska ha följande lydelse.
\end{dels}
\begin{lydelse}
  \item Sektionen upplöses genom beslut på två på varandra följande sektionsmöten med minst 3/4 majoritet och minst 25 bifallande medlemmar.
  \item Om sektionsmötet beslutar att upplösa sektionen skall samtliga dess tillgångar och skulder, som framgår av upprättad balansräkning, i och med upplösning tillfalla Chalmers Studentkår.
  \item I det fall medlen utgörs av tillgångar skall Chalmers Studentkår fondera och förvalta dessa till ny sektion bildats för studerande på utbildningsprogrammet för Teknisk fysik och/eller Teknisk matematik eller motsvarande.
  \switchcolumn
  \item Sektionen upplöses genom beslut på två sektionsmöten, \emph{mellan vilka minst 4 läsveckor förflutit}, med \sfrac{3}{4} majoritet och minst 25 bifallande medlemmar.
  \item \emph{Vid upplösning avvecklar sektionsstyrelsen snarast sektionen.
      Eventuellt överskott tillfaller Chalmers studentkår.
      Sektionens handlingar arkiveras på Göteborgs föreningsarkiv.}
  \item \emph{Chalmers studentkår förvaltar eventuellt överskott i fond fram tills} ny sektion bildats för studerande på utbildningsprogrammet för Teknisk fysik och/eller Teknisk matematik eller motsvarande.
\end{lydelse}

\clearpage
\part{Reglemente}

\end{document}
